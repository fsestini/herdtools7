\usepackage[T1]{fontenc} % makes names with underscore searchable in the PDF.
\usepackage{amsmath}  % Classic math package
\usepackage{amssymb}  % Classic math package
\usepackage{mathtools}  % Additional math package
\usepackage{amssymb}  % Classic math package
\usepackage{mathtools}  % Additional math package
\usepackage{url}  % Automatically escapes urls
\usepackage{hyperref}  % Insert links inside pdfs
\hypersetup{
    colorlinks=true,
    linkcolor=blue,
    filecolor=magenta,
    urlcolor=cyan
}

% The \hypertarget macro suffers from having the corresponding \hyperlink
% point one line below where they should, which is a known issue.
% To fix this, the next two versions of \hypertarget use \Hy@raisedlink.
% \mathhypertarget{} is suitable for math environments,
% whereas the \texthypertarget{} is suitable for text environments.
\makeatletter
\newcommand\mathhypertarget[1]{\Hy@raisedlink{\hypertarget{#1}}} % DO NOT LINT
\newcommand\texthypertarget[1]{\Hy@raisedlink{\hypertarget{#1}{}}} % DO NOT LINT
\makeatother

\usepackage{listings}
\lstdefinelanguage{ASL}
{
    morekeywords={
      accessor,
      AND,
      array,
      as,
      assert,
      begin,
      bit,
      bits,
      boolean,
      case,
      catch,
      collection,
      config,
      constant,
      DIV,
      DIVRM,
      do,
      downto,
      else,
      elsif,
      end,
      enumeration,
      XOR,
      exception,
      FALSE,
      for,
      func,
      getter,
      if,
      impdef,
      implementation,
      IN,
      integer,
      let,
      looplimit,
      MOD,
      NOT,
      noreturn,
      of,
      OR,
      otherwise,
      pass,
      pragma,
      print,
      println,
      pure,
      readonly,
      real,
      record,
      recurselimit,
      repeat,
      return,
      setter,
      string,
      subtypes,
      then,
      throw,
      to,
      try,
      TRUE,
      type,
      ARBITRARY,
      unreachable,
      until,
      var,
      when,
      where,
      while,
      with
    },
    keywordstyle=\color{red},
    morecomment=[l][\color{blue}]{//},
    morecomment=[s][\color{blue}]{/*}{*/},
    morestring=[b][\color{cyan}]",
    morestring=[d][\color{cyan}]',
}
\lstset{
  language=ASL,
  basicstyle=\ttfamily\scriptsize,
  showspaces=false,
  showstringspaces=false,
  breaklines=true,
  backgroundcolor=\color{yellow!10}
}

% "Dark theme"
\usepackage{xcolor}
% \pagecolor[rgb]{0.2,0.2,0.2}
% \color[rgb]{0.5,0.5,0.5}

\usepackage[inline]{enumitem}  % For inline lists
\usepackage[export]{adjustbox}  % For centering too wide figures
\usepackage[export]{adjustbox}  % For centering too wide figures
\usepackage{mathpartir}  % For deduction rules and equations paragraphs
\usepackage{comment}
\usepackage{fancyvrb}
\usepackage[
  % Even pages have notes on the left, odd on the right
  twoside,
  % Notes on the right, should be less than outer
  marginparwidth=100pt,
  % margins
  top=4.5cm, bottom=4.5cm, inner=3.5cm, outer=4.5cm
  % To visualize:
  % showframe
]{geometry}
\input{ifempty}
\input{ifcode}
\input{control}

\newcommand\aslrefterm[0]{\href{https://github.com/herd/herdtools7/tree/master/asllib}{aslref}}
\newcommand\linuxbashshellterm[0]{Linux bash shell}

\fvset{fontsize=\small}

\usepackage{enumitem}
\renewlist{itemize}{itemize}{20}
\setlist[itemize,1]{label=\textbullet}
\setlist[itemize,2]{label=\textasteriskcentered}
\setlist[itemize,3]{label=\textendash}
\setlist[itemize,4]{label=$\triangleright$}
\setlist[itemize,5]{label=+}
\setlist[itemize,6]{label=\textbullet}
\setlist[itemize,7]{label=\textasteriskcentered}
\setlist[itemize,8]{label=\textendash}
\setlist[itemize,9]{label=$\triangleright$}
\setlist[itemize,10]{label=+}
\setlist[itemize,11]{label=\textbullet}
\setlist[itemize,12]{label=\textasteriskcentered}
\setlist[itemize,13]{label=\textendash}
\setlist[itemize,14]{label=\textendash}
\setlist[itemize,15]{label=$\triangleright$}
\setlist[itemize,16]{label=+}
\setlist[itemize,17]{label=\textbullet}
\setlist[itemize,18]{label=\textasteriskcentered}
\setlist[itemize,19]{label=\textendash}
\setlist[itemize,20]{label=\textendash}

\newcommand\BackupOriginalAST[1]{}
\newcommand\BackupOriginalType[1]{}
\newcommand\BackupOriginalRelation[1]{}
\newcommand\MarkedIgnoredRelationDefinition[1]{#1} % DO NOT LINT
\newcommand\PendingRelationDefinition[1]{#1} % DO NOT LINT

\newcommand\ChapterOutline[0]{\paragraph{Outline} The rest of this chapter is organized as follows:}
\newcommand\FormalRelationsDef[1]{\section{Formal Relations for #1\label{sec:Formal Relations for #1}}}
\newcommand\FormalRelationsRef[1]{\secref{Formal Relations for #1}}
\newcommand\SyntaxDef[1]{\section{Syntax of #1\label{sec:Syntax of #1}}}
\newcommand\SyntaxRef[1]{\secref{Syntax of #1}}
\newcommand\AbstractSyntaxDef[1]{\section{Abstract Syntax of #1\label{sec:Abstract Syntax of #1}}}
\newcommand\AbstractSyntaxRef[1]{\secref{Abstract Syntax of #1}}
\newcommand\TypeRulesDef[1]{\section{Type Rules for #1\label{sec:Type Rules for #1}}}
\newcommand\TypeRulesRef[1]{\secref{Type Rules for #1}}
\newcommand\SemanticsRulesDef[1]{\section{Dynamic Semantics Rules for #1\label{sec:Dynamic Semantics Rules for #1}}}
\newcommand\SemanticsRulesRef[1]{\secref{Dynamic Semantics Rules for #1}}
\newcommand\lrmcomment[1]{}
\newcommand\textfunc[1]{\textit{#1}}
\newcommand\textastlabel[1]{\textsc{#1}}
\newcommand\ProseParagraph[0]{\subsubsection{Prose}}
\newcommand\FormallyParagraph[0]{\subsubsection{Formally}}
\newcommand\CaseDef[1]{\textsc{#1}}
\newcommand\CaseName[1]{case \textsc{#1}}
\newcommand\AllApply[0]{All of the following apply:}
\newcommand\AllApplyCase[1]{All of the following apply (\CaseDef{#1}):}
\newcommand\OneApplies[0]{One of the following applies:}
\newcommand\Proseeqdef[2]{define #1 as #2}
\newcommand\ProseEqdef[2]{Define #1 as #2}
\newcommand\ExampleDef[1]{\hypertarget{example-#1}{}\subsubsection{Example: #1}} % DO NOT LINT
\newcommand\ExampleRef[1]{\hyperlink{example-#1}{Example: #1}} % DO NOT LINT

\ifcode
% First argument is \<rule>Begin, second is \<rule>End, third is the file name.
% Example: for SemanticsRule.Lit, use the following:
% \CodeSubsection{\LitBegin}{\LitEnd}{../Interpreter.ml}
\newcommand\CodeSubsection[3]{\subsection{Code} \VerbatimInput[firstline=#1, lastline=#2]{#3}}
\else
\newcommand\CodeSubsection[3]{}
\fi

\definecolor{verylightgray}{RGB}{240,240,240}

\newcommand\ASLListing[3]{
\begin{center}
\lstinputlisting[caption=#1\label{listing:#2}]{#3}
\end{center}
}

\ifcode
% First argument is \<rule>Begin, second is \<rule>End, third is the file name.
% Example: for SemanticsRule.Lit, use the following:
% \CodeSubsubsection{\LitBegin}{\LitEnd}{../Interpreter.ml}
\newcommand\CodeSubsubsection[3]{\subsubsection{Code} \VerbatimInput[firstline=#1, lastline=#2]{#3}}
\else
\newcommand\CodeSubsubsection[3]{}
\fi

%%%%%%%%%%%%%%%%%%%%%%%%%%%%%%%%%%%%%%%%%%%%%%%%%%
% Typesetting macros
\newtheorem{definition}{Definition}
\newtheorem{example}{Example}
\newcommand\ConventionDef[1]{\paragraph{Convention.#1:\label{sec:Convention.#1}}}
\newcommand\RequirementDef[1]{\paragraph{Guide.#1\label{Guide.#1}}}
\newcommand\RequirementDefExample[1]{\emph{Guide.#1}}
\newcommand\SyntacticSugarDef[1]{\paragraph{Guide.SyntacticSugar.#1\label{Guide.SyntacticSugar.#1}}}
\newcommand\SyntacticSugarExample[1]{\emph{Guide.SyntacticSugar.#1}}
\newcommand\LexicalRuleDef[1]{\subsubsection{LexicalRule.#1\label{sec:LexicalRule.#1}}}
\newcommand\ASTRuleDef[1]{\subsubsection{ASTRule.#1\label{sec:ASTRule.#1}}}
\newcommand\ASTRuleDefExample[1]{\emph{ASTRule.#1}}
\newcommand\TypingRuleDef[1]{\subsubsection{TypingRule.#1\label{sec:TypingRule.#1}}}
\newcommand\TypingRuleDefExample[1]{\emph{TypingRule.#1}}
\newcommand\SemanticsRuleDef[1]{\subsubsection{SemanticsRule.#1\label{sec:SemanticsRule.#1}}}
\newcommand\SemanticsRuleDefExample[1]{\emph{SemanticsRule.#1}}
\newcommand\listingref[1]{Listing~\ref{listing:#1}}
\newcommand\taref[1]{Table.~\ref{ta:#1}}
\newcommand\defref[1]{Definition~\ref{def:#1}}
\newcommand\secref[1]{Section~\ref{sec:#1}}
\newcommand\chapref[1]{Chapter~\ref{chap:#1}}
\newcommand\ASTRuleRef[1]{\nameref{sec:ASTRule.#1}}
\newcommand\ASTRuleCaseRef[2]{\nameref{sec:ASTRule.#1}.\textsc{#2}}
\newcommand\RequirementRef[1]{\nameref{Guide.#1}}
\newcommand\LexicalRuleRef[1]{\nameref{sec:LexicalRule.#1}}
\newcommand\TypingRuleRef[1]{\nameref{sec:TypingRule.#1}}
\newcommand\SemanticsRuleRef[1]{\nameref{sec:SemanticsRule.#1}}
\newcommand\ie{i.\,e.}
\newcommand\eg{e.\,g.}
\newcommand\wrappedline[0]{{\hyperlink{def-wrapline}{\color{red}\hookrightarrow}}}
\newcommand\commonprefixline[0]{{\hyperlink{def-commonprefixline}{\color{cyan}{\text{\tiny******** common prefix ********}} }}}
\newcommand\commonsuffixline[0]{{\hyperlink{def-commonsuffixline}{\color{cyan}{\text{\tiny******** common suffix ********}} }}}
\newcommand\stdlibfunc[1]{standard library function \texttt{#1}}

%%%%%%%%%%%%%%%%%%%%%%%%%%%%%%%%%%%%%%%%%%%%%$%%%%%
%% Mathematical notations and Inference Rule macros
%%%%%%%%%%%%%%%%%%%%%%%%%%%%%%%%%%%%%%%%%%%%$%%%%%%
\usepackage{relsize}
\let\OldLand\land
\renewcommand\land[0]{\hyperlink{def-land}{\OldLand}}
\let\OldLor\lor
\renewcommand\lor[0]{\hyperlink{def-lor}{\OldLor}}
\let\OldNeg\neg
\renewcommand\neg[0]{\hyperlink{def-neg}{\OldNeg}}
\let\OldTriangleq\triangleq
\let\OldEmptyset\emptyset
\renewcommand\emptyset[0]{\hyperlink{constant-emptyset}{\OldEmptyset}}
\let\OldBot\bot
\renewcommand\bot[0]{\hyperlink{constant-bot}{\OldBot}}
\renewcommand\triangleq[0]{\hyperlink{def-triangleq}{\OldTriangleq}}
\newcommand\cartimes[0]{\hyperlink{def-cartimes}{\times}}

\newcommand\eqname[0]{\hyperlink{def-deconstruction}{\stackrel{\mathsmaller{\mathsf{is}}}{=}}}
\newcommand\eqdef[0]{\hyperlink{def-eqdef}{:=}}
\newcommand\overname[2]{\overbrace{#1}^{#2}}
\newcommand\overtext[2]{\overbracket{#1}^{\text{#2}}}
\newcommand\emptyfunc[0]{\hyperlink{def-emptyfunc}{{\emptyset}_\lambda}}
\newcommand\restrictfunc[2]{{#1}\hyperlink{def-restrictfunc}{|}_{#2}}

\newcommand\emptylist[0]{\hyperlink{constant-emptylist}{[\ ]}}
\newcommand\Proseemptylist[1]{#1 is the empty list}
\newcommand\choice[3]{\hyperlink{def-choice}{\textsf{choice}}(#1,#2,#3)}
\newcommand\choicename[0]{\hyperlink{def-choice}{\textfunc{choice}}}
\newcommand\ifthenelse[3]{
  \left\{\begin{array}{ll}
  \hyperlink{def-choice}{\textbf{if}} & #1\\
  \hyperlink{def-choice}{\textbf{then}} & #2\\
  \hyperlink{def-choice}{\textbf{else}} & #3\\
  \end{array}\right.
}
\newcommand\equallength[0]{\hyperlink{def-equallength}{\textfunc{equal\_length}}}
\newcommand\listrange[0]{\hyperlink{def-listrange}{\textfunc{indices}}}
\newcommand\Proselistrange[2]{\hyperlink{def-listrange}{index} #1 in the list of indices for #2}
\newcommand\listlen[1]{\hyperlink{def-listlen}{|}#1\hyperlink{def-listlen}{|}}
\newcommand\cardinality[1]{\hyperlink{def-cardinality}{|}#1\hyperlink{def-cardinality}{|}}
\newcommand\unziplist[0]{\hyperlink{def-unziplist}{\textfunc{unzip}}}
\newcommand\unziplistthree[0]{\hyperlink{def-unziplistthree}{\textfunc{unzip3}}}
\newcommand\inlist[0]{\hyperlink{def-inlist}{\in}}
\newcommand\uniquep[0]{\hyperlink{def-uniquep}{\textfunc{unique'}}}
\newcommand\uniquelist[0]{\hyperlink{def-uniquelist}{\textfunc{unique}}}
\newcommand\listset[0]{\hyperlink{def-listset}{\textfunc{list\_set}}}
\newcommand\concat[0]{\hyperlink{def-concat}{+}}
\newcommand\concatlist[0]{\hyperlink{def-concatlist}{\textfunc{concat}}}
\newcommand\listprefix[0]{\hyperlink{def-listprefix}{\textfunc{prefix}}}
\newcommand\stringconcat[0]{\hyperlink{def-stringconcat}{\texttt{+}}}
\newcommand\stringofnat[0]{\hyperlink{def-stringofnat}{\texttt{string\_of\_nat}}}

\newcommand\Ignore[0]{\hyperlink{def-ignore}{\underline{\;\;}}}
\newcommand\Option[1]{\hyperlink{def-Option}{\textsf{Option}}\left({#1}\right)} % The optional type term
\newcommand\optionalterm[0]{\hyperlink{def-Option}{optional}}
\newcommand\None[0]{\hyperlink{def-none}{\texttt{None}}}
\newcommand\some[1]{\hyperlink{def-some}{\textsf{Some}}\left({#1}\right)} % The optional expression with one element
\newcommand\KleeneStar[1]{{#1}^{\hyperlink{def-kleenestar}{*}}}
\newcommand\KleenePlus[1]{{#1}^{\hyperlink{def-kleeneplus}{+}}}

% General types
\newcommand\N[0]{\hyperlink{type-N}{\mathbb{N}}}
\newcommand\Npos[0]{\hyperlink{type-Npos}{\mathbb{N}^{+}}}
\newcommand\Q[0]{\hyperlink{type-Q}{\mathbb{Q}}}
\newcommand\Z[0]{\hyperlink{type-Z}{\mathbb{Z}}}
\newcommand\Bool[0]{\hyperlink{type-Bool}{\mathbb{B}}}
\newcommand\Identifier[0]{\hyperlink{type-Identifier}{\mathbb{I}}}
\newcommand\Strings[0]{\hyperlink{type-Strings}{\mathbb{S}}}
\newcommand\ASTLabels[0]{\hyperlink{type-ASTLabels}{\mathbb{L}}}

\newcommand\pow[1]{\hyperlink{def-pow}{\mathcal{P}}\left(#1\right)}
\newcommand\powfin[1]{\hyperlink{def-powfin}{\mathcal{P}_{\text{fin}}}\left(#1\right)}
\newcommand\partialto[0]{\hyperlink{def-partialfunc}{\rightharpoonup}}
\newcommand\rightarrowfin[0]{\hyperlink{def-finfunction}{\rightarrow_{\text{fin}}}}
\newcommand\funcgraph[0]{\hyperlink{def-funcgraph}{\texttt{func\_graph}}}
\DeclareMathOperator{\dom}{\hyperlink{def-dom}{dom}}
\newcommand\sign[0]{\hyperlink{def-sign}{\texttt{sign}}}
\newcommand\positivesign[0]{\hyperlink{constant-positivesign}{\texttt{1}}}
\newcommand\negativesign[0]{\hyperlink{constant-negativesign}{\texttt{-1}}}
\newcommand\equalsign[0]{\hyperlink{constant-equalsign}{\texttt{0}}}
\newcommand\graphtransitive[1]{#1^{\hyperlink{def-graphtransitive}{+}}}
\newcommand\graphtransitivereflexive[1]{#1^{\hyperlink{def-graphtransitivereflexive}{*}}}

\newcommand\configdomain[1]{\hyperlink{def-configdomain}{\texttt{config\_dom}}({#1})}

\newcommand\sslash[0]{\mathbin{/\mkern-6mu/}}
\newcommand\terminateas[0]{\hyperlink{def-terminateas}{\sslash}\;}
\newcommand\booltrans[0]{\hyperlink{relation-booltransition}{\textfunc{bool\_transition}}}
\newcommand\booltransarrow[0]{\longrightarrow}
\newcommand\Prosetecheck[2]{\hyperlink{relation-techeck}{checking} #1 yields $\True$\ProseTerminateAs{#2}}
\newcommand\Prosebecheck[2]{\hyperlink{relation-becheck}{checking} #1 yields $\True$\ProseTerminateAs{#2}}
\newcommand\becheck[0]{\hyperlink{relation-becheck}{\textfunc{be\_check}}}

\newcommand\termx[0]{\mathit{tx}}
\newcommand\termy[0]{\mathit{ty}}
\newcommand\rulearrow[0]{\xrightarrow{R}}

\newcommand\XP[0]{\mathit{XP}}
\newcommand\XQ[0]{\mathit{XQ}}

\newcommand\mapopt[1]{\hyperlink{def-mapopt}{\textsf{optional}}[#1]}
\newcommand\Prosemapopt[3]{\hyperlink{def-mapopt}{applying} #1 to the optional value #2 via \hyperlink{def-mapopt}{\textsf{optional}}
  yields #3}

\newcommand\derivationtreeterm[0]{\hyperlink{def-derivationtree}{derivation tree}}
\newcommand\derivationtreesterm[0]{\hyperlink{def-derivationtree}{derivation trees}}

%%%%%%%%%%%%%%%%%%%%%%%%%%%%%%%%%%%%%%%%%%%%%%%%%%
% Syntax macros
\newcommand\nonterminal[1]{\texttt{#1}}
\newcommand\terminal[1]{\mathtt{\mathbf{#1}}}
\newcommand\verbatimterminal[2]{\texttt{"}\texttt{#2}\texttt{"}} % Grammar terminals
\newcommand\emptysentence[0]{\hyperlink{def-emptysentence}{\epsilon}}
\newcommand\astof[1]{\overline{{#1}}}
\newcommand\parsenode[1]{\hyperlink{def-parsenode}{\mathbb{P}\mathbb{A}\mathbb{R}\mathbb{S}\mathbb{E}}[#1]}
\newcommand\namednode[2]{#1:#2} % #1 is a free variable, #2 is the grammar symbol
\newcommand\punnode[1]{#1}
\newcommand\epsilonnode[0]{\hyperlink{def-epsilonnode}{\textsf{epsilon\_node}}}
\newcommand\yield[0]{\hyperlink{def-yield}{\textsf{yield}}}

\newcommand\substrecordfield[0]{\hyperlink{def-substrecordfield}{\textfunc{subst\_record\_field}}}
\newcommand\uniquesymb[1]{\textsf{unique}(#1)}
\newcommand\Ndecllist[0]{\nonterminal{decl\_list}}

%%%%%%%%%%%%%%%%%%%%%%%%%%%%%%%%%%%%%%%%%%%%%%%%%%%%%%%%%%%%%%%%%%%%%%%%%%%%%%
% Macros for terminal tokens
\newcommand\Tlooplimit[0]{\verbatimterminal{LOOPLIMIT}{looplimit}}
\newcommand\Trecurselimit[0]{\verbatimterminal{RECURSELIMIT}{recurselimit}}
\newcommand\Taccessor[0]{\verbatimterminal{ACCESSOR}{accessor}}
\newcommand\Tand[0]{\verbatimterminal{AND}{AND}}
\newcommand\Tarray[0]{\verbatimterminal{ARRAY}{array}}
\newcommand\Tarrow[0]{\verbatimterminal{ARROW}{=>}}
\newcommand\Tas[0]{\verbatimterminal{AS}{as}}
\newcommand\Tassert[0]{\verbatimterminal{ASSERT}{assert}}
\newcommand\Tband[0]{\verbatimterminal{BAND}{\&\&}}
\newcommand\Tbimpl[0]{\verbatimterminal{BAND}{==>}}
\newcommand\Tbegin[0]{\verbatimterminal{BEGIN}{begin}}
\newcommand\Tbeq[0]{\verbatimterminal{BEQ}{<=>}}
\newcommand\Tbit[0]{\verbatimterminal{BIT}{bit}}
\newcommand\Tbits[0]{\verbatimterminal{BITS}{bits}}
\newcommand\Tbnot[0]{\verbatimterminal{BNOT}{!}}
\newcommand\Tboolean[0]{\verbatimterminal{BOOLEAN}{boolean}}
\newcommand\Tbor[0]{\verbatimterminal{BOR}{||}}
\newcommand\Tcase[0]{\verbatimterminal{CASE}{case}}
\newcommand\Tcatch[0]{\verbatimterminal{CATCH}{catch}}
\newcommand\Tcollection[0]{\verbatimterminal{COLLECTION}{collection}}
\newcommand\Tcolon[0]{\verbatimterminal{COLON}{:}}
\newcommand\Tcoloncolon[0]{\verbatimterminal{COLON\_COLON}{::}}
\newcommand\Tcomma[0]{\verbatimterminal{COMMA}{,}}
\newcommand\Tconfig[0]{\verbatimterminal{CONFIG}{config}}
\newcommand\Tconstant[0]{\verbatimterminal{CONSTANT}{constant}}
\newcommand\Tdiv[0]{\verbatimterminal{DIV}{DIV}}
\newcommand\Tdivrm[0]{\verbatimterminal{DIVRM}{DIVRM}}
\newcommand\Tdo[0]{\verbatimterminal{DO}{do}}
\newcommand\Tdot[0]{\verbatimterminal{DOT}{.}}
\newcommand\Tdownto[0]{\verbatimterminal{DOWNTO}{downto}}
\newcommand\Telse[0]{\verbatimterminal{ELSE}{else}}
\newcommand\Telseif[0]{\verbatimterminal{ELSIF}{elsif}}
\newcommand\Tend[0]{\verbatimterminal{END}{end}}
\newcommand\Tenumeration[0]{\verbatimterminal{ENUMERATION}{enumeration}}
\newcommand\Txor[0]{\verbatimterminal{XOR}{XOR}}
\newcommand\Teq[0]{\verbatimterminal{EQ}{=}}
\newcommand\Teqop[0]{\verbatimterminal{EQ\_EQ}{==}}
\newcommand\Texception[0]{\verbatimterminal{EXCEPTION}{exception}}
\newcommand\Tfor[0]{\verbatimterminal{FOR}{for}}
\newcommand\Tfunc[0]{\verbatimterminal{FUNC}{func}}
\newcommand\Tgeq[0]{\verbatimterminal{GE}{>=}}
\newcommand\Tgetter[0]{\verbatimterminal{GETTER}{getter}}
\newcommand\Tgt[0]{\verbatimterminal{GT}{>}}
\newcommand\Tif[0]{\verbatimterminal{IF}{if}}
\newcommand\Timpdef[0]{\verbatimterminal{IMPDEF}{impdef}}
\newcommand\Timplementation[0]{\verbatimterminal{IMPLEMENTATION}{implementation}}
\newcommand\Timpl[0]{\verbatimterminal{IMPL}{==>}}
\newcommand\Tin[0]{\verbatimterminal{IN}{IN}}
\newcommand\Tinteger[0]{\verbatimterminal{INTEGER}{integer}}
\newcommand\Tlbrace[0]{\verbatimterminal{LBRACE}{\{}}
\newcommand\Tlbracket[0]{\verbatimterminal{LBRACKET}{[}}
\newcommand\Tllbracket[0]{\verbatimterminal{LLBRACKET}{[[}}
\newcommand\Tleq[0]{\verbatimterminal{LE}{<=}}
\newcommand\Tlet[0]{\verbatimterminal{LET}{let}}
\newcommand\Tlpar[0]{\verbatimterminal{LPAR}{(}}
\newcommand\Tlt[0]{\verbatimterminal{LT}{<}}
\newcommand\Tminus[0]{\verbatimterminal{MINUS}{-}}
\newcommand\Tmod[0]{\verbatimterminal{MOD}{MOD}}
\newcommand\Tmul[0]{\verbatimterminal{MUL}{*}}
\newcommand\Tneq[0]{\verbatimterminal{NE}{!=}}
\newcommand\Tnoreturn[0]{\verbatimterminal{NORETURN}{noreturn}}
\newcommand\Tnot[0]{\verbatimterminal{NOT}{NOT}}
\newcommand\Tof[0]{\verbatimterminal{OF}{of}}
\newcommand\Tor[0]{\verbatimterminal{OR}{OR}}
\newcommand\Totherwise[0]{\verbatimterminal{OTHERWISE}{otherwise}}
\newcommand\Tpass[0]{\verbatimterminal{PASS}{pass}}
\newcommand\Tplus[0]{\verbatimterminal{PLUS}{+}}
\newcommand\Tpluscolon[0]{\verbatimterminal{PLUS\_COLON}{+:}}
\newcommand\Tplusplus[0]{\verbatimterminal{PLUS\_PLUS}{++}}
\newcommand\Tpow[0]{\verbatimterminal{POW}{\^{}}}
\newcommand\Tpragma[0]{\verbatimterminal{PRAGMA}{pragma}}
\newcommand\Tprint[0]{\verbatimterminal{PRINT}{print}}
\newcommand\Tprintln[0]{\verbatimterminal{PRINTLN}{println}}
\newcommand\Tpure[0]{\verbatimterminal{PURE}{pure}}
\newcommand\Tunreachable[0]{\verbatimterminal{UNREACHABLE}{unreachable}}
\newcommand\Trbrace[0]{\verbatimterminal{RBRACE}{\}}}
\newcommand\Trbracket[0]{\verbatimterminal{RBRACKET}{]}}
\newcommand\Trrbracket[0]{\verbatimterminal{RRBRACKET}{]]}}
\newcommand\Trdiv[0]{\verbatimterminal{RDIV}{/}}
\newcommand\Treadonly[0]{\verbatimterminal{READONLY}{readonly}}
\newcommand\Treal[0]{\verbatimterminal{REAL}{real}}
\newcommand\Trecord[0]{\verbatimterminal{RECORD}{record}}
\newcommand\Trepeat[0]{\verbatimterminal{REPEAT}{repeat}}
\newcommand\Treturn[0]{\verbatimterminal{RETURN}{return}}
\newcommand\Trpar[0]{\verbatimterminal{RPAR}{)}}
\newcommand\Tstarcolon[0]{\verbatimterminal{STAR\_COLON}{*:}}
\newcommand\Tsemicolon[0]{\verbatimterminal{SEMI\_COLON}{;}}
\newcommand\Tsetter[0]{\verbatimterminal{SETTER}{setter}}
\newcommand\Tshl[0]{\verbatimterminal{SHL}{{<}{<}}}
\newcommand\Tshr[0]{\verbatimterminal{SHR}{{>}{>}}}
\newcommand\Tslicing[0]{\verbatimterminal{SLICING}{..}}
\newcommand\Tstring[0]{\verbatimterminal{STRING}{string}}
\newcommand\Tsubtypes[0]{\verbatimterminal{SUBTYPES}{subtypes}}
\newcommand\Tthen[0]{\verbatimterminal{THEN}{then}}
\newcommand\Tthrow[0]{\verbatimterminal{THROW}{throw}}
\newcommand\Tto[0]{\verbatimterminal{TO}{to}}
\newcommand\Ttry[0]{\verbatimterminal{TRY}{try}}
\newcommand\Ttype[0]{\verbatimterminal{TYPE}{type}}
\newcommand\Tarbitrary[0]{\verbatimterminal{ARBITRARY}{ARBITRARY}}
\newcommand\Tuntil[0]{\verbatimterminal{UNTIL}{until}}
\newcommand\Tvar[0]{\verbatimterminal{VAR}{var}}
\newcommand\Twhen[0]{\verbatimterminal{WHEN}{when}}
\newcommand\Twhere[0]{\verbatimterminal{WHERE}{where}}
\newcommand\Twhile[0]{\verbatimterminal{WHILE}{while}}
\newcommand\Twith[0]{\verbatimterminal{WITH}{with}}

\newcommand\Tidentifier[0]{\hyperlink{def-tidentifier}{\terminal{ID}}}
\newcommand\Tstringlit[0]{\hyperlink{def-tstringlit}{\terminal{STRING\_LIT}}}
\newcommand\Tmasklit[0]{\hyperlink{def-tmasklit}{\terminal{MASK\_LIT}}}
\newcommand\Tbitvectorlit[0]{\hyperlink{def-tbitvectorlit}{\terminal{BITVECTOR\_LIT}}}
\newcommand\Tintlit[0]{\hyperlink{def-tintlit}{\terminal{INT\_LIT}}}
\newcommand\Treallit[0]{\hyperlink{def-treallit}{\terminal{REAL\_LIT}}}
\newcommand\Tboollit[0]{\hyperlink{def-tboollit}{\terminal{BOOL\_LIT}}}
\newcommand\Tlexeme[0]{\hyperlink{def-tlexeme}{\terminal{LEXEME}}}

\newcommand\Tunops[0]{\hyperlink{def-tunops}{\terminal{UNOPS}}}
\newcommand\precedence[1]{\textsf{precedence: }#1}

%%%%%%%%%%%%%%%%%%%%%%%%%%%%%%%%%%%%%%%%%%%%%%%%%%%%%%%%%%%%%%%%%%%%%%%%%%%%%%
% Macros for non-terminals
\newcommand\Nspec[0]{\hyperlink{def-nspec}{\nonterminal{spec}}}
\newcommand\Ndecl[0]{\hyperlink{def-ndecl}{\nonterminal{decl}}}
\newcommand\Nparamsopt[0]{\hyperlink{def-nparamsopt}{\nonterminal{params\_opt}}}
\newcommand\Ncall[0]{\hyperlink{def-ncall}{\nonterminal{call}}}
\newcommand\Nelidedparamcall[0]{\hyperlink{def-nelidedparamcall}{\nonterminal{elided\_param\_call}}}
\newcommand\Nfuncargs[0]{\hyperlink{def-nfuncargs}{\nonterminal{func\_args}}}
\newcommand\Naccessorbody[0]{\hyperlink{def-naccessorbody}{\nonterminal{accessor\_body}}}
\newcommand\Naccessors[0]{\hyperlink{def-naccessors}{\nonterminal{accessors}}}
\newcommand\Nqualifier[0]{\hyperlink{def-nqualifier}{\nonterminal{qualifier}}}
\newcommand\Npuritykeyword[0]{\hyperlink{def-npuritykeyword}{\nonterminal{purity\_keyword}}}
\newcommand\Nisreadonly[0]{\hyperlink{def-nisreadonly}{\nonterminal{is\_readonly}}}
\newcommand\Noverride[0]{\hyperlink{def-noverride}{\nonterminal{override}}}
\newcommand\Nreturntype[0]{\hyperlink{def-nreturntype}{\nonterminal{return\_type}}}
\newcommand\Nfuncbody[0]{\hyperlink{def-nfuncbody}{\nonterminal{func\_body}}}
\newcommand\Ntypedidentifier[0]{\hyperlink{def-ntypedidentifier}{\nonterminal{typed\_identifier}}}
\newcommand\Nasty[0]{\hyperlink{def-nasty}{\nonterminal{as\_ty}}}
\newcommand\Ntydecl[0]{\hyperlink{def-ntydecl}{\nonterminal{ty\_decl}}}
\newcommand\Nsubtype[0]{\hyperlink{def-nsubtype}{\nonterminal{subtype}}}
\newcommand\Nsubtypeopt[0]{\hyperlink{def-nsubtypeopt}{\nonterminal{subtype\_opt}}}
\newcommand\Nglobaldeclkeywordnonconfig[0]{\hyperlink{def-nglobaldeclkeywordnonconfig}{\nonterminal{global\_keyword\_non\_config}}}
\newcommand\Nglobaldeclkeyword[0]{\hyperlink{def-nglobaldeclkeyword}{\nonterminal{global\_keyword}}}
\newcommand\Nignoredoridentifier[0]{\hyperlink{def-nignoredoridentifier}{\nonterminal{ignored\_or\_identifier}}}
\newcommand\Nty[0]{\hyperlink{def-nty}{\nonterminal{ty}}}
\newcommand\Ntyorcollection[0]{\hyperlink{def-ntyorcollection}{\nonterminal{ty\_or\_collection}}}
\newcommand\Nexpr[0]{\hyperlink{def-nexpr}{\nonterminal{expr}}}
\newcommand\Naccess[0]{\hyperlink{def-naccess}{\nonterminal{access}}}
\newcommand\Nbasiclexpr[0]{\hyperlink{def-nbasiclexpr}{\nonterminal{basic\_lexpr}}}
\newcommand\Ndiscardorbasiclexpr[0]{\hyperlink{def-ndiscardorbasiclexpr}{\nonterminal{discard\_or\_basic\_lexpr}}}
\newcommand\Ndiscardoridentifier[0]{\hyperlink{def-ndiscardoridentifier}{\nonterminal{discard\_or\_identifier}}}
\newcommand\Nsetteraccess[0]{\hyperlink{def-nsetteraccess}{\nonterminal{setter\_access}}}
\newcommand\Nlexpr[0]{\hyperlink{def-nlexpr}{\nonterminal{lexpr}}}
\newcommand\Nfields[0]{\hyperlink{def-nfields}{\nonterminal{fields}}}
\newcommand\Nopttypedidentifier[0]{\hyperlink{def-nopttypeidentifier}{\nonterminal{opt\_typed\_identifier}}}
\newcommand\Nstmtlist[0]{\hyperlink{def-nstmtlist}{\nonterminal{stmt\_list}}}
\newcommand\Nlocaldeclkeyword[0]{\hyperlink{def-nlocaldeclkeyword}{\nonterminal{local\_decl\_keyword}}}
\newcommand\Ndirection[0]{\hyperlink{def-ndirection}{\nonterminal{direction}}}
\newcommand\Ncasealt[0]{\hyperlink{def-ncasealt}{\nonterminal{case\_alt}}}
\newcommand\Ncasealtlist[0]{\hyperlink{def-ncasealtlist}{\nonterminal{case\_alt\_list}}}
\newcommand\Npatternlist[0]{\hyperlink{def-npatternlist}{\nonterminal{pattern\_list}}}
\newcommand\Notherwiseopt[0]{\hyperlink{def-notherwiseopt}{\nonterminal{otherwise\_opt}}}
\newcommand\Ncatcher[0]{\hyperlink{def-ncatcher}{\nonterminal{catcher}}}
\newcommand\Nstmt[0]{\hyperlink{def-nstmt}{\nonterminal{stmt}}}
\newcommand\Nselse[0]{\hyperlink{def-nselse}{\nonterminal{s\_else}}}
\newcommand\Ndeclitem[0]{\hyperlink{def-ndeclitem}{\nonterminal{decl\_item}}}
\newcommand\Nslice[0]{\hyperlink{def-nslice}{\nonterminal{slice}}}
\newcommand\Nslices[0]{\hyperlink{def-nslices}{\nonterminal{slices}}}
\newcommand\Nconstraintkind[0]{\hyperlink{def-nintconstraints}{\nonterminal{constraint\_kind}}}
\newcommand\Nconstraintkindopt[0]{\hyperlink{def-nintconstraintsopt}{\nonterminal{constraint\_kind\_opt}}}
\newcommand\Nintconstraint[0]{\hyperlink{def-nintconstraint}{\nonterminal{int\_constraint}}}
\newcommand\Nexprpattern[0]{\hyperlink{def-nexprpattern}{\nonterminal{expr\_pattern}}}
\newcommand\Npatternset[0]{\hyperlink{def-npatternset}{\nonterminal{pattern\_set}}}
\newcommand\Npattern[0]{\hyperlink{def-npattern}{\nonterminal{pattern}}}
\newcommand\Nbitfields[0]{\hyperlink{def-nbitfields}{\nonterminal{bitfields}}}
\newcommand\Nbitfield[0]{\hyperlink{def-nbitfield}{\nonterminal{bitfield}}}
\newcommand\Nvalue[0]{\hyperlink{def-nvalue}{\nonterminal{value}}}
\newcommand\Nbinop[0]{\hyperlink{def-nbinop}{\nonterminal{binop}}}
\newcommand\Nunop[0]{\hyperlink{def-nunop}{\nonterminal{unop}}}
\newcommand\Nfieldassign[0]{\hyperlink{def-nfieldassign}{\nonterminal{field\_assign}}}
\newcommand\Nlooplimit[0]{\hyperlink{def-nlooplimit}{\nonterminal{loop\_limit}}}
\newcommand\Nrecurselimit[0]{\hyperlink{def-nrecurselimit}{\nonterminal{recurse\_limit}}}

%%%%%%%%%%%%%%%%%%%%%%%%%%%%%%%%%%%%%%%%%%%%%%%%%%%%%%%%%%%%%%%%%%%%%%%%%%%%%%
% Macros for defining associativity
\newcommand\nonassoc[0]{\textsf{nonassoc}}
\newcommand\leftassoc[0]{\textsf{left}}
\newcommand\rightassoc[0]{\textsf{right}}

%%%%%%%%%%%%%%%%%%%%%%%%%%%%%%%%%%%%%%%%%%%%%%%%%%%%%%%%%%%%%%%%%%%%%%%%%%%%%%
% Macros for generic parsing symbols and parsing functions
\newcommand\derives[0]{\longrightarrow}
\newcommand\derivesinline[0]{\xlongrightarrow{\textsf{inline}}}
\newcommand\parsesep[0]{\ } % separates symbols in a single production

\newcommand\maybeemptylist[1]{\hyperlink{def-maybeemptylist}{\textsf{list}^{*}}(#1)} % This stands for list(x)
\newcommand\ListOne[1]{\hyperlink{def-listone}{\textsf{list1}}(#1)} % This stands for non_empty_list(x)
\newcommand\ClistOne[1]{\hyperlink{def-clistone}{\textsf{clist1}}(#1)}
\newcommand\PlistZero[1]{\hyperlink{def-plistzero}{\textsf{plist0}}(#1)}
\newcommand\Plisttwo[1]{\hyperlink{def-plisttwo}{\textsf{plist2}}(#1)}
\newcommand\ClistZero[1]{\hyperlink{def-clistzero}{\textsf{clist0}}(#1)}
\newcommand\Clisttwo[1]{\hyperlink{def-clisttwo}{\textsf{clist2}}(#1)}
\newcommand\TClistOne[1]{\hyperlink{def-tclistone}{\textsf{tclist1}}(#1)}
\newcommand\TClistZero[1]{\hyperlink{def-tclistzero}{\textsf{tclist0}}(#1)}
\newcommand\option[1]{\hyperlink{def-option}{\textsf{option}}(#1)}

%%%%%%%%%%%%%%%%%%%%%%%%%%%%%%%%%%%%%%%%%%%%%%%%%%%%%%%%%%%%%%%%%%%%%%%%%%%%%%%
%% AST macros with hyperlinks

\newcommand\zerobit[0]{\hyperlink{constant-zerobit}{\texttt{0}}}
\newcommand\onebit[0]{\hyperlink{constant-onebit}{\texttt{1}}}
\newcommand\xbit[0]{\hyperlink{constant-xbit}{\texttt{x}}}

\newcommand\ARBITRARY[0]{\hyperlink{ast-EArbitrary}{\textastlabel{ARBITRARY}}}

% Non-terminal names

%%%%%%%%%%%%%%%%%%%%%%%%%%%%%%%%%%%%%%%%%%%%%%%%%%%%%%%%%%%%%%%%%%%%%%%%%%%%%%%%
% Special macros for the typed AST:
% We currently do not distinguish between untyped expressions and typed
% expressions and so we typeset them both with the same symbol.
\newcommand\typedSThrow[0]{\hyperlink{ast-typedSThrow}{\SThrow}}
\newcommand\typedSliceLength[0]{\hyperlink{ast-typedSliceLength}{\SliceLength}}
\newcommand\typedWellConstrained[0]{\hyperlink{ast-typedWellConstrained}{\WellConstrained}}
%%%%%%%%%%%%%%%%%%%%%%%%%%%%%%%%%%%%%%%%%%%%%%%%%%%%%%%%%%%%%%%%%%%%%%%%%%%%%%%%

\newcommand\precisionlossindicatorterm[0]{\hyperlink{ast-precisionlossindicator}{precision loss indicator}}
\newcommand\lhsaccess[0]{\hyperlink{ast-lhsaccess}{\textsf{lhs\_access}}}
\newcommand\fieldorarrayaccess[0]{\hyperlink{ast-fieldorarrayaccess}{\textsf{field\_or\_array\_access}}}
\newcommand\FieldAccess[0]{\hyperlink{ast-fieldaccess}{\textsf{FieldAccess}}}
\newcommand\ArrayAccess[0]{\hyperlink{ast-arrayaccess}{\textsf{ArrayAccess}}}
\newcommand\accessorpair[0]{\hyperlink{ast-accessorpair}{\textsf{accessor\_pair}}}

% Left-hand-side expression labels
\newcommand\ProseLESlice{\hyperlink{ast-LESlice}{left-hand-side slicing expression}}

\newcommand\Proseimpdefsubprogram[0]{implementation-defined subprogram}
\newcommand\Proseimpdefsubprograms[0]{implementation-defined subprograms}
\newcommand\Proseimplementationsubprogram[0]{implementation subprogram}
\newcommand\Proseimplementationsubprograms[0]{implementation subprograms}

%%%%%%%%%%%%%%%%%%%%%%%%%%%%%%%%%%%%%%%%%%%%%%%%%%%%%%%%%%%%%%%%%%%%%%%%%%%%%%%
% Macros for AST builders
\newcommand\BuildError[1]{\hyperlink{type-builderror}{\textsf{BuildError}}(\texttt{#1})}
\newcommand\BuildErrorConfig[0]{\hyperlink{def-builderrorconfig}{\texttt{\#BE}}}
\newcommand\TBuildError[0]{\hyperlink{def-tbuilderror}{\textsf{TBuildError}}}
\newcommand\ProseOtherwiseBuildError[0]{Otherwise, the result is a build error.}
\newcommand\OrBuildError[0]{\;\terminateas \BuildErrorConfig}

\newcommand\astarrow[0]{\xrightarrow{\textsf{ast}}}
\newcommand\scanarrow[0]{\xrightarrow{\hyperlink{def-aslscan}{\textsf{scan}}}}

\newcommand\buildidentity[0]{\hyperlink{build-identity}{\textfunc{build\_identity}}}
\newcommand\buildlist[0]{\hyperlink{build-list}{\textfunc{build\_list}}}
\newcommand\buildplist[0]{\hyperlink{build-plist}{\textfunc{build\_plist}}}
\newcommand\buildclist[0]{\hyperlink{build-clist}{\textfunc{build\_clist}}}
\newcommand\buildtclist[0]{\hyperlink{build-tclist}{\textfunc{build\_tclist}}}
\newcommand\buildoption[0]{\hyperlink{build-option}{\textfunc{build\_option}}}

\newcommand\Prosebuildast[2]{\hyperlink{build-ast}{building} an untyped AST from the parse tree #1 yields #2}
\newcommand\buildast[0]{\hyperlink{build-ast}{\textfunc{build\_ast}}}
\newcommand\builddecl[0]{\hyperlink{build-decl}{\textfunc{build\_decl}}}
\newcommand\builddeclitem[0]{\hyperlink{build-declitem}{\textfunc{build\_decl\_item}}}
\newcommand\makesetter[0]{\hyperlink{def-makesetter}{\textfunc{make\_setter}}}
\newcommand\desugarsetter[0]{\hyperlink{def-desugarsetter}{\textfunc{desugar\_setter}}}
\newcommand\desugarsettersetfields[0]{\hyperlink{def-desugarsettersetfields}{\textfunc{desugar\_setter\_setfields}}}
\newcommand\readmodifywrite[0]{\hyperlink{def-readmodifywrite}{\textfunc{read\_modify\_write}}}
\newcommand\desugaraccessorpair[0]{\hyperlink{def-desugaraccessorpair}{\textfunc{desugar\_accessor\_pair}}}
\newcommand\buildlooplimit[0]{\hyperlink{build-looplimit}{\textfunc{build\_looplimit}}}
\newcommand\buildrecurselimit[0]{\hyperlink{build-recurselimit}{\textfunc{build\_recurselimit}}}
\newcommand\desugarelidedparameter[0]{\hyperlink{def-desugarelidedparameter}{\textfunc{desugar\_elided\_parameter}}}
\newcommand\buildstmt[0]{\hyperlink{build-stmt}{\textfunc{build\_stmt}}}
\newcommand\buildexpr[0]{\hyperlink{build-expr}{\textfunc{build\_expr}}}
\newcommand\buildlexpr[0]{\hyperlink{build-lexpr}{\textfunc{build\_lexpr}}}
\newcommand\buildbasiclexpr[0]{\hyperlink{build-basiclexpr}{\textfunc{build\_basic\_lexpr}}}
\newcommand\builddiscardorbasiclexpr[0]{\hyperlink{build-discardorbasiclexpr}{\textfunc{build\_discard\_or\_basic\_lexpr}}}
\newcommand\buildaccess[0]{\hyperlink{build-access}{\textfunc{build\_access}}}
\newcommand\builddiscardoridentifier[0]{\hyperlink{build-discardoridentifier}{\textfunc{build\_discard\_or\_identifier}}}
\newcommand\desugarlhsaccess[0]{\hyperlink{def-desugarlhsaccess}{\textfunc{desugar\_lhs\_access}}}
\newcommand\desugarlhsaccessopt[0]{\hyperlink{def-desugarlhsaccessopt}{\textfunc{desugar\_lhs\_access\_opt}}}
\newcommand\desugarlhstuple[0]{\hyperlink{def-desugarlhstuple}{\textfunc{desugar\_lhs\_tuple}}}
\newcommand\desugarlhsfieldopt[0]{\hyperlink{def-desugarlhsfieldopt}{\textfunc{desugar\_lhs\_field\_opt}}}
\newcommand\desugarlhsfieldstuple[0]{\hyperlink{def-desugarlhsfieldstuple}{\textfunc{desugar\_lhs\_fields\_tuple}}}
\newcommand\buildparamsopt[0]{\hyperlink{build-paramsopt}{\textfunc{build\_params\_opt}}}
\newcommand\buildfuncargs[0]{\hyperlink{build-funcargs}{\textfunc{build\_func\_args}}}
\newcommand\buildcall[0]{\hyperlink{build-call}{\textfunc{build\_call}}}
\newcommand\buildelidedparamcall[0]{\hyperlink{build-elided-param-call}{\textfunc{build\_elided\_param\_call}}}
\newcommand\setcalltype[0]{\hyperlink{def-setcalltype}{\textfunc{set\_call\_type}}}
\newcommand\buildreturntype[0]{\hyperlink{build-returntype}{\textfunc{build\_return\_type}}}
\newcommand\buildfuncbody[0]{\hyperlink{build-funcbody}{\textfunc{build\_func\_body}}}
\newcommand\buildaccessorbody[0]{\hyperlink{build-accessorbody}{\textfunc{build\_accessor\_body}}}
\newcommand\buildaccessors[0]{\hyperlink{build-accessors}{\textfunc{build\_accessors}}}
\newcommand\buildisreadonly[0]{\hyperlink{build-isreadonly}{\textfunc{build\_is\_readonly}}}
\newcommand\buildqualifier[0]{\hyperlink{build-qualifier}{\textfunc{build\_func\_qualifier}}}
\newcommand\buildoverride[0]{\hyperlink{build-override}{\textfunc{build\_override}}}
\newcommand\buildtypedidentifier[0]{\hyperlink{build-typedidentifier}{\textfunc{build\_typed\_identifier}}}
\newcommand\buildopttypedidentifier[0]{\hyperlink{build-opttypedidentifier}{\textfunc{build\_opt\_typed\_identifier}}}
\newcommand\buildtydecl[0]{\hyperlink{build-tydecl}{\textfunc{build\_ty\_decl}}}
\newcommand\buildtyorcollection[0]{\hyperlink{build-tyorcollection}{\textfunc{build\_ty\_or\_collection}}}
\newcommand\buildsubtype[0]{\hyperlink{build-subtype}{\textfunc{build\_subtype}}}
\newcommand\buildsubtypeopt[0]{\hyperlink{build-subtypeopt}{\textfunc{build\_subtype\_opt}}}
\newcommand\buildglobaldeclkeywordnonconfig[0]{\hyperlink{build-globaldeclkeywordnonconfig}{\textfunc{build\_global\_keyword\_non\_config}}}
\newcommand\buildglobaldeclkeyword[0]{\hyperlink{build-globaldeclkeyword}{\textfunc{build\_global\_keyword}}}
\newcommand\builddirection[0]{\hyperlink{build-direction}{\textfunc{build\_direction}}}
\newcommand\buildcasealt[0]{\hyperlink{build-casealt}{\textfunc{build\_case\_alt}}}
\newcommand\buildcasealtlist[0]{\hyperlink{build-casealtlist}{\textfunc{build\_case\_alt\_list}}}
\newcommand\buildotherwiseopt[0]{\hyperlink{build-otherwiseopt}{\textfunc{build\_otherwise\_opt}}}
\newcommand\buildcatcher[0]{\hyperlink{build-catcher}{\textfunc{build\_catcher}}}
\newcommand\buildasty[0]{\hyperlink{build-as-ty}{\textfunc{build\_as\_ty}}}
\newcommand\buildignoredoridentifier[0]{\hyperlink{build-ignoredoridentifier}{\textfunc{build\_ignored\_or\_identifier}}}
\newcommand\buildlocaldeclkeyword[0]{\hyperlink{build-localdeclkeyword}{\textfunc{build\_local\_decl\_keyword}}}
\newcommand\buildstmtlist[0]{\hyperlink{build-stmtlist}{\textfunc{build\_stmt\_list}}}
\newcommand\buildselse[0]{\hyperlink{build-selse}{\textfunc{build\_s\_else}}}
\newcommand\buildconstraintkind[0]{\hyperlink{build-constraintkind}{\textfunc{build\_constraint\_kind}}}
\newcommand\buildconstraintkindopt[0]{\hyperlink{build-constraintkindopt}{\textfunc{build\_constraint\_kind\_opt}}}
\newcommand\buildintconstraint[0]{\hyperlink{build-intconstraint}{\textfunc{build\_int\_constraint}}}
\newcommand\buildexprpattern[0]{\hyperlink{build-exprpattern}{\textfunc{build\_expr\_pattern}}}
\newcommand\buildfieldassign[0]{\hyperlink{build-fieldassign}{\textfunc{build\_field\_assign}}}
\newcommand\buildpatternset[0]{\hyperlink{build-patternset}{\textfunc{build\_pattern\_set}}}
\newcommand\buildpatternlist[0]{\hyperlink{build-patternlist}{\textfunc{build\_pattern\_list}}}
\newcommand\buildpattern[0]{\hyperlink{build-pattern}{\textfunc{build\_pattern}}}
\newcommand\buildfields[0]{\hyperlink{build-fields}{\textfunc{build\_fields}}}
\newcommand\buildslices[0]{\hyperlink{build-slices}{\textfunc{build\_slices}}}
\newcommand\buildslice[0]{\hyperlink{build-slice}{\textfunc{build\_slice}}}
\newcommand\buildbitfields[0]{\hyperlink{build-bitfields}{\textfunc{build\_bitfields}}}
\newcommand\buildbitfield[0]{\hyperlink{build-bitfield}{\textfunc{build\_bitfield}}}
\newcommand\buildty[0]{\hyperlink{build-ty}{\textfunc{build\_ty}}}
\newcommand\buildvalue[0]{\hyperlink{build-value}{\textfunc{build\_value}}}
\newcommand\buildunop[0]{\hyperlink{build-unop}{\textfunc{build\_unop}}}
\newcommand\buildbinop[0]{\hyperlink{build-binop}{\textfunc{build\_binop}}}

\newcommand\binopprec[0]{\hyperlink{build-binopprec}{\textfunc{binop\_prec}}}
\newcommand\checknotsameprec[0]{\hyperlink{build-checknotsameprec}{\textfunc{check\_not\_same\_prec}}}

\newcommand\stmtfromlist[0]{\hyperlink{def-stmtfromlist}{\textfunc{stmt\_from\_list}}}
\newcommand\sequencestmts[0]{\hyperlink{def-sequencestmts}{\textfunc{sequence\_stmts}}}

\newcommand\eof[0]{\hyperlink{def-eof}{\textsf{eof}}}

\newcommand\anycharacter[1]{$\underbracket{#1}$}
\newcommand\Underscore[0]{\anycharacter{\texttt{ \_ }}}
\newcommand\regexminus[1]{\hyperlink{def-regexminus}{\texttt{-}}}

\newcommand\REasciichar[0]{\hyperlink{def-reasciichar}{\texttt{<}\textsf{ascii\_char}\texttt{>}}}
\newcommand\REchar[0]{\hyperlink{def-rechar}{\texttt{<}\textsf{char}\texttt{>}}}
\newcommand\REdigit[0]{\hyperlink{def-redigit}{\texttt{<}\textsf{digit}\texttt{>}}}
\newcommand\REintlit[0]{\hyperlink{def-reintlit}{\texttt{<}\textsf{int\_lit}\texttt{>}}}
\newcommand\REhexletter[0]{\hyperlink{def-rehexletter}{\texttt{<}\textsf{hex\_letter}\texttt{>}}}
\newcommand\REhexlit[0]{\hyperlink{def-rehexlit}{\texttt{<}\textsf{hex\_lit}\texttt{>}}}
\newcommand\REreallit[0]{\hyperlink{def-reallit}{\texttt{<}\textsf{real\_lit}\texttt{>}}}
\newcommand\REforbiddenhexremaining[0]{\hyperlink{def-forbiddenhexremaining}{\texttt{<}\textsf{forbidden\_hex\_remaining}\texttt{>}}}
\newcommand\REforbiddenhexfirst[0]{\hyperlink{def-forbiddenhexfirst}{\texttt{<}\textsf{forbidden\_hex\_first}\texttt{>}}}
\newcommand\REforbiddenrealremaining[0]{\hyperlink{def-forbiddenrealremaining}{\texttt{<}\textsf{forbidden\_real\_remaining}\texttt{>}}}
\newcommand\REforbiddenrealfirst[0]{\hyperlink{def-forbiddenrealfirst}{\texttt{<}\textsf{forbidden\_real\_first}\texttt{>}}}
\newcommand\REstrchar[0]{\hyperlink{def-restrchar}{\texttt{<}\textsf{str\_char}\texttt{>}}}
\newcommand\REstringlit[0]{\hyperlink{def-restringlit}{\texttt{<}\textsf{string\_lit}\texttt{>}}}
\newcommand\REbit[0]{\hyperlink{def-rebit}{\texttt{<}\textsf{bit}\texttt{>}}}
\newcommand\REbitvectorlit[0]{\hyperlink{def-rebitvectorlit}{\texttt{<}\textsf{bitvector\_lit}\texttt{>}}}
\newcommand\REbitmasklit[0]{\hyperlink{def-rebitmasklit}{\texttt{<}\textsf{bitmask\_lit}\texttt{>}}}
\newcommand\REletter[0]{\hyperlink{def-reletter}{\texttt{<}\textsf{letter}\texttt{>}}}
\newcommand\REidentifier[0]{\hyperlink{def-reidentifier}{\texttt{<}\textsf{identifier}\texttt{>}}}
\newcommand\RElinecomment[0]{\hyperlink{def-relinecomment}{\texttt{<}\textsf{line\_comment}\texttt{>}}}
\newcommand\REwhitespace[0]{\hyperlink{def-rewhitespace}{\texttt{<}\textsf{whitespace}\texttt{>}}}
\newcommand\shiftleftlexeme[0]{\texttt{"<}\texttt{<"}}
\newcommand\shiftrightlexeme[0]{\texttt{">}\texttt{>"}}

\newcommand\parsearrow[0]{\xrightarrow{\hyperlink{def-aslparse}{\textsf{parse}}}}
\newcommand\aslparse[0]{\hyperlink{def-aslparse}{\textfunc{asl\_parse}}}
\newcommand\Proseaslparse[2]{\hyperlink{def-aslparse}{parsing} the list of tokens #1 yields the parse tree #2}
\newcommand\aslscan[0]{\hyperlink{def-aslscan}{\textfunc{scan}}}
\newcommand\Proseaslscan[3]{\hyperlink{def-aslscan}{applying lexical analysis} to #1 with the lexical specification #2 yields the list of tokens #3}
\newcommand\maxmatches[0]{\hyperlink{def-maxmatch}{\textfunc{max\_matches}}}
\newcommand\remaxmatch[0]{\hyperlink{def-rematch}{\textfunc{re\_max\_match}}}
\newcommand\Token[0]{\hyperlink{def-token}{\mathbb{T}\mathbb{O}\mathbb{K}\mathbb{E}\mathbb{N}}}
\newcommand\discard[0]{\hyperlink{def-discard}{\textfunc{discard}}}
\newcommand\decimaltolit[0]{\hyperlink{def-decimaltolit}{\textfunc{dec\_to\_lit}}}
\newcommand\hextolit[0]{\hyperlink{def-hextolit}{\textfunc{hex\_to\_lit}}}
\newcommand\realtolit[0]{\hyperlink{def-realtolit}{\textfunc{real\_to\_lit}}}
\newcommand\strtolit[0]{\hyperlink{def-strtolit}{\textfunc{str\_to\_lit}}}
\newcommand\bitstolit[0]{\hyperlink{def-bitstolit}{\textfunc{bits\_to\_lit}}}
\newcommand\masktolit[0]{\hyperlink{def-masktolit}{\textfunc{mask\_to\_lit}}}
\newcommand\truetolit[0]{\hyperlink{def-truetolit}{\textfunc{true\_to\_lit}}}
\newcommand\falsetolit[0]{\hyperlink{def-falsetolit}{\textfunc{false\_to\_lit}}}
\newcommand\tokenid[0]{\hyperlink{def-tokenid}{\textfunc{token\_id}}}
\newcommand\lexicalerror[0]{\hyperlink{def-lexicalerror}{\textfunc{lexical\_error}}}
\newcommand\toidentifier[0]{\hyperlink{def-toidentifier}{\textfunc{to\_identifier}}}
\newcommand\eoftoken[0]{\hyperlink{def-eoftoken}{\textfunc{eof\_token}}}
\newcommand\Teof[0]{\hyperlink{def-teof}{\textsf{EOF}}}
\newcommand\Terror[0]{\hyperlink{def-terror}{\terminal{T\_ERR}}}
\newcommand\Twhitespace[0]{\hyperlink{def-twhitespace}{\textsf{WHITE\_SPACE}}}
\newcommand\LexSpec[0]{\hyperlink{def-lexspec}{\textsf{LexSpec}}}
\newcommand\Lang[0]{\hyperlink{def-lang}{\textsf{Lang}}}
\newcommand\RegExp[0]{\hyperlink{def-regex}{\textsf{RegExp}}}
\newcommand\ascii[1]{\hyperlink{def-ascii}{\textsf{ASCII}}\texttt{\{#1\}}}
\newcommand\vnewline[0]{\hyperlink{def-newline}{\texttt{newline}}}

\newcommand\spectoken[0]{\hyperlink{def-spectoken}{\textsc{spec\_token}}}
\newcommand\specstring[0]{\hyperlink{def-specstring}{\textsc{spec\_string}}}
\newcommand\speccomment[0]{\hyperlink{def-speccomment}{\textsc{spec\_comment}}}

\newcommand\actiondiscard[0]{\hyperlink{def-actiondiscard}{\textfunc{discard}}}
\newcommand\discardcommentchar[0]{\hyperlink{def-discardcommentchar}{\textfunc{discard\_comment\_char}}}
\newcommand\actiontoken[0]{\hyperlink{def-actiontoken}{\textfunc{return\_token}}}
\newcommand\actionstartstring[0]{\hyperlink{def-actionstartstring}{\textfunc{start\_string}}}
\newcommand\actionstartcomment[0]{\hyperlink{def-actionstartcomment}{\textfunc{start\_comment}}}

\newcommand\stringchar[0]{\hyperlink{def-stringchar}{\textfunc{string\_char}}}
\newcommand\stringfinish[0]{\hyperlink{def-stringfinish}{\textfunc{string\_finish}}}
\newcommand\stringescape[0]{\hyperlink{def-stringescape}{\textfunc{string\_escape}}}
\newcommand\Tstringchar[0]{\hyperlink{def-tstringchar}{\terminal{STRING\_CHAR}}}
\newcommand\Tstringend[0]{\hyperlink{def-tstringend}{\terminal{STRING\_END}}}
\newcommand\scanstring[0]{\hyperlink{def-scanstring}{\textfunc{scan\_string}}}

\newcommand\astversion[1]{#1\texttt{\_ast}}

% Fields
\newcommand\funcname[0]{\text{name}}
\newcommand\funcparameters[0]{\text{parameters}}
\newcommand\funcargs[0]{\text{args}}
\newcommand\funcbody[0]{\text{body}}
\newcommand\funcreturntype[0]{\text{return\_type}}
\newcommand\funcsubprogramtype[0]{\text{subprogram\_type}}
\newcommand\funcrecurselimit[0]{\text{recurse\_limit}}
\newcommand\funcbuiltin[0]{\text{builtin}}
\newcommand\funcqualifierfield[0]{\text{qualifier}}
\newcommand\funcoverride[0]{\text{override}}
\newcommand\GDkeyword[0]{\text{keyword}}
\newcommand\GDname[0]{\text{name}}
\newcommand\GDty[0]{\text{ty}}
\newcommand\GDinitialvalue[0]{\text{initial\_value}}
\newcommand\CasePattern[0]{\text{pattern}}
\newcommand\CaseWhere[0]{\text{where}}
\newcommand\CaseStmt[0]{\text{stmt}}
\newcommand\Forindexname[0]{\text{index\_name}}
\newcommand\Forstarte[0]{\text{start\_e}}
\newcommand\Fordir[0]{\text{dir}}
\newcommand\Forende[0]{\text{end\_e}}
\newcommand\Forbody[0]{\text{body}}
\newcommand\Forlimit[0]{\text{limit}}
\newcommand\EArrayLength[0]{\text{length}}
\newcommand\EArrayLabels[0]{\text{labels}}
\newcommand\EArrayValue[0]{\text{value}}
\newcommand\callname[0]{\text{name}}
\newcommand\callparams[0]{\text{params}}
\newcommand\callargs[0]{\text{args}}
\newcommand\callcalltype[0]{\text{call\_type}}
\newcommand\lhsaccessaccess[0]{\text{access}}
\newcommand\lhsaccessslices[0]{\text{slices}}
\newcommand\accessorpairisreadonly[0]{\text{is\_readonly}}
\newcommand\accessorpairgetter[0]{\text{getter}}
\newcommand\accessorpairsetter[0]{\text{setter}}

\newcommand\itemprefix[0]{\texttt{item}}

\newcommand\True[0]{\hyperlink{constant-True}{\texttt{TRUE}}}
\newcommand\False[0]{\hyperlink{constant-False}{\texttt{FALSE}}}

%%%%%%%%%%%%%%%%%%%%%%%%%%%%%%%%%%%%%%%%%%%%%%%%%%

\newcommand\torexpr[0]{\hyperlink{def-rexpr}{\textfunc{rexpr}}}

%%%%%%%%%%%%%%%%%%%%%%%%%%%%%%%%%%%%%%%%%%%%%%%%%%
%% Type System macros
%%%%%%%%%%%%%%%%%%%%%%%%%%%%%%%%%%%%%%%%%%%%%%%%%%
\newcommand\TypeErrorConfig[0]{\hyperlink{type-TypeErrorconfig}{\texttt{\#TE}}}
\newcommand\TypeErrorVal[1]{\TypeError(\texttt{#1})}

\newcommand\aslto[0]{\longrightarrow}

\newcommand\staticenvs[0]{\hyperlink{type-staticenvs}{\mathbb{SE}}}
\newcommand\staticenvironmentterm[0]{\hyperlink{type-staticenvs}{static environment}}
\newcommand\localstaticenvs[0]{\hyperlink{type-localstaticenvs}{\mathbb{LSE}}}
\newcommand\localstaticenvironmentterm[0]{\hyperlink{type-localstaticenvs}{local static environment}}
\newcommand\globalstaticenvs[0]{\hyperlink{type-globalstaticenvs}{\mathbb{GSE}}}
\newcommand\globalstaticenvironmentterm[0]{\hyperlink{type-globalstaticenvs}{global static environment}}
\newcommand\emptytenv[0]{\hyperlink{constant-emptytenv}{\emptyset_{\mathbb{SE}}}}
\newcommand\tstruct[0]{\hyperlink{def-tstruct}{\textfunc{get\_structure}}}
\newcommand\astlabel[0]{\hyperlink{def-astlabel}{\textfunc{ast\_label}}}
\newcommand\ProsetypesatTrue[3]{#2 \typesatisfiesterm{} #3 in the static environment #1}
\newcommand\ProsetypesatFalse[3]{#2 does not \typesatisfyterm{} #3 in the static environment #1}
\newcommand\typesat[0]{\hyperlink{def-typesatisfies}{\textfunc{type\_satisfies}}}

\newcommand\constantvalues[0]{\hyperlink{def-constantvalues}{\textsf{constant\_values}}}
\newcommand\globalstoragetypes[0]{\hyperlink{def-globalstoragetypes}{\textsf{global\_storage\_types}}}
\newcommand\localstoragetypes[0]{\hyperlink{def-localstoragetypes}{\textsf{local\_storage\_types}}}
\newcommand\returntype[0]{\hyperlink{def-returntype}{\textsf{return\_type}}}
\newcommand\declaredtypes[0]{\hyperlink{def-declaredtypes}{\textsf{declared\_types}}}
\newcommand\subtypes[0]{\hyperlink{def-subtypes}{\textsf{subtypes}}}
\newcommand\subprograms[0]{\hyperlink{def-subprograms}{\textsf{subprograms}}}
\newcommand\overloadedsubprograms[0]{\hyperlink{def-overloadedsubprograms}{\textsf{overloaded\_subprograms}}}
\newcommand\exprequiv[0]{\hyperlink{def-exprequiv}{\textsf{expr\_equiv}}}

\newcommand\Prosebinopliterals[2]{statically evaluating #1 yields #2}

\newcommand\Effect[0]{\textsf{Effect}}

%%%%%%%%%%%%%%%%%%%%%%%%%%%%%%%%%%%%%%%%%%%%%%%%%%
%% Semantics macros
%%%%%%%%%%%%%%%%%%%%%%%%%%%%%%%%%%%%%%%%%%%%%%%%%%
\newcommand\aslrel[0]{\bigtimes}
\newcommand\nativevalue[0]{\hyperlink{type-nativevalue}{\mathbb{V}}}
\newcommand\nvint[0]{\hyperlink{def-nvint}{\texttt{Int}}}
\newcommand\nvbool[0]{\hyperlink{def-nvbool}{\texttt{Bool}}}
\newcommand\nvreal[0]{\hyperlink{def-nvreal}{\texttt{Real}}}
\newcommand\nvstring[0]{\hyperlink{def-nvstring}{\texttt{String}}}
\newcommand\nvlabel[0]{\hyperlink{def-nvlabel}{\texttt{Label}}}
\newcommand\nvbitvector[0]{\hyperlink{def-nvbitvector}{\texttt{Bitvector}}}
\newcommand\tint[0]{\hyperlink{type-tint}{\mathcal{Z}}}
\newcommand\tbool[0]{\hyperlink{type-tbool}{\mathcal{B}}}
\newcommand\treal[0]{\hyperlink{type-treal}{\mathcal{R}}}
\newcommand\tstring[0]{\hyperlink{type-tstring}{\mathcal{S}\mathcal{T}\mathcal{R}}}
\newcommand\tbitvector[0]{\hyperlink{type-tbitvector}{\mathcal{B}\mathcal{V}}}
\newcommand\tvector[0]{\hyperlink{type-tvector}{\mathcal{V}\mathcal{E}\mathcal{C}}}
\newcommand\trecord[0]{\hyperlink{type-trecord}{\mathcal{R}\mathcal{E}\mathcal{C}}}
\newcommand\tlabel[0]{\hyperlink{type-tlabel}{\mathcal{L}\mathcal{A}\mathcal{B}\mathcal{E}\mathcal{L}}}

\newcommand\evalrel[0]{\hyperlink{def-evalrel}{\textsf{eval}}}
\newcommand\evalarrow[0]{\xrightarrow{\evalrel}}
\newcommand\envs[0]{\hyperlink{type-envs}{\mathbb{E}}}
\newcommand\dynamicenvs[0]{\hyperlink{type-dynamicenvs}{\mathbb{DE}}}
\newcommand\localdynamicenvs[0]{\hyperlink{type-localdynamicenvs}{\mathbb{LDE}}}
\newcommand\globaldynamicenvs[0]{\hyperlink{type-globaldynamicenvs}{\mathbb{GDE}}}
\newcommand\emptydenv[0]{\hyperlink{constant-emptydenv}{\emptyset_{\mathbb{DE}}}}
\newcommand\storage[0]{\hyperlink{def-storage}{\textsf{storage}}}
\newcommand\pendingcalls[0]{\hyperlink{def-pending-calls}{\textsf{pending\_calls}}}
\newcommand\xgraph[0]{\textsf{g}}
\newcommand\emptygraph[0]{{\hyperlink{constant-emptygraph}{\emptyset_\xgraph}}}
\newcommand\Proseemptygraph[1]{#1 is the empty \executiongraphterm{}}
\newcommand\asldata[0]{\hyperlink{constant-asldata}{\mathtt{asl\_data}}}
\newcommand\aslctrl[0]{\hyperlink{constant-aslctrl}{\mathtt{asl\_ctrl}}}
\newcommand\aslpo[0]{\hyperlink{constant-aslpo}{\mathtt{asl\_po}}}

\newcommand\Nodes[0]{\hyperlink{type-Nodes}{\mathcal{N}}}
\newcommand\Read[0]{\hyperlink{constant-Read}{\text{Read}}}
\newcommand\Write[0]{\hyperlink{constant-Write}{\text{Write}}}
\newcommand\Labels[0]{\hyperlink{type-Labels}{\mathcal{L}}}
\newcommand\XGraphs[0]{\hyperlink{type-XGraphs}{\mathcal{G}}}
\newcommand\WriteEffect[0]{\hyperlink{def-writeeffect}{\textsf{WriteEffect}}}
\newcommand\ReadEffect[0]{\hyperlink{def-readeffect}{\textsf{ReadEffect}}}

\newcommand\emptyenv[0]{\hyperlink{def-emptyenv}{\emptyset_{\mathbb{E}}}}
\newcommand\ordered[3]{{#1}\hyperlink{def-ordered}{\xrightarrow{#2}}{#3}}
\newcommand\parallelcomp[0]{\hyperlink{def-parallel}{\parallel}}

\newcommand\graphof[1]{\hyperlink{def-graphof}{\textfunc{graph}}({#1})}
\newcommand\withgraph[2]{{#1}(\hyperlink{def-withgraph}{\textfunc{graph}}\mapsto{#2})}
\newcommand\environof[1]{\hyperlink{def-environof}{\textfunc{environ}}({#1})}
\newcommand\withenviron[2]{{#1}(\hyperlink{def-withenviron}{\textfunc{environ}}\mapsto{#2})}

\newcommand\ContinuingConfig[0]{\hyperlink{def-continuingconfig}{\texttt{\#C}}}
\newcommand\ReturningConfig[0]{\hyperlink{def-returningconfig}{\texttt{\#R}}}
\newcommand\ThrowingConfig[0]{\hyperlink{def-throwingconfig}{\texttt{\#T}}}
\newcommand\DynErrorConfig[0]{\hyperlink{def-errorconfig}{\texttt{\#DE}}}
\newcommand\DivergingConfig[0]{\hyperlink{def-divergingconfig}{\texttt{\#DIV}}}
\newcommand\OrAbnormal[0]{\;\terminateas \ThrowingConfig, \DynErrorConfig, \DivergingConfig}
\newcommand\OrAbnormalReturning[0]{\;\terminateas \ReturningConfig, \ThrowingConfig, \DynErrorConfig, \DivergingConfig}
\newcommand\OrReturningOrAbnormal[0]{\;\terminateas \ReturningConfig, \ThrowingConfig, \DynErrorConfig, \DivergingConfig}
\newcommand\OrDynError[0]{\;\terminateas \DynErrorConfig}
\newcommand\OrDynErrorDiverging[0]{\;\terminateas \DynErrorConfig, \DivergingConfig}
\newcommand\ProseOtherwiseReturningOrAbnormal[0]{Otherwise, the result is either a returning configuration or abnormal.}
\newcommand\ProseOtherwiseAbnormal[0]{Otherwise, the result is abnormal.}
\newcommand\ProseOtherwiseDynamicError[0]{Otherwise, the result is a \dynamicerrorterm.}
\newcommand\ProseOtherwiseDynamicErrorOrDiverging[0]{Otherwise, the result is either a \dynamicerrorterm{} or the evaluation diverges.}
\newcommand\ProseOrAbnormal[0]{\ProseTerminateAs{\ThrowingConfig, \DynErrorConfig, \DivergingConfig}}
\newcommand\ProseOrAbnormalReturning[0]{\ProseTerminateAs{\ReturningConfig, \ThrowingConfig, \DynErrorConfig, \DivergingConfig}}
\newcommand\ProseReturningOrAbnormal[0]{\ProseTerminateAs{\TReturning, \ThrowingConfig, \DynErrorConfig, \DivergingConfig}}
\newcommand\ProseOrError[0]{\ProseTerminateAs{\DynErrorConfig}}
\newcommand\ProseOrDynErrorDiverging[0]{\ProseTerminateAs{\DynErrorConfig, \DivergingConfig}}

\newcommand\Proseevalexpr[3]{
    \hyperlink{relation-evalexpr}{evaluating} the expression #2 in the environment #1 yields #3}
\newcommand\Proseevallexpr[4]{\hyperlink{relation-evalexpr}{evaluating} the left-hand-side expression #2 in the environment #1 and #3, yields #4}
\newcommand\maskmatch[0]{\hyperlink{def-maskmatch}{\textfunc{mask\_match}}}
\newcommand\evallocaldecl[0]{\hyperlink{def-evallocaldecl}{\textfunc{eval\_local\_decl}}}
\newcommand\ldituplefolder[0]{\hyperlink{def-ldituplefolder}{\textfunc{ldi\_tuple\_folder}}}
\newcommand\poplocalscope[0]{\hyperlink{def-poplocalscope}{\textfunc{pop\_local\_scope}}}
\newcommand\Prosepoplocalscope[2]{#2 after \hyperlink{def-poplocalscope}{restoring} the variable bindings of #1 with the updated values of #2}
\newcommand\evalmultiassignment[0]{\hyperlink{relation-evalmultiassignment}{\textfunc{multi\_assign}}}
\newcommand\Prosebuildgenv[4]{\hyperlink{relation-buildgenv}{building} an environment from the static environment #1 and specification #2 yields #3 and the execution graph #4}
\newcommand\Proseevalspec[3]{\hyperlink{relation-evalspec}{evaluating} the typed AST #1 in the static environment #2 yields #3}
\newcommand\lexprisvar[0]{\hyperlink{def-lexprisvar}{\textfunc{lexpr\_is\_var}}}
\newcommand\desugarcasestmt[0]{\hyperlink{def-desugarcasestmt}{\textfunc{desugar\_case\_stmt}}}
\newcommand\casestocond[0]{\hyperlink{def-casestocond}{\textfunc{cases\_to\_cond}}}
\newcommand\casetocond[0]{\hyperlink{def-casetocond}{\textfunc{case\_to\_cond}}}
\newcommand\findcatcher[0]{\hyperlink{def-findcatcher}{\textfunc{find\_catcher}}}
\newcommand\Proseevallimit[3]{\hyperlink{relation-evallimit}{evaluating the limit expression} #2 in the static environment #1 yields #3}
\newcommand\Proseticklooplimit[2]{\hyperlink{relation-ticklooplimit}{decrementing} the optional loop limit value #1 yields the updated optional limit value #2}

\newcommand\outputtoconsole[0]{\hyperlink{relation-outputtoconsole}{\textfunc{output\_to\_console}}}
\newcommand\literaltostring[0]{\hyperlink{def-literaltostring}{\textfunc{literal\_to\_string}}}
\newcommand\Proseoutputtoconsole[1]{\hyperlink{def-outputtoconsole}{outputs} #1 to the console, if one exists}


\newcommand\DynamicError[0]{\hyperlink{type-DynamicError}{\textsf{DynError}}}
\newcommand\DynamicErrorConfigurationTerm[0]{\hyperlink{type-DynamicError}{dynamic error configuration}}

\newcommand\bitfunc[0]{\textfunc{bit}}

%%%%%%%%%%%%%%%%%%%%%%%%%%%%%%%%%%%%%%%%%%%%%%%%%%
% Typing macros
%%%%%%%%%%%%%%%%%%%%%%%%%%%%%%%%%%%%%%%%%%%%%%%%%%

\newcommand\ProseOtherwiseTypeError[0]{Otherwise, the result is a \typingerrorterm.}
\newcommand\OrTypeError[0]{\;\terminateas \TypeErrorConfig}
\newcommand\ProseOrTypeError[0]{\ProseTerminateAs{\TypeErrorConfig}}

\newcommand\annotaterel[0]{\hyperlink{def-annotaterel}{\textsf{type}}}
\newcommand\typearrow[0]{\xrightarrow{\annotaterel}}

\newcommand\dynamicdomain[0]{\hyperlink{def-dyndomain}{\textsf{dyn\_dom}}}

\newcommand\isbuiltinsingular[0]{\hyperlink{def-isbuiltinsingular}{\textfunc{is\_builtin\_singular}}}
\newcommand\isnamed[0]{\hyperlink{def-isnamed}{\textfunc{is\_named}}}
\newcommand\isanonymous[0]{\hyperlink{def-isanonymous}{\textfunc{is\_anonymous}}}
\newcommand\issingular[0]{\hyperlink{def-issingular}{\textfunc{is\_singular}}}
\newcommand\isstructured[0]{\hyperlink{def-isstructured}{\textfunc{is\_structured}}}

\newcommand\isunconstrainedinteger[0]{\hyperlink{def-isunconstrainedinteger}{\textsf{is\_unconstrained\_integer}}}
\newcommand\isparameterizedinteger[0]{\hyperlink{def-isparameterizedinteger}{\textsf{is\_parameterized\_integer}}}
\newcommand\iswellconstrainedinteger[0]{\hyperlink{def-iswellconstrainedinteger}{\textsf{is\_well\_constrained\_integer}}}
\newcommand\unconstrainedinteger[0]{\hyperlink{def-unconstrainedinteger}{\textsf{unconstrained\_integer}}}

\newcommand\Prosecheckconstrainedinteger[2]{\hyperlink{relation-checkconstrainedinteger}{checking} that the type #2 is a \constrainedintegerterm\ in
  the static environment #1 yields $\True$\ProseOrTypeError}

\newcommand\Prosestaticeval[3]{\hyperlink{def-staticeval}{statically evaluating} the expression #2 in the static environment #1 yields the literal #3}
\newcommand\staticeval[0]{\hyperlink{def-staticeval}{\textfunc{static\_eval}}}

\newcommand\Prosemakeanonymous[3]{\hyperlink{relation-makeanonymous}{obtaining} the \underlyingtypeterm\ of #2 in the static environment #1 yields #3}
\newcommand\subtypesrel[0]{\hyperlink{def-subtypesrel}{\textfunc{is\_subtype}}}
\newcommand\subtypesat[0]{\hyperlink{def-subtypesat}{\textfunc{subtype\_satisfies}}}
\newcommand\checktypesat[0]{\hyperlink{def-checktypesat}{\textfunc{checked\_typesat}}}
\newcommand\Prosetypeclashes[0]{\hyperlink{relation-typeclashes}{type-clashes}}
\newcommand\Prosetypeclash[0]{\hyperlink{relation-typeclashes}{type-clash}}
\newcommand\Prosetypeclashing[0]{\hyperlink{relation-typeclashes}{type-clashing}}
\newcommand\lca[0]{\hyperlink{def-lowestcommonancestor}{\textfunc{lowest\_common\_ancestor}}}
\newcommand\Proselca[0]{\hyperlink{def-lowestcommonancestor}{lowest common ancestor}}
\newcommand\namedlca[0]{\hyperlink{relation-namedlowestcommonancestor}{\textfunc{named\_lowest\_common\_ancestor}}}
\newcommand\Supers{\textsf{Supers}}
\newcommand\bitfieldsincluded[0]{\hyperlink{def-bitfieldsincluded}{\textfunc{bitfields\_included}}}
\newcommand\Prosecheckisnotcollection[2]{\hyperlink{relation-checkisnotcollection}{determining}
  whether #2 is not a \collectiontypeterm{} in #1 yields $\True$\ProseOrTypeError}
\newcommand\widthplus[0]{\hyperlink{def-widthplus}{\textfunc{width\_plus}}}
\newcommand\assocopt[0]{\hyperlink{def-assocopt}{\textfunc{assoc\_opt}}}
\newcommand\annotatefieldinit[0]{\hyperlink{def-annotatefieldinit}{\textfunc{annotate\_field\_init}}}
\newcommand\Proseannotatesymbolicallyevaluableexpr[3]{\hyperlink{relation-annotatesymbolicallyevaluableexpr}{annotating} the
    \symbolicallyevaluableterm\ expression #2 in the static environment #1 yields #3}
\newcommand\checkstructurelabel[0]{\hyperlink{def-checkstructurelabel}{\textfunc{check\_structure}}}
\newcommand\checkunderlyinginteger[0]{\hyperlink{def-checkunderlyinginteger}{\textfunc{check\_underlying\_integer}}}
\newcommand\checksymbolicallyevaluable[0]{\hyperlink{def-checksymbolicallyevaluable}{\textfunc{check\_symbolically\_evaluable}}}
\newcommand\Prosechecksymbolicallyevaluable[1]{\hyperlink{def-checksymbolicallyevaluable}{checking} that #1 is \symbolicallyevaluableterm\ yields $\True$\ProseOrTypeError}
\newcommand\getforconstraints[0]{\hyperlink{relation-getforconstraints}{\textfunc{for\_constraints}}}
\newcommand\findbitfieldsslices[0]{\hyperlink{def-findbitfieldsslices}{\textfunc{find\_bitfields\_slices}}}
\newcommand\Prosetypeof[3]{\hyperlink{relation-typeof}{obtaining} the type of #2 in the static environment #1 yields #3}
\newcommand\Prosecheckglobalpragma[2]{\hyperlink{relation-checkglobalpragma}{checking} the global pragma #2 for correctness in the static environment #1 yields $\True$}
\newcommand\paramsofty[0]{\hyperlink{relation-paramsofty}{\textfunc{parameters\_of\_ty}}}
\newcommand\paramsofexpr[0]{\hyperlink{relation-paramsofexpr}{\textfunc{parameters\_of\_expr}}}
\newcommand\paramsofconstraint[0]{\hyperlink{relation-paramsofconstraint}{\textfunc{parameters\_of\_constraint}}}
\newcommand\addimmutableexpression[0]{\hyperlink{relation-addimmutableexpression}{\textfunc{add\_immutable\_expr}}}
\newcommand\sort[0]{\hyperlink{def-sort}{\textfunc{sort}}}
\newcommand\useexpr[0]{\hyperlink{relation-useexpr}{\textfunc{use\_e}}} % SPECIFICATION_DEFERRED
\newcommand\uselexpr[0]{\hyperlink{relation-uselexpr}{\textfunc{use\_le}}} % SPECIFICATION_DEFERRED
\newcommand\usestmt[0]{\hyperlink{relation-usestmt}{\textfunc{use\_s}}} % SPECIFICATION_DEFERRED
\newcommand\Prosebuilddependencies[2]{\hyperlink{relation-builddependencies}{building} the \dependencygraphterm\ of #1 yields #2}

% Macros related to side effects

\newcommand\purity[0]{\hyperlink{type-TPurity}{purity}}
\newcommand\purities[0]{\hyperlink{type-TPurity}{purities}}
\newcommand\purityless[0]{\hyperlink{def-purityless}{<_{\text{pure}}}}
\newcommand\puritygeq[0]{\hyperlink{def-puritygeq}{\geq_{\text{pure}}}}

\newcommand\TSideEffectSet[0]{\pow{\TSideEffect}}

% sets of side effect descriptors
\newcommand\sideeffectispure[0]{\hyperlink{def-sideeffectispure}{\textfunc{side\_effect\_is\_pure}}}
\newcommand\sideeffectisreadonly[0]{\hyperlink{def-sideeffectisreadonly}{\textfunc{side\_effect\_is\_readonly}}}
\newcommand\sideeffectissymbolicallyevaluable[0]{\hyperlink{def-sideeffectissymbolicallyevaluable}{\textfunc{side\_effect\_symbolically\_evaluable}}}

% Symbolic equivalence testing macros
\newcommand\symmulexpr[0]{\hyperlink{def-symmulexpr}{\textfunc{sym\_mul\_expr}}}
\newcommand\compareidentifier[0]{\hyperlink{def-compareidentifier}{\textfunc{compare\_identifier}}}

\newcommand\domainsubset[0]{\hyperlink{def-domainsubset}{\textfunc{domain\_subset}}}

\newcommand\ProseOrTypeErrorOrCannotBeTransformed[0]{\ProseTerminateAs{\CannotBeTransformed,\TypeErrorConfig}}
\newcommand\Qnonzero[0]{\hyperlink{type-Qnonzero}{\Q{} \setminus \{0\}}}
\newcommand\mulpolynomials[0]{\hyperlink{def-mulpolynomials}{\textfunc{mul\_polynomials}}}
\newcommand\polynomialdividebyterm[0]{\hyperlink{def-polynomialdividebyterm}{polynomial\_divide\_by\_term}}


\newcommand\fieldnames[0]{\hyperlink{def-fieldnames}{\textfunc{field\_names}}}
\newcommand\fieldtype[0]{\hyperlink{def-fieldtype}{\textfunc{field\_type}}}

\newcommand\eliteral[1]{\ELiteral(#1)}

% An expression that we do not statically evaluate.
\newcommand\CannotBeTransformed[0]{\hyperlink{constant-CannotBeTransformed}{\top}}

%%%%%%%%%%%%%%%%%%%%%%%%%%%%%%%%%%
%% Primitive Operations macros
\newcommand\unopsignatures[0]{\hyperlink{def-unopsignatures}{\textfunc{unop\_signatures}}}
\newcommand\binopsignatures[0]{\hyperlink{def-binopsignatures}{\textfunc{binop\_signatures}}}

\newcommand\aslnotbool[0]{\hyperlink{def-notbool}{\text{not\_bool}}}
\newcommand\andbool[0]{\hyperlink{def-andbool}{\text{and\_bool}}}
\newcommand\orbool[0]{\hyperlink{def-orbool}{\text{or\_bool}}}
\newcommand\eqbool[0]{\hyperlink{def-eqbool}{\text{eq\_bool}}}
\newcommand\nebool[0]{\hyperlink{def-nebool}{\text{ne\_bool}}}
\newcommand\impliesbool[0]{\hyperlink{def-impliesbool}{\text{implies\_bool}}}
\newcommand\equivbool[0]{\hyperlink{def-equivbool}{\text{equiv\_bool}}}

\newcommand\addbits[0]{\hyperlink{def-addbits}{\text{add\_bits}}}
\newcommand\addbitsint[0]{\hyperlink{def-addbitsint}{\text{add\_bits\_int}}}
\newcommand\subbits[0]{\hyperlink{def-subbits}{\text{sub\_bits}}}
\newcommand\subbitsint[0]{\hyperlink{def-subbitsint}{\text{sub\_bits\_int}}}
\newcommand\notbits[0]{\hyperlink{def-notbits}{\text{not\_bits}}}
\newcommand\andbits[0]{\hyperlink{def-andbits}{\text{and\_bits}}}
\newcommand\orbits[0]{\hyperlink{def-orbits}{\text{or\_bits}}}
\newcommand\xorbits[0]{\hyperlink{def-xorbits}{\text{xor\_bits}}}
\newcommand\eqbits[0]{\hyperlink{def-eqbits}{\text{eq\_bits}}}
\newcommand\nebits[0]{\hyperlink{def-nebits}{\text{ne\_bits}}}
\newcommand\concatbits[0]{\hyperlink{def-concatbits}{\text{concat\_bits}}}

\newcommand\negateint[0]{\hyperlink{def-negateint}{\text{negate\_int}}}
\newcommand\addint[0]{\hyperlink{def-addint}{\text{add\_int}}}
\newcommand\subint[0]{\hyperlink{def-subint}{\text{sub\_int}}}
\newcommand\mulint[0]{\hyperlink{def-mulint}{\text{mul\_int}}}
\newcommand\expint[0]{\hyperlink{def-expint}{\text{exp\_int}}}
\newcommand\shiftleftint[0]{\hyperlink{def-shiftleftint}{\text{shiftleft\_int}}}
\newcommand\shiftrightint[0]{\hyperlink{def-shiftrightint}{\text{shiftright\_int}}}
\newcommand\divint[0]{\hyperlink{def-divint}{\text{div\_int}}}
\newcommand\fdivint[0]{\hyperlink{def-fdivint}{\text{fdiv\_int}}}
\newcommand\fremint[0]{\hyperlink{def-fremint}{\text{frem\_int}}}
\newcommand\eqint[0]{\hyperlink{def-eqint}{\text{eq\_int}}}
\newcommand\neint[0]{\hyperlink{def-neint}{\text{ne\_int}}}
\newcommand\leint[0]{\hyperlink{def-leint}{\text{le\_int}}}
\newcommand\ltint[0]{\hyperlink{def-ltint}{\text{lt\_int}}}
\newcommand\gtint[0]{\hyperlink{def-gtint}{\text{gt\_int}}}
\newcommand\geint[0]{\hyperlink{def-geint}{\text{ge\_int}}}

\newcommand\mulintreal[0]{\hyperlink{def-mulintreal}{\text{mul\_int\_real}}}
\newcommand\mulrealint[0]{\hyperlink{def-mulrealint}{\text{mul\_real\_int}}}
\newcommand\negatereal[0]{\hyperlink{def-negatereal}{\text{negate\_real}}}
\newcommand\addreal[0]{\hyperlink{def-addreal}{\text{add\_real}}}
\newcommand\subreal[0]{\hyperlink{def-subreal}{\text{sub\_real}}}
\newcommand\mulreal[0]{\hyperlink{def-mulreal}{\text{mul\_real}}}
\newcommand\expreal[0]{\hyperlink{def-expreal}{\text{exp\_real}}}
\newcommand\divreal[0]{\hyperlink{def-divreal}{\text{div\_real}}}
\newcommand\eqreal[0]{\hyperlink{def-eqreal}{\text{eq\_real}}}
\newcommand\nereal[0]{\hyperlink{def-nereal}{\text{ne\_real}}}
\newcommand\lereal[0]{\hyperlink{def-lereal}{\text{le\_real}}}
\newcommand\ltreal[0]{\hyperlink{def-ltreal}{\text{lt\_real}}}
\newcommand\gtreal[0]{\hyperlink{def-gtreal}{\text{gt\_real}}}
\newcommand\gereal[0]{\hyperlink{def-gereal}{\text{ge\_real}}}

\newcommand\eqstring[0]{\hyperlink{def-eqstring}{\text{eq\_string}}}
\newcommand\nestring[0]{\hyperlink{def-nestring}{\text{ne\_string}}}

\newcommand\eqenum[0]{\hyperlink{def-eqenum}{\text{eq\_enum}}}
\newcommand\neenum[0]{\hyperlink{def-neenum}{\text{ne\_enum}}}

% AST abbreviations
\newcommand\abbrevfont[1]{\scriptscriptstyle{\textsf{#1}}}
\newcommand{\ETrue}{\hyperlink{def-etrue}{\textsf{E\_True}}}
\newcommand\ELInt[1]{\overbracket{#1}^{\hyperlink{def-elint}{\ELiteral(\LInt)}}}
\newcommand\elint[1]{\ELiteral(\LInt(#1))}
\newcommand\AbbrevConstraintExact[1]{\overbracket{#1}^{\hyperlink{def-abbrevconstraintexact}{\abbrevfont{Constraint\_Exact}}}}
\newcommand\AbbrevConstraintRange[2]{\overbracket{#1..#2}^{\hyperlink{def-abbrevconstraintrange}{\abbrevfont{Constraint\_Range}}}}
\newcommand\AbbrevEBinop[3]{\overbracket{#2\ #1\ #3}^{\hyperlink{def-abbrevebinop}{\abbrevfont{E\_Binop}}}}
\newcommand\AbbrevTArray[2]{\overbracket{\texttt{array }[#1]\texttt{ of }#2}^{\hyperlink{def-abbrevtarray}{\abbrevfont{T\_Array}}}}
\newcommand\AbbrevTArrayLengthExpr[2]{\overbracket{\texttt{array [[}#1\texttt{]] of }#2}^{\hyperlink{def-abbrevtarraylengthexpr}{\abbrevfont{T\_Array(ArrayLength\_Expr)}}}}
\newcommand\AbbrevTArrayLengthEnum[3]{\overbracket{\texttt{array [[}#1 \# #2\texttt{]] of }#3}^{\hyperlink{def-abbrevtarraylengthenum}{\abbrevfont{T\_Array(Array\_Length\_Enum)}}}}
\newcommand\AbbrevEVar[1]{\overbracket{#1}^{\hyperlink{def-abbrevevar}{\abbrevfont{E\_Var}}}}

%%%%%%%%%%%%%%%%%%%%%%%%%%%%%%%%%%%%%%%%%%%%%%%%%%
%% Top level macros
%%%%%%%%%%%%%%%%%%%%%%%%%%%%%%%%%%%%%%%%%%%%%%%%%%
\newcommand\checkandinterpret[0]{\hyperlink{def-checkandinterpret}{\textfunc{check\_and\_interpret}}}
\newcommand\setbuiltin[0]{\hyperlink{def-setbuiltin}{\textfunc{set\_builtin}}}

\newcommand\renamelocals[0]{\hyperlink{def-renamelocals}{\textfunc{rename\_locals}}}
\newcommand\Proserenamelocals[2]{\hyperlink{def-renamelocals}{renaming}
  the local storage elements in the list of global declarations #1 yields the list of global declarations #2}
\newcommand\renamelocalsname[0]{\hyperlink{def-renamelocalsname}{\textfunc{rename\_locals\_name}}}
\newcommand\Proserenamelocalsname[2]{\hyperlink{def-renamelocalsname}{renames}
  the identifier #1, yielding the identifier #2}
\newcommand\renamelocalsdecl[0]{\hyperlink{def-renamelocalsdecl}{\textfunc{rename\_locals\_decl}}}
\newcommand\Proserenamelocalsdecl[2]{\hyperlink{def-renamelocalsdecl}{renaming}
  the local storage elements in declaration #1 yields the declaration #2}
\newcommand\renamelocalsfunc[0]{\hyperlink{def-renamelocalsfunc}{\textfunc{rename\_locals\_func}}}
\newcommand\Proserenamelocalsfunc[2]{\hyperlink{def-renamelocalsfunc}{renaming}
  the local storage elements in subprogram description #1 yields the function description #2}
\newcommand\renamelocalsargs[0]{\hyperlink{def-renamelocalsargs}{\textfunc{rename\_locals\_args}}}
\newcommand\Proserenamelocalsargs[2]{\hyperlink{def-renamelocalsargs}{renaming}
  the list local arguments #1 yields the list of arguments #2}
\newcommand\renamelocalsnamedargs[0]{\hyperlink{def-renamelocalsnamedargs}{\textfunc{rename\_locals\_named\_args}}}
\newcommand\Proserenamelocalsnamedargs[2]{\hyperlink{def-renamelocalsnamedargs}{renaming}
  the local storage elements in the list of parameters #1 yields the list of parameters #2}
\newcommand\renamelocalsstmt[0]{\hyperlink{def-renamelocalsstmt}{\textfunc{rename\_locals\_stmt}}}
\newcommand\Proserenamelocalsstmt[2]{\hyperlink{def-renamelocalsstmt}{renaming}
  the local storage elements in the statement #1 yields the statement #2}
\newcommand\renamelocalsexpr[0]{\hyperlink{def-renamelocalsexpr}{\textfunc{rename\_locals\_expr}}}
\newcommand\Proserenamelocalsexpr[2]{\hyperlink{def-renamelocalsexpr}{renaming}
  the local storage elements in the expression #1 yields the expression #2}
\newcommand\renamelocalslexpr[0]{\hyperlink{def-renamelocalslexpr}{\textfunc{rename\_locals\_lexpr}}}
\newcommand\Proserenamelocalslexpr[2]{\hyperlink{def-renamelocalslexpr}{renaming}
  the local storage elements in the assignable expression #1 yields the
  assignable expression #2}
\newcommand\renamelocalsldi[0]{\hyperlink{def-renamelocalsldi}{\textfunc{rename\_locals\_ldi}}}
\newcommand\Proserenamelocalsldi[2]{\hyperlink{def-renamelocalsldi}{renaming}
  the local storage elements in the local declaration item #1 yields
  the local declaration item #2}
\newcommand\renamelocalsty[0]{\hyperlink{def-renamelocalsty}{\textfunc{rename\_locals\_ty}}}
\newcommand\Proserenamelocalsty[2]{\hyperlink{def-renamelocalsty}{renaming}
  the local storage elements in the type #1 yields the type #2}
\newcommand\renamelocalsslice[0]{\hyperlink{def-renamelocalsslice}{\textfunc{rename\_locals\_slice}}}
\newcommand\renamelocalsconstraint[0]{\hyperlink{def-renamelocalsconstraint}{\textfunc{rename\_locals\_constraint}}}
\newcommand\renamelocalsarrayindex[0]{\hyperlink{def-renamelocalsarrayindex}{\textfunc{rename\_locals\_array\_index}}}
\newcommand\renamelocalspattern[0]{\hyperlink{def-renamelocalspattern}{\textfunc{rename\_locals\_pattern}}}
\newcommand\renamelocalscatcher[0]{\hyperlink{def-renamelocalscatcher}{\textfunc{rename\_locals\_catcher}}}

\newcommand\stdliblocalprefix[0]{\texttt{\_\_stdlib\_local\_}}

%%%%%%%%%%%%%%%%%%%%%%%%%%%%%%%%%%%%%%%%%%%%%%%%%%
%% Glossary
%%%%%%%%%%%%%%%%%%%%%%%%%%%%%%%%%%%%%%%%%%%%%%%%%%
\newcommand\ProseMSB[0]{MSB}
\newcommand\ProseLSB[0]{LSB}
\newcommand\Prosenormalconfiguration[0]{\hyperlink{def-normal}{normal configuration}}
\newcommand\Prosereturningconfiguration[0]{\hyperlink{type-Returning}{returning configuration}}
\newcommand\Prosecontinuingconfiguration[0]{\hyperlink{type-Continuing}{continuing configuration}}
\newcommand\errorcodeterm[0]{\hyperlink{def-errorcodeterm}{error code}}
\newcommand\errorcodesterm[0]{\hyperlink{def-errorcodeterm}{error codes}}
\newcommand\staticerrorterm[0]{\hyperlink{def-staticerrorterm}{static error}}
\newcommand\staticerrorsterm[0]{\hyperlink{def-staticerrorterm}{static errors}}
\newcommand\typingerrorterm[0]{\hyperlink{def-typingerrorterm}{type error}}
\newcommand\typingerrorsterm[0]{\hyperlink{def-typingerrorterm}{type errors}}
\newcommand\Typingerrorsterm[0]{\hyperlink{def-typingerrorterm}{Type errors}}
\newcommand\builderrorterm[0]{\hyperlink{def-builderrorterm}{build error}}
\newcommand\builderrorsterm[0]{\hyperlink{def-builderrorterm}{build errors}}
\newcommand\Builderrorsterm[0]{\hyperlink{def-builderrorterm}{Build errors}}
\newcommand\dynamicerrorterm[0]{\hyperlink{def-dynamicerrorterm}{dynamic error}}
\newcommand\dynamicerrorsterm[0]{\hyperlink{def-dynamicerrorterm}{dynamic errors}}

\newcommand\listprefixterm[0]{\hyperlink{def-listprefix}{prefix}}

\newcommand\head[0]{\hyperlink{def-head}{\texttt{head}}}
\newcommand\tail[0]{\hyperlink{def-tail}{\texttt{tail}}}
\newcommand\Proselist[2]{list with \head{} #1 and \tail{} #2}

\newcommand\inferencerule[0]{\hyperlink{def-inferencerule}{inference rule}}
\newcommand\inferencerules[0]{\hyperlink{def-inferencerule}{inference rules}}
\newcommand\shortcircuitrulemacros[0]{\hyperlink{def-shortcircuitrulemacro}{short-circuit rule macros}}
\newcommand\Shortcircuitrulemacros[0]{\hyperlink{def-shortcircuitrulemacro}{Short-circuit rule macros}}
\newcommand\specificationterm[0]{\hyperlink{def-specificationterm}{specification}}
\newcommand\globaldeclarationsterm[0]{\hyperlink{def-globaldeclarationterm}{global declarations}}

\newcommand\typedast[0]{\hyperlink{def-typedast}{typed AST}}
\newcommand\untypedast[0]{\hyperlink{def-untypedast}{untyped AST}}
\newcommand\syntacticsugar[0]{\hyperlink{def-syntacticsugar}{syntactic sugar}}
\newcommand\desugared[0]{\hyperlink{def-syntacticsugar}{desugared}}
\newcommand\desugaring[0]{\hyperlink{def-syntacticsugar}{desugaring}}

\newcommand\anonymoustype[0]{\hyperlink{def-isanonymous}{anonymous type}}
\newcommand\Anonymoustypes[0]{\hyperlink{def-isanonymous}{Anonymous types}}
\newcommand\namedtype[0]{\hyperlink{def-isnamed}{named type}}
\newcommand\namedtypes[0]{\hyperlink{def-isnamed}{named types}}
\newcommand\subtypesterm[0]{\hyperlink{def-subtypes}{subtypes}}
\newcommand\subtypeterm[0]{\hyperlink{def-subtypes}{subtype}}
\newcommand\supertypeterm[0]{\hyperlink{def-supertypeterm}{supertype}}
\newcommand\normalizedexpressionterm[0]{\hyperlink{def-normalizedterm}{normalized expression}}
\newcommand\symbolicexpressionterm[0]{\hyperlink{def-symbolicexpressionterm}{symbolic expression}}
\newcommand\symbolicexpressionsterm[0]{\hyperlink{def-symbolicexpressionterm}{symbolic expressions}}
\newcommand\equivalenttypesterm[0]{\hyperlink{def-equivalenttypesterm}{equivalent types}}
\newcommand\equivalentexprsterm[0]{\hyperlink{def-equivalentexprsterm}{equivalent expressions}}
\newcommand\equivalentconstraintsterm[0]{\hyperlink{def-equivalentconstraintsterm}{equivalent constraints}}

%% Type macros
\newcommand\integertypeterm[0]{\hyperlink{integertypeterm}{integer type}}
\newcommand\integertypesterm[0]{\hyperlink{integertypeterm}{integer types}}
\newcommand\bitvectortypeterm[0]{\hyperlink{bitvectortypeterm}{bitvector type}}
\newcommand\bitvectortypesterm[0]{\hyperlink{bitvectortypeterm}{bitvector types}}
\newcommand\bitfieldsterm[0]{\hyperlink{bitfieldterm}{bitfields}}
\newcommand\bitslicesterm[0]{\hyperlink{bitsliceterm}{bitslices}}
\newcommand\realtypeterm[0]{\hyperlink{realtypeterm}{real type}}
\newcommand\stringtypeterm[0]{\hyperlink{stringtypeterm}{string type}}
\newcommand\stringtypesterm[0]{\hyperlink{stringtypesterm}{string types}}
\newcommand\booleantypeterm[0]{\hyperlink{booleantypeterm}{boolean type}}
\newcommand\booleantypesterm[0]{\hyperlink{booleantypeterm}{boolean types}}
\newcommand\enumerationtypeterm[0]{\hyperlink{enumerationtypeterm}{enumeration type}}
\newcommand\enumerationtypesterm[0]{\hyperlink{enumerationtypeterm}{enumeration types}}
\newcommand\Enumerationtypesterm[0]{\hyperlink{enumerationtypeterm}{Enumeration types}}
\newcommand\tupletypeterm[0]{\hyperlink{tupletypeterm}{tuple type}}
\newcommand\tupletypesterm[0]{\hyperlink{tupletypeterm}{tuple types}}
\newcommand\Tupletypesterm[0]{\hyperlink{tupletypeterm}{Tuple types}}
\newcommand\Prosetupletype[1]{\tupletypeterm{} with member types #1}
\newcommand\arraytypeterm[0]{\hyperlink{arraytypeterm}{array type}}
\newcommand\arraytypesterm[0]{\hyperlink{arraytypeterm}{array types}}
\newcommand\intarraysterm[0]{\hyperlink{intarraytypeterm}{integer-indexed arrays}}
\newcommand\intarraytypeterm[0]{\hyperlink{intarraytypeterm}{integer-indexed array type}}
\newcommand\Intarraytypeterm[0]{\hyperlink{intarraytypeterm}{Integer-indexed array type}}
\newcommand\enumarraysterm[0]{\hyperlink{enumarraytypeterm}{Enumeration-indexed arrays}}
\newcommand\Enumarraytypeterm[0]{\hyperlink{enumarraytypeterm}{Enumeration-indexed array type}}
\newcommand\namedtypeterm[0]{\hyperlink{namedtypeterm}{named type}}
\newcommand\namedtypesterm[0]{\hyperlink{namedtypeterm}{named types}}
\newcommand\recordtypeterm[0]{\hyperlink{recordtypeterm}{record type}}
\newcommand\exceptiontypeterm[0]{\hyperlink{exceptiontypeterm}{exception type}}
\newcommand\collectiontypeterm[0]{\hyperlink{collectiontypeterm}{collection type}}
\newcommand\collectiontypesterm[0]{\hyperlink{collectiontypeterm}{collection types}}
\newcommand\recordtypesterm[0]{\hyperlink{recordtypeterm}{record types}}
\newcommand\unconstrainedintegertypeterm[0]{\hyperlink{def-unconstrainedintegertype}{unconstrained integer type}}
\newcommand\unconstrainedintegertypesterm[0]{\hyperlink{def-unconstrainedintegertype}{unconstrained integer types}}
\newcommand\parameterizedintegertypeterm[0]{\hyperlink{def-parameterizedintegertype}{parameterized integer type}}
\newcommand\parameterizedintegertypesterm[0]{\hyperlink{def-parameterizedintegertype}{parameterized integer types}}
\newcommand\wellconstrainedintegertypeterm[0]{\hyperlink{def-wellconstrainedintegertype}{well-constrained integer type}}
\newcommand\wellconstrainedintegertypesterm[0]{\hyperlink{def-wellconstrainedintegertype}{well-constrained integer types}}
\newcommand\pendingconstrainedintegertypeterm[0]{\hyperlink{def-pendingconstrainedintegertype}{pending constrained integer type}}
\newcommand\pendingconstrainedintegertypesterm[0]{\hyperlink{def-pendingconstrainedintegertype}{pending constrained integer types}}
\newcommand\structuredtypeterm[0]{\hyperlink{def-structuredtype}{structured type}}
\newcommand\structureterm[0]{\hyperlink{relation-getstructure}{structure}}
\newcommand\underlyingtypeterm[0]{\hyperlink{relation-makeanonymous}{underlying type}}
\newcommand\underlyingtypesterm[0]{\hyperlink{relation-makeanonymous}{underlying types}}
\newcommand\symbolicdomainterm[0]{\hyperlink{def-symbolicdomain}{symbolic domain}}
\newcommand\typesatisfiesterm[0]{\hyperlink{def-typesatisfies}{type-satisfies}}
\newcommand\typesatisfyterm[0]{\hyperlink{def-typesatisfies}{type-satisfy}}
\newcommand\subtypesatisfiesterm[0]{\hyperlink{def-subtypesat}{subtype-satisfies}}
\newcommand\subtypesatisfyterm[0]{\hyperlink{def-subtypesat}{subtype-satisfy}}
\newcommand\typeequivalentterm[0]{\hyperlink{relation-typeequal}{type-equivalent}}
\newcommand\bitwidthequivalentterm[0]{\hyperlink{relation-bitwidthequal}{bitwidth-equivalent}}
\newcommand\constrainedintegerterm[0]{\hyperlink{relation-checkconstrainedinteger}{constrained integer}}
\newcommand\constrainedintegersterm[0]{\hyperlink{relation-checkconstrainedinteger}{constrained integers}}
\newcommand\structureofinteger[0]{\hyperlink{def-checkunderlyinginteger}{structure of an integer}}
\newcommand\wellconstrainedstructureterm[0]{\hyperlink{relation-getwellconstrainedstructure}{well-constrained structure}}
\newcommand\wellconstrainedversionterm[0]{\hyperlink{relation-towellconstrained}{well-constrained version}}
\newcommand\namedlowestcommonancestorterm[0]{\hyperlink{relation-namedlowestcommonancestor}{named lowest common ancestor}}
\newcommand\arrayaccessterm[0]{\hyperlink{def-arrayaccess}{array access}}
\newcommand\symbolicallysimplifiesterm[0]{\hyperlink{def-symbolicallysimplifies}{symbolically simplifies}}
\newcommand\singulartypeterm[0]{\hyperlink{def-issingular}{singular type}}
\newcommand\singulartypesterm[0]{\hyperlink{def-issingular}{singular types}}

\newcommand\nativevalueterm[0]{\hyperlink{def-nativevalue}{native value}}
\newcommand\executiongraphterm[0]{\hyperlink{def-executiongraph}{execution graph}}
\newcommand\executiongraphs[0]{\hyperlink{def-executiongraph}{execution graphs}}
\newcommand\nativevaluesterm[0]{\hyperlink{def-nativevalues}{native values}}
\newcommand\concurrentnativevalue[0]{\hyperlink{def-concurrentnativevalue}{concurrent native value}}
\newcommand\concurrentnativevalues[0]{\hyperlink{def-concurrentnativevalue}{concurrent native values}}

\newcommand\singleslice[0]{\hyperlink{def-singleslice}{single slice}}
\newcommand\rangeslice[0]{\hyperlink{def-rangeslice}{range slice}}
\newcommand\lengthslice[0]{\hyperlink{def-lengthslice}{length slice}}
\newcommand\scaledslice[0]{\hyperlink{def-scaledslice}{scaled slice}}

\newcommand\rhsexpression[0]{\hyperlink{def-rhsexpression}{right-hand-side expression}}
\newcommand\rhsexpressions[0]{\hyperlink{def-rhsexpression}{right-hand-side expressions}}
\newcommand\assignableexpression[0]{\hyperlink{def-assignableexpression}{assignable expression}}
\newcommand\assignableexpressions[0]{\hyperlink{def-assignableexpression}{assignable expressions}}
\newcommand\localdeclarationkeyword[0]{\hyperlink{def-localdeclarationkeyword}{local declaration keyword}}
\newcommand\localdeclarationitem[0]{\hyperlink{def-localdeclarationitem}{local declaration item}}
\newcommand\localdeclaration[0]{\hyperlink{def-localdeclaration}{local declaration}}

\newcommand\basevalueterm[0]{\hyperlink{def-basevalueterm}{base value}}

\newcommand\defusedependencyterm[0]{\hyperlink{relation-builddependencies}{def-use dependency}}
\newcommand\defusedependenciesterm[0]{\hyperlink{relation-builddependencies}{def-use dependencies}}
\newcommand\dependencygraphterm[0]{\hyperlink{relation-builddependencies}{def-use dependency graph}}

\newcommand\casediscriminantterm[0]{\hyperlink{def-casediscriminantterm}{case discriminant}}
\newcommand\casealternativeterm[0]{\hyperlink{def-casealternativeterm}{case alternative}}
\newcommand\casealternativesterm[0]{\hyperlink{def-casealternativeterm}{case alternatives}}
\newcommand\otherwisecaseterm[0]{\hyperlink{def-otherwisecaseterm}{otherwise case}}

%% Constraint macros
\newcommand\exactconstraintterm[0]{\hyperlink{def-exactconstraintterm}{exact constraint}}
\newcommand\Proseexactconstraint[1]{\hyperlink{def-exactconstraintterm}{exact constraint} for the expression #1}
\newcommand\rangeconstraintterm[0]{\hyperlink{def-rangeconstraintterm}{range constraint}}
\newcommand\rangeconstraintsterm[0]{\hyperlink{def-rangeconstraintterm}{range constraints}}
\newcommand\Proserangeconstraint[2]{\hyperlink{def-rangeconstraintterm}{range constraint} for the lower end expression #1 and upper end expression #2}

%% Expression macros
\newcommand\literalexpressionterm[0]{\hyperlink{def-literalexpressionterm}{literal expression}}
\newcommand\literalexpressionsterm[0]{\hyperlink{def-literalexpressionterm}{literal expressions}}
\newcommand\variableexpressionterm[0]{\hyperlink{def-variableexpressionterm}{variable expression}}
\newcommand\variableexpressionsterm[0]{\hyperlink{def-variableexpressionterm}{variable expressions}}
\newcommand\variableexpression[1]{\hyperlink{def-variableexpressionterm}{variable expression} for the identifier #1}
\newcommand\atcexpressionterm[0]{\hyperlink{def-atceexpressionterm}{asserting type conversion}}
\newcommand\atcexpression[2]{\hyperlink{def-atceexpressionterm}{asserting type conversion} for the expression #1 and type #2}
\newcommand\binopexpressionterm[0]{\hyperlink{def-binopexpressionterm}{binary operation expression}}
\newcommand\binopexpressionsterm[0]{\hyperlink{def-binopexpressionterm}{binary operation expressions}}
\newcommand\binopexpression[3]{\hyperlink{def-binopexpressionterm}
  {binary operation expression} for the binary operator #1, left-hand-side expression #2, and right-hand-side expression #3}
\newcommand\unopexpressionterm[0]{\hyperlink{def-unopexpressionterm}{unary operation expression}}
\newcommand\unopexpressionsterm[0]{\hyperlink{def-unopexpressionterm}{unary operation expressions}}
\newcommand\unopexpression[2]{\hyperlink{def-unopexpressionterm}
  {unary operation expression} for the unary operator #1 and expression #2}
\newcommand\callexpressionterm[0]{\hyperlink{def-callexpressionterm}{call expression}}
\newcommand\callexpressionsterm[0]{\hyperlink{def-callexpressionterm}{call expressions}}
\newcommand\callexpression[4]{\hyperlink{def-callexpressionterm}
  {call expression} for the subprogram #1, list of parameters #2, list of arguments #3, and subprogram type #4}
\newcommand\slicingexpressionsterm[0]{\hyperlink{def-slicingexpressionsterm}{slicing expressions}}
\newcommand\sliceexpression[2]{\hyperlink{def-slicingexpressionsterm}{slicing expression} for the bitvector expression #1 and list of slices #2}
\newcommand\condexpressionterm[0]{\hyperlink{def-conditionexpressionterm}{conditional expression}}
\newcommand\condexpressionsterm[0]{\hyperlink{def-conditionexpressionterm}{conditional expressions}}
\newcommand\condexpression[3]{\hyperlink{def-conditionexpressionterm}
  {conditional expression} for the condition expression #1, left-hand-side expression #2, and right-hand-side expression #3}
\newcommand\arrayaccessexpressionsterm[0]{\hyperlink{def-arrayaccessexpressionsterm}{array access expressions}}
\newcommand\getarrayexpression[2]{\hyperlink{def-getarrayexpressionterm}
  {array read expression} for the array base expression #1 and index expression #2}
\newcommand\getenumarrayexpression[2]{\hyperlink{def-getenumarrayexpression}
{array read expression} for the array base expression #1 and enumeration-typed index expression #2}
\newcommand\getfieldexpression[2]{\hyperlink{def-getfieldexpressionterm}
  {field read expression} for the record base expression #1 and field identifier #2}
\newcommand\getfieldsexpression[2]{\hyperlink{def-getfieldsexpressionterm}
  {multi-field read expression} for the record base expression #1 and list of field identifiers #2}
\newcommand\getitemexpression[2]{\hyperlink{def-getfieldexpressionterm}
{tuple item expression} for the tuple base expression #1 and item index #2}
\newcommand\recordexpression[2]{\hyperlink{def-recordexpressionterm}
  {record construction expression} for the record type #1 and list of field initializers #2}
\newcommand\tupleexpression[1]{\hyperlink{def-tupleexpressionterm}
  {tuple expression} for the list of expressions #1}
\newcommand\arbitraryexpression[1]{\hyperlink{def-arbitraryexpressionterm}
  {\ARBITRARY{} expression} for the type #1}
\newcommand\patternexpressionterm[0]{\hyperlink{def-patternexpressionterm}{pattern expression}}

%% Assignable expression macros
\newcommand\discardlexprterm[0]{\hyperlink{def-discardlexprterm}{discarding assignment expression}}
\newcommand\varlexpr[1]{\hyperlink{def-varlexprterm}{assignable variable expression} for the identifier #1}
\newcommand\slicelexpr[2]{\hyperlink{def-slicelexprterm}{assignable slicing expression} for the assignable expression #1 and list of slices #1}
\newcommand\setarraylexpr[2]{\hyperlink{def-setarraylexprterm}{assignable array expression} for the assignable array base expression #1 and index expression #2}
\newcommand\setfieldslexpr[2]{\hyperlink{def-setfieldlexprterm}{assignable multi-field expression} for the assignable record base expression #1 and list of field identifier #2}
\newcommand\destructuringlexpr[1]{\hyperlink{def-destructuringlexprterm}{assignable multi-expression} for the list of assignable expressions #1}

%% Statement macros
\newcommand\passstatementterm[0]{\hyperlink{def-passstatementterm}{pass statement}}
\newcommand\passstatementsterm[0]{\hyperlink{def-passstatementterm}{pass statements}}
\newcommand\assignmentstatementterm[0]{\hyperlink{def-assignmentstatementterm}{assignment statement}}
\newcommand\assignmentstatement[2]{\hyperlink{def-assignmentstatementterm}
  {assignment statement} for the assignable expression #1 and right-hand-side expression #2}
\newcommand\declarationstatementterm[0]{\hyperlink{def-declarationstatementterm}{declaration statement}}
\newcommand\declarationstatement[4]{\hyperlink{def-declarationstatementterm}
  {declaration statement} for the local declaration keyword #1, local declaration item #2, optional type annotation #3, and optional initializing expression #4}
\newcommand\sequencingstatementterm[0]{\hyperlink{def-sequencestatementterm}{sequencing statement}}
\newcommand\sequencingstatement[2]{\hyperlink{def-sequencestatementterm}{sequencing statement for #1 followed by #2}}
\newcommand\callstatementterm[0]{\hyperlink{def-callstatementterm}{call statement}}
\newcommand\callstatement[4]{\hyperlink{def-callstatementterm}{call statement} for the subprogram #1 with list of arguments #2, list of parameters #3, and subprogram type #4}
\newcommand\conditionalstatementterm[0]{\hyperlink{def-conditionalstatementterm}{conditional statement}}
\newcommand\conditionalstatement[3]{\hyperlink{def-conditionalstatementterm}
  {conditional statement} with condition expression #1, \texttt{then} statement #2, and \texttt{else} statement #3 }
\newcommand\casestatementterm[0]{\hyperlink{def-casestatementterm}{case statement}}
\newcommand\assertionstatementterm[0]{\hyperlink{def-assertionstatementterm}{assertion statement}}
\newcommand\assertionstatement[1]{\hyperlink{def-assertionstatementterm}{assertion statement} for the expression #1}
\newcommand\whilestatementterm[0]{\hyperlink{def-whilestatementterm}{while statement}}
\newcommand\whilestatement[3]{\hyperlink{def-whilestatementterm}
  {while statement} for the condition #1, optional limit expression #2, and body statement #2}
\newcommand\repeatstatementsterm[0]{\hyperlink{def-repeatstatementterm}{repeat statements}}
\newcommand\repeatstatement[3]{\hyperlink{def-repeatstatementterm}
  {repeat statement} for the body statement #1, condition expression #2, and optional limit expression #3}
\newcommand\forstatementterm[0]{\hyperlink{def-forstatementterm}{for statement}}
\newcommand\forstatement[6]{\hyperlink{def-forstatementterm}
  {for statement} for the index variable #1, start expression #2, direction #3, end expression #4, body statement #5, and optional limit expression #6}
\newcommand\throwstatementsterm[0]{\hyperlink{def-throwstatementterm}{throw statements}}
\newcommand\throwstatement[1]{\hyperlink{def-throwstatementterm}{throw statement} for the optional exception expression #1}
\newcommand\trystatementterm[0]{\hyperlink{def-trystatementterm}{try statement}}
\newcommand\trystatementsterm[0]{\hyperlink{def-trystatementterm}{try statements}}
\newcommand\trystatement[3]{\hyperlink{def-trystatementterm}{try statement} with body statement #1, list of \texttt{catch} clauses #2, and optional \texttt{otherwise} statement #3}
\newcommand\returnstatementsterm[0]{\hyperlink{def-returnstatementterm}{return statements}}
\newcommand\returnstatement[1]{\hyperlink{def-returnstatementterm}{return statement} for the optional expression #1}
\newcommand\printstatementterm[0]{\hyperlink{def-printstatementterm}{print statement}}
\newcommand\printstatementsterm[0]{\hyperlink{def-printstatementterm}{print statements}}
\newcommand\printstatement[2]{\hyperlink{def-printstatementterm}{print statement} with list of expressions #1 and newline flag #2}
\newcommand\unreachablestatementterm[0]{\hyperlink{def-unreachablestatementterm}{unreachable statement}}
\newcommand\pragmastatementterm[0]{\hyperlink{def-pragmastatementterm}{pragma statement}}
\newcommand\pragmastatement[2]{\hyperlink{def-pragmastatementterm}{pragma statement} for the pragma #1 and list of expressions #2}

% Side effect macros
\newcommand\absolutenames[0]{\hyperlink{def-absolutename}{absolute names}}
\newcommand\absolutename[0]{\hyperlink{def-absolutename}{absolute name}}
\newcommand\absoluteslice[0]{\hyperlink{def-absoluteslice}{absolute slice}}
\newcommand\absoluteslices[0]{\hyperlink{def-absoluteslice}{absolute slices}}
\newcommand\absolutebitfield[0]{\hyperlink{def-absolutebitfield}{absolute bitfield}}
\newcommand\absolutebitfields[0]{\hyperlink{def-absolutebitfield}{absolute bitfields}}
\newcommand\bitfieldscope[0]{\hyperlink{def-bitfieldscope}{bitfield scope}}
\newcommand\LocalEffectTerm[0]{\hyperlink{def-term-LocalEffect}{local side effect descriptor}}
\newcommand\GlobalEffectTerm[0]{\hyperlink{def-term-GlobalEffect}{global side effect descriptor}}
\newcommand\ImmutabilityTerm[0]{\hyperlink{def-term-Immutability}{immutability side effect descriptor}}

\newcommand\staticallyevaluableterm[0]{\hyperlink{def-staticallyevaluable}{statically evaluable}}
\newcommand\pureterm[0]{\hyperlink{relation-sesispure}{pure}}
\newcommand\readonlyterm[0]{\hyperlink{relation-sesisreadonly}{read-only}}
\newcommand\symbolicallyevaluableterm[0]{\hyperlink{relation-issymbolicallyevaluable}{symbolically evaluable}}
\newcommand\sideeffectdescriptorterm[0]{\hyperlink{def-sideeffectdescriptorterm}{side effect descriptor}}
\newcommand\sideeffectdescriptorsterm[0]{\hyperlink{def-sideeffectdescriptorterm}{side effect descriptors}}
\newcommand\sideeffectsetterm[0]{\hyperlink{def-sideeffectdescriptorterm}{set of side effect descriptors}}
\newcommand\sideeffectdescriptorsetsterm[0]{\hyperlink{def-sideeffectdescriptorterm}{sets of side effect descriptors}}

\newcommand\approximationdirectionterm[0]{\hyperlink{def-approximationdirectionterm}{approximation direction}}

%%%%%%%%%%%%%%%%%%%%%%%%%%%%%%%%%%%%%%%%%%%%%%%%%%%%%%%%%%%%%%%%%%%%%%%%%%%%%%%%
%% Type functions
\newcommand\Proseapplybinoptypes[5]{\hyperlink{relation-applybinoptypes}{applying} #2 to the type #3 and type #4 in the static environment #1 yields the type #5}
\newcommand\explodeconstraint[0]{\hyperlink{def-explodeconstraint}{\textfunc{explode\_constraint}}}
\newcommand\maxconstraintsize[0]{2^{17}}
\newcommand\maxexplodedintervalsize[0]{2^{14}}
\newcommand\Proseannotateexpr[3]{\hyperlink{relation-annotateexpr}{annotating} the expression #2 in the static environment #1 yields #3}
\newcommand\Proseannotatelexpr[4]{\hyperlink{relation-annotatelexpr}{annotating}
    the assignable expression #2 with the right-hand side type #3 in the static environment #1 yields #4}
\newcommand\annotateexprlist[0]{\hyperlink{relation-annotateexprs}{\textfunc{annotate\_exprs}}}
\newcommand\annotateslice[0]{\hyperlink{def-annotateslice}{\textfunc{annotate\_slice}}}
\newcommand\annotatelocaldeclitem[0]{\hyperlink{def-annotatelocaldeclitem}{\textfunc{annotate\_local\_decl\_item}}}
\newcommand\Prosenoprecisionloss[1]{determining whether #1 has been computed with no precision loss via\\
    $\checknoprecisionloss$ yields $\True$\ProseOrTypeError}
\newcommand\Proseannotatesubprogram[5]{\hyperlink{relation-annotatesubprogram}{annotating} the subprogram definition #2
  in the static environment #1 with \sideeffectdescriptorsetsterm{} #3 yields the subprogram definition #4
  and \sideeffectdescriptorsetsterm{} #5}

\newcommand\Proseabstractconfiguration{\hyperlink{type-abstractconfiguration}{abstract configuration}}
\newcommand\Proseabstractconfigurations{\hyperlink{type-abstractconfiguration}{abstract configurations}}
\newcommand\AbsConfig{\hyperlink{type-abstractconfiguration}{\textsf{AbsConfig}}}

\newcommand\Prosetypecheckast[4]{\hyperlink{relation-typecheckast}{typechecking} #1 in #2 yields the typed AST #3 and static environment #4}
\newcommand\Proseoverridesubprograms[0]{\hyperlink{relation-overridesubprograms}{overriding subprograms}}
\newcommand\Proseprocessoverrides[2]{\hyperlink{relation-processoverrides}{processing overrides} in #1 and #2}
\newcommand\Prosecheckimplementationsunique[1]{\hyperlink{relation-checkimplementationsunique}{checking} that implementations are unique in #1}
\newcommand\Prosesignaturesmatch[2]{\hyperlink{relation-signaturesmatch}{checking} that the signatures of #1 and #2 match}
\newcommand\Proserenamesubprograms[2]{\hyperlink{relation-renamesubprograms}{renaming} subprograms in #1 yields #2}
\newcommand\ProseSCC[3]{\hyperlink{def-scc}{partitioning} the set of declarations #1 with the set of edges #2 yields the list of strongly-connected components #3}
\newcommand\SCC[0]{\hyperlink{def-scc}{\textfunc{SCC}}}
\newcommand\topologicalorderingcomps[0]{\hyperlink{def-topologicalorderingcomps}{\textfunc{topological\_ordering\_comps}}}
\newcommand\Prosetopologicalorderingcomps[3]{\hyperlink{def-topologicalorderingcomps}{ordering} the set of strongly-connected components #1, with respect to #2, yields the list of strongly-connected components #3}
\newcommand\Proseannotatedeclcomps[4]{\hyperlink{relation-annotatedeclcomps}{annotating} the list of declaration components #2 in the global static environment #1 yields the list of annotated declarations #3 and new global static environment #4}
\newcommand\Prosecheckcommonbitfieldsalign[3]{\hyperlink{relation-checkcommonbitfieldsalign}{checking} that all pairs of bitfields in #2 that are in the same scope
  and share the same name correspond to the same slice of the containing bitvector type in the static environment #1 yields $\True$}
\newcommand\TAbsField[0]{\hyperlink{type-TAbsField}{\mathbb{A}\mathbb{B}\mathbb{F}}}
\newcommand\Prosebitfieldstoabsolute[4]{\hyperlink{relation-bitfieldstoabsolute}{generating} the \absolutebitfields\ for the list of bitfields #2 and its nested bitfields
  with #3 as the parent \absolutebitfield\ in the static environment #1 yields the set of fields #4}
\newcommand\Proseslicetoindices[3]{\hyperlink{relation-slicetoindices}{obtaining} the sequence of indices for a slice #1 in the static environment #2 yields #3}
\newcommand\Proseselectindicesbyslices[3]{\hyperlink{relation-selectindicesbyslices}{selecting}
  the integers from the list #1 specified by the list of indices #2 yields the list #3}

\newcommand\filteroptionlist[0]{\hyperlink{def-filteroptionlist}{\textfunc{filter\_option\_list}}}
\newcommand\bintounsigned[0]{\hyperlink{def-bintounsigned}{\textsf{binary\_to\_unsigned}}}
\newcommand\inttobits[0]{\hyperlink{def-inttobits}{\textfunc{int\_to\_bits}}}

% Symbolic domain subsumption
\newcommand\CannotOverapproximate[0]{\hyperlink{constant-CannotOverapproximate}{\overline{\textsf{Approx}}}}
\newcommand\CannotUnderapproximate[0]{\hyperlink{constant-CannotUnderapproximate}{\underline{\textsf{Approx}}}}
\newcommand\symdomsubsettest[0]{\hyperlink{def-symdomsubsettest}{\textfunc{symdom\_subset\_test}}}
\newcommand\Proseapproxconstraints[4]{\hyperlink{relation-approxconstraints}
  {Overapproximating} the list of constraints #3 in the static environment #1 with the \approximationdirectionterm\ #2
  yields the set of integers #3}
\newcommand\ProseapproxconstraintsOver[3]{\hyperlink{relation-approxconstraints}
  {Overapproximating} the list of constraints #2 in the static environment #1 yields the set of
  integers #3}
\newcommand\ProseapproxconstraintsUnder[3]{\hyperlink{relation-approxconstraints}
  {Underapproximating} the list of constraints #2 in the static environment #1 yields the set of
  integers #3}
\newcommand\Proseapproxconstraint[4]{\hyperlink{relation-approxconstraint}
  {approximating} the constraint #2 for all environments consisting of the static environment
  #1 with the \approximationdirectionterm\ #3 yields the set #4}
\newcommand\Proseapproxexpr[4]{\hyperlink{relation-approxexpr}
  {approximating} the set integers represented by the expression #3 in the static environment #1
  with the symbol #2 yields #4}
\newcommand\Proseapproxexprmin[3]{\hyperlink{relation-approxexprmin}
  {approximating} the minimal integer in the set of integers represented by the expression #2
  in the static environment #1 yields #3}
\newcommand\Proseapproxexprmax[3]{\hyperlink{relation-approxexprmax}
  {approximating} the maximal integer in the set of integers represented by the expression #2
  in the static environment #1 yields #3}
\newcommand\Proseapproxtype[4]{\hyperlink{relation-approxtype}
  {approximating} the type #3 in any environment consisting of the static environment #1
  with the \approximationdirectionterm\ #2 yields the set #4}

%% Build Error Codes
\newcommand\BuildErrorPrefix[0]{\texttt{BE}}
\newcommand\BuildErrorCode[1]{\texttt{\BuildErrorPrefix\_#1}}
\newcommand\LexicalError[0]{\hyperlink{def-lexicalerrorresult}{\texttt{\#}\BuildErrorCode{LE}}}
\newcommand\LexicalErrorConfig[0]{\LexicalError}
\newcommand\ParseError[0]{\hyperlink{def-parseerror}{\BuildErrorCode{PE}}}
\newcommand\ParseErrorConfig[0]{\hyperlink{def-parseerror}{\texttt{\#}\BuildErrorCode{PE}}}
\newcommand\ReservedIdentifier[0]{\hyperlink{def-reservedidentifier}{\BuildErrorCode{RI}}}
\newcommand\BinopPrecedence[0]{\hyperlink{def-binopprecedence}{\BuildErrorCode{BOP}}}
\newcommand\BuildBadDeclaration[0]{\hyperlink{def-buildbaddeclaration}{\BuildErrorCode{BD}}}

%% Type Error Codes
\newcommand\TypeErrorPrefix[0]{\texttt{TE}}
\newcommand\TypeErrorCode[1]{\texttt{\TypeErrorPrefix\_#1}}
\newcommand\UndefinedIdentifier[0]{\hyperlink{def-undefinedidentifier}{\TypeErrorCode{UI}}}
\newcommand\IdentifierAlreadyDeclared[0]{\hyperlink{def-identifieralreadydeclared}{\TypeErrorCode{IAD}}}
\newcommand\PrecisionLostDefining[0]{\hyperlink{def-precisionlostdefining}{\TypeErrorCode{PLD}}}
\newcommand\AssignmentToImmutable[0]{\hyperlink{def-aim}{\TypeErrorCode{AIM}}}
\newcommand\TypeSatisfactionFailure[0]{\hyperlink{def-typesatisfactionfailure}{\TypeErrorCode{TSF}}}
\newcommand\StaticEvaluationFailure[0]{\hyperlink{def-staticevaluationfailure}{\TypeErrorCode{SEF}}}
\newcommand\NoLCA[0]{\hyperlink{def-nolca}{\TypeErrorCode{LCA}}}
\newcommand\NoBaseValue[0]{\hyperlink{def-nobasevalue}{\TypeErrorCode{NBV}}}
\newcommand\TypeAssertionFailure[0]{\hyperlink{def-typeassertionfailure}{\TypeErrorCode{TAF}}}
\newcommand\BadOperands[0]{\hyperlink{def-badoperands}{\TypeErrorCode{BO}}}
\newcommand\UnexpectedType[0]{\hyperlink{def-unexpectedtype}{\TypeErrorCode{UT}}}
\newcommand\BadTupleIndex[0]{\hyperlink{def-badtupleindex}{\TypeErrorCode{BTI}}}
\newcommand\BadSlices[0]{\hyperlink{def-badslices}{\TypeErrorCode{BS}}}
\newcommand\BadField[0]{\hyperlink{def-badfield}{\TypeErrorCode{BF}}}
\newcommand\BadSubprogramDeclaration[0]{\hyperlink{def-badsubprogramdeclaration}{\TypeErrorCode{BSPD}}}
\newcommand\BadDeclaration[0]{\hyperlink{def-baddeclaration}{\TypeErrorCode{BD}}}
\newcommand\BadCall[0]{\hyperlink{def-badcall}{\TypeErrorCode{BC}}}
\newcommand\SideEffectViolation[0]{\hyperlink{def-sideeffectviolation}{\TypeErrorCode{SEV}}}
\newcommand\OverridingError[0]{\hyperlink{def-overridingerror}{\TypeErrorCode{OE}}}

%% Dynamic Error Codes
\newcommand\DynamicErrorPrefix[0]{\texttt{DE}}
\newcommand\DynamicErrorVal[1]{\DynamicError(\texttt{#1})}
\newcommand\DynamicErrorCode[1]{\texttt{\DynamicErrorPrefix\_#1}}
\newcommand\UnreachableError[0]{\hyperlink{def-unreachableerror}{\DynamicErrorCode{UNR}}}
\newcommand\DynamicAssertionFailure[0]{\hyperlink{def-dynamicassertionfailure}{\DynamicErrorCode{DAF}}}
\newcommand\DynamicTypeAssertionFailure[0]{\hyperlink{def-dynamictypeassertionfailure}{\DynamicErrorCode{TAF}}}
\newcommand\ArbitraryEmptyType[0]{\hyperlink{def-arbitraryemptytype}{\DynamicErrorCode{AET}}}
\newcommand\DynamicBadOperands[0]{\hyperlink{def-dynamicbadoperands}{\DynamicErrorCode{BO}}}
\newcommand\LimitExceeded[0]{\hyperlink{def-limitexceeded}{\DynamicErrorCode{LE}}}
\newcommand\UncaughtException[0]{\hyperlink{def-uncaughtexception}{\DynamicErrorCode{UE}}}
\newcommand\BadIndex[0]{\hyperlink{def-badindex}{\DynamicErrorCode{BI}}}
\newcommand\OverlappingSliceAssignment[0]{\hyperlink{def-overlappingsliceassignment}{\DynamicErrorCode{OSA}}}
\newcommand\NegativeArrayLength[0]{\hyperlink{def-negativearraylength}{\DynamicErrorCode{NAL}}}
\newcommand\NoEntryPoint[0]{\hyperlink{def-noentrypoint}{\DynamicErrorCode{NEP}}}

%%%%%%%%%%%%%%%%%%%%%%%%%%%%%%%%%%%%%%%%%%%%%%%%%%
% Semantics/Typing Shared macros
\newcommand\ProseTerminateAs[1]{\hyperlink{def-proseterminateas}{${}^{\sslash #1 }$}}
\newcommand\aslsep[0]{\mathbf{,}}

\newcommand\discardvar[0]{\texttt{-}}
\newcommand\discardvarstr[0]{\texttt{"-"}}
\newcommand\ttdotstr[0]{\texttt{"."}}

%%%%%%%%%%%%%%%%%%%%%%%%%%%%%%%%%%%%%%%%%%%%%%%%%%
% LRM Ident info
\newcommand\ident[2]{}
%\newcommand\ident[2]{\texttt{#1}\textsubscript{\texttt{\MakeUppercase{#2}}}}
\newcommand\identi[1]{\ident{I}{#1}}
\newcommand\identr[1]{\ident{R}{#1}}
\newcommand\identd[1]{\ident{D}{#1}}
\newcommand\identg[1]{\ident{G}{#1}}

%%%%%%%%%%%%%%%%%%%%%%%%%%%%%%%%%%%%%%%%%%%%%%%%%%
% Macros for typesetting variables.
% These will all go away eventually, when aslspec
% generates variable names directly.
\newcommand\acc[0]{\texttt{acc}}
\newcommand\accp[0]{\texttt{acc'}}
\newcommand\actualargs[0]{\texttt{actual\_args}}
\newcommand\actuals[0]{\texttt{actuals}}
\newcommand\actualsone[0]{\texttt{actuals1}}
\newcommand\argdecls[0]{\texttt{arg\_decls}}
\newcommand\args[0]{\texttt{args}}
\newcommand\argnames[0]{\texttt{arg\_names}}
\newcommand\argtypes[0]{\texttt{arg\_types}}
\newcommand\argtypesone[0]{\texttt{arg\_types1}}
\newcommand\argtys[0]{\texttt{a\_tys}}
\newcommand\aritymatch[0]{\texttt{arity\_match}}
\newcommand\bd[0]{\texttt{bd}}
\newcommand\bfone[0]{\texttt{bf1}}
\newcommand\bfoneone[0]{\texttt{bf1\_1}}
\newcommand\bfsone[0]{\texttt{bfs1}}
\newcommand\bfss[0]{\texttt{bfs\_s}}
\newcommand\bfst[0]{\texttt{bfs\_t}}
\newcommand\bfstwo[0]{\texttt{bfs2}}
\newcommand\bfstwop[0]{\texttt{bfs2'}}
\newcommand\bftwo[0]{\texttt{bf2}}
\newcommand\bftwoone[0]{\texttt{bf2\_1}}
\newcommand\bitfields[0]{\texttt{bitfields}}
\newcommand\bitfieldsp[0]{\texttt{bitfields'}}
\newcommand\bitfieldspp[0]{\texttt{bitfields''}}
\newcommand\bits[0]{\texttt{bits}}
\newcommand\body[0]{\texttt{body}}
\newcommand\buff[0]{\texttt{buf}}
\newcommand\bv[0]{\texttt{bv}}
\newcommand\callee[0]{\texttt{callee}}
\newcommand\calleeargtypes[0]{\texttt{callee\_arg\_types}}
\newcommand\callerargtypes[0]{\texttt{caller\_arg\_types}}
\newcommand\calltype[0]{\texttt{call\_type}}
\newcommand\candidates[0]{\texttt{candidates}}
\newcommand\candidatesone[0]{\texttt{candidates1}}
\newcommand\catchers[0]{\texttt{catchers}}
\newcommand\catchersone[0]{\texttt{catchers1}}
\newcommand\catchersp[0]{\texttt{catchers'}}
\newcommand\comp[0]{\texttt{comp}}
\newcommand\compare[0]{\texttt{compare}}
\newcommand\compdecls[0]{\texttt{comp\_decls}}
\newcommand\compfordir[0]{\texttt{comp\_for\_dir}}
\newcommand\orderedcomps[0]{\texttt{orderedcomps}}
\newcommand\comps[0]{\texttt{comps}}
\newcommand\compsone[0]{\texttt{comps1}}
\newcommand\cond[0]{\texttt{cond}}
\newcommand\condm[0]{\texttt{cond\_m}}
\newcommand\constraints[0]{\texttt{constraints}}
\newcommand\cs[0]{\texttt{cs}}
\newcommand\csone[0]{\texttt{cs1}}
\newcommand\csonearg[0]{\texttt{cs1\_arg}}
\newcommand\csonee[0]{\texttt{cs1\_e}}
\newcommand\csonep[0]{\texttt{cs1'}}
\newcommand\css[0]{\texttt{cs\_s}}
\newcommand\cst[0]{\texttt{cs\_t}}
\newcommand\cstwo[0]{\texttt{cs2}}
\newcommand\cstwoarg[0]{\texttt{cs2\_arg}}
\newcommand\cstwoe[0]{\texttt{cs2\_e}}
\newcommand\cstwof[0]{\texttt{cs2\_f}}
\newcommand\csvanilla[0]{\texttt{cs\_vanilla}}
\newcommand\declaredt[0]{\texttt{declared\_t}}
\newcommand\decls[0]{\texttt{decls}}
\newcommand\declsone[0]{\texttt{decls1}}
\newcommand\declsp[0]{\texttt{decls'}}
\newcommand\declstwo[0]{\texttt{decls2}}
\newcommand\defs[0]{\texttt{defs}}
\newcommand\denvone[0]{\textsf{denv1}}
\newcommand\denvthrow[0]{\texttt{denv\_throw}}
\newcommand\denvtwo[0]{\textsf{denv2}}
\newcommand\dependencies[0]{\texttt{depends}}
\newcommand\reverseddependencies[0]{\texttt{rev\_deps}}
\newcommand\dir[0]{\texttt{dir}}
\newcommand\ds[0]{\texttt{ds}}
\newcommand\dst[0]{\texttt{dst}}
\newcommand\dt[0]{\texttt{dt}}
\newcommand\earray[0]{\texttt{e\_array}}
\newcommand\eactual[0]{\texttt{e\_actual}}
\newcommand\ebase[0]{\texttt{e\_base}}
\newcommand\lebase[0]{\texttt{le\_base}}
\newcommand\ebaseannot[0]{\texttt{e\_base\_annot}}
\newcommand\lebaseannot[0]{\texttt{le\_base\_annot}}
\newcommand\ebasep[0]{\texttt{e\_base'}}
\newcommand\ebv[0]{\texttt{e\_bv}}
\newcommand\econd[0]{\texttt{e\_cond}}
\newcommand\econdp[0]{\texttt{e\_cond'}}
\newcommand\efactor[0]{\texttt{factor}}
\newcommand\efalse[0]{\texttt{e\_false}}
\newcommand\efalsep[0]{\texttt{e\_false'}}
\newcommand\efields[0]{\texttt{e\_fields}}
\newcommand\eindex[0]{\texttt{e\_index}}
\newcommand\eindexp[0]{\texttt{e\_index'}}
\newcommand\elength[0]{\texttt{length}}
\newcommand\elengthp[0]{\texttt{length'}}
\newcommand\elist[0]{\texttt{e\_list}}
\newcommand\envandfs[0]{\texttt{env\_and\_fs}}
\newcommand\envandfsone[0]{\texttt{env\_and\_fs1}}
\newcommand\envandfstwo[0]{\texttt{env\_and\_fs2}}
\newcommand\envcont[0]{\texttt{env\_cont}}
\newcommand\envfour[0]{\texttt{env4}}
\newcommand\envm[0]{\texttt{envm}}
\newcommand\envone[0]{\texttt{env1}}
\newcommand\envp[0]{\texttt{env'}}
\newcommand\envret[0]{\texttt{env\_ret}}
\newcommand\envthree[0]{\texttt{env3}}
\newcommand\envthrow[0]{\texttt{env\_throw}}
\newcommand\envtwo[0]{\texttt{env2}}
\newcommand\eoffset[0]{\texttt{offset}}
\newcommand\eoffsetp[0]{\texttt{offset'}}
\newcommand\eonen[0]{\texttt{e1\_n}}
\newcommand\eqs[0]{\texttt{eqs}}
\newcommand\eqsthree[0]{\texttt{eqs3}}
\newcommand\erecord[0]{\texttt{e\_record}}
\newcommand\es[0]{\texttt{es}}
\newcommand\estart[0]{\texttt{e\_start}}
\newcommand\estartp[0]{\texttt{e\_start'}}
\newcommand\eend[0]{\texttt{e\_end}}
\newcommand\eendp[0]{\texttt{e\_end'}}
\newcommand\etop[0]{\texttt{e\_top}}
\newcommand\etrue[0]{\texttt{e\_true}}
\newcommand\etruep[0]{\texttt{e\_true'}}
\newcommand\etwon[0]{\texttt{e2\_n}}
\newcommand\ety[0]{\texttt{e\_ty}}
\newcommand\etuple[0]{\texttt{e\_tuple}}
\newcommand\evalue[0]{\texttt{e\_value}}
\newcommand\ekey[0]{\texttt{e\_key}}
\newcommand\ewidth[0]{\texttt{e\_width}}
\newcommand\ewidthone[0]{\texttt{e\_width1}}
\newcommand\ewidthp[0]{\texttt{e\_width'}}
\newcommand\explodedinterval[0]{\texttt{exploded\_interval}}
\newcommand\expropt[0]{\texttt{expr\_opt}}
\newcommand\exprs[0]{\texttt{exprs}}
\newcommand\exprsone[0]{\texttt{exprs1}}
\newcommand\extrafields[0]{\texttt{extra\_fields}}
\newcommand\factor[0]{\texttt{factor}}
\newcommand\falsep[0]{\texttt{false'}}
\newcommand\vfield[0]{\texttt{field}}
\newcommand\fieldmap[0]{\texttt{field\_map}}
\newcommand\fieldmapp[0]{\texttt{field\_map'}}
\newcommand\fieldname[0]{\texttt{field\_name}}
\newcommand\fields[0]{\texttt{fields}}
\newcommand\fieldsone[0]{\texttt{fields1}}
\newcommand\fieldopt[0]{\texttt{field\_opt}}
\newcommand\fieldopts[0]{\texttt{field\_opts}}
\newcommand\fieldsp[0]{\texttt{fields'}}
\newcommand\fieldtypes[0]{\texttt{field\_types}}
\newcommand\formals[0]{\texttt{formals}}
\newcommand\formaltypes[0]{\texttt{formal\_types}}
\newcommand\formaltys[0]{\texttt{f\_tys}}
\newcommand\funcdef[0]{\texttt{func\_def}}
\newcommand\funcsig[0]{\texttt{func\_sig}}
\newcommand\funcsigone[0]{\texttt{func\_sig1}}
\newcommand\funcsigargs[0]{\texttt{func\_sig\_args}}
\newcommand\funcsigargsone[0]{\texttt{func\_sig\_args1}}
\newcommand\funcsigfone[0]{\texttt{func\_sig\_f1}}
\newcommand\funcsigp[0]{\texttt{func\_sig'}}
\newcommand\funcsigparams[0]{\texttt{func\_sig\_params}}
\newcommand\funcsigparamsone[0]{\texttt{func\_sig\_params1}}
\newcommand\funcsigrettyopt[0]{\texttt{func\_sig\_ret\_ty\_opt}}
\newcommand\gdk[0]{\texttt{gdk}}
\newcommand\genv[0]{\texttt{genv}}
\newcommand\genvone[0]{\texttt{genv1}}
\newcommand\genvtwo[0]{\texttt{genv2}}
\newcommand\gsd[0]{\texttt{gsd}}
\newcommand\gsdp[0]{\texttt{gsd'}}
\newcommand\icsdown[0]{\texttt{ics\_down}}
\newcommand\icsup[0]{\texttt{ics\_up}}
\newcommand\id[0]{\texttt{id}}
\newcommand\idone[0]{\texttt{id1}}
\newcommand\ids[0]{\texttt{ids}}
\newcommand\idsp[0]{\texttt{ids'}}
\newcommand\idsone[0]{\texttt{ids1}}
\newcommand\idstwo[0]{\texttt{ids2}}
\newcommand\idthree[0]{\texttt{id3}}
\newcommand\idtwo[0]{\texttt{id2}}
\newcommand\indices[0]{\texttt{indices}}
\newcommand\sliceindices[0]{\texttt{slice\_indices}}
\newcommand\initializedfields[0]{\texttt{initialized\_fields}}
\newcommand\initialvalue[0]{\texttt{initial\_value}}
\newcommand\initialvaluep[0]{\texttt{initial\_value'}}
\newcommand\initialvaluetype[0]{\texttt{initial\_value\_type}}
\newcommand\irone[0]{\texttt{ir1}}
\newcommand\irtwo[0]{\texttt{ir2}}
\newcommand\iswhile[0]{\texttt{is\_while}}
\newcommand\keyword[0]{\texttt{keyword}}
\newcommand\ldi[0]{\texttt{ldi}}
\newcommand\ldip[0]{\texttt{ldi'}}
\newcommand\ldis[0]{\texttt{ldis}}
\newcommand\ldk[0]{\texttt{ldk}}
\newcommand\length[0]{\texttt{length}}
\newcommand\lengthp[0]{\texttt{length'}}
\newcommand\offset[0]{\texttt{offset}}
\newcommand\lenv[0]{\texttt{lenv}}
\newcommand\lenvtwo[0]{\texttt{lenv2}}
\newcommand\les[0]{\texttt{les}}
\newcommand\lesp[0]{\texttt{les'}}
\newcommand\lhs[0]{\texttt{lhs}}
\newcommand\lhsp[0]{\texttt{lhs'}}
\newcommand\liv[0]{\texttt{liv}}
\newcommand\marray[0]{\texttt{m\_array}}
\newcommand\matches[0]{\texttt{matches}}
\newcommand\matchesone[0]{\texttt{matches1}}
\newcommand\matchingrenamings[0]{\texttt{matching\_renamings}}
\newcommand\maxpos[0]{\texttt{max\_pos}}
\newcommand\mbv[0]{\texttt{m\_bv}}
\newcommand\mcond[0]{\texttt{m\_cond}}
\newcommand\mfactor[0]{\texttt{m\_factor}}
\newcommand\mindex[0]{\texttt{m\_index}}
\newcommand\minpos[0]{\texttt{min\_pos}}
\newcommand\mlength[0]{\texttt{m\_length}}
\newcommand\monoms[0]{\texttt{monoms}}
\newcommand\monomsone[0]{\texttt{monoms1}}
\newcommand\mpositions[0]{\texttt{m\_positions}}
\newcommand\msliceranges[0]{\texttt{m\_sliceranges}}
\newcommand\ms[0]{\texttt{ms}}
\newcommand\mstart[0]{\texttt{m\_start}}
\newcommand\mtop[0]{\texttt{m\_top}}
\newcommand\name[0]{\texttt{name}}
\newcommand\namedargs[0]{\texttt{named\_args}}
\newcommand\nameone[0]{\texttt{name1}}
\newcommand\nameopt[0]{\texttt{name\_opt}}
\newcommand\namep[0]{\texttt{name'}}
\newcommand\names[0]{\texttt{names}}
\newcommand\namesp[0]{\texttt{names'}}
\newcommand\namesubs[0]{\texttt{name\_s}}
\newcommand\namesubt[0]{\texttt{name\_t}}
\newcommand\nametwo[0]{\texttt{name2}}
\newcommand\newargs[0]{\texttt{new\_args}}
\newcommand\newbody[0]{\texttt{new\_body}}
\newcommand\newc[0]{\texttt{new\_c}}
\newcommand\newconstraints[0]{\texttt{new\_constraints}}
\newcommand\newcs[0]{\texttt{new\_cs}}
\newcommand\newd[0]{\texttt{new\_d}}
\newcommand\newdecls[0]{\texttt{new\_decls}}
\newcommand\newdenv[0]{\texttt{new\_denv}}
\newcommand\newe[0]{\texttt{new\_e}}
\newcommand\neweopt[0]{\texttt{new\_e\_opt}}
\newcommand\newenv[0]{\texttt{new\_env}}
\newcommand\newenvandfs[0]{\texttt{new\_env\_and\_fs}}
\newcommand\newf[0]{\texttt{new\_f}}
\newcommand\newfield[0]{\texttt{new\_field}}
\newcommand\newfields[0]{\texttt{new\_fields}}
\newcommand\newfuncdef[0]{\texttt{new\_func\_def}}
\newcommand\newfuncsig[0]{\texttt{new\_func\_sig}}
\newcommand\newg[0]{\texttt{new\_g}}
\newcommand\newgenv[0]{\texttt{new\_genv}}
\newcommand\newgsd[0]{\texttt{new\_gsd}}
\newcommand\newl[0]{\texttt{new\_l}}
\newcommand\newldi[0]{\texttt{new\_ldi}}
\newcommand\newle[0]{\texttt{new\_le}}
\newcommand\newli[0]{\texttt{new\_li}}
\newcommand\newleint[0]{\texttt{new\_le\_int}}
\newcommand\newleenum[0]{\texttt{new\_le\_enum}}
\newcommand\newlocalstoragetypes[0]{\texttt{new\_local\_storagetypes}}
\newcommand\newmbv[0]{\texttt{new\_m\_bv}}
\newcommand\newname[0]{\texttt{new\_name}}
\newcommand\newp[0]{\texttt{new\_p}}
\newcommand\newq[0]{\texttt{new\_q}}
\newcommand\newreturntype[0]{\texttt{new\_return\_type}}
\newcommand\news[0]{\texttt{new\_s}}
\newcommand\newsone[0]{\texttt{new\_s1}}
\newcommand\newstmt[0]{\texttt{new\_stmt}}
\newcommand\newstwo[0]{\texttt{new\_s2}}
\newcommand\newtenv[0]{\texttt{new\_tenv}}
\newcommand\newtenvp[0]{\texttt{new\_tenv'}}
\newcommand\newty[0]{\texttt{new\_ty}}
\newcommand\newtys[0]{\texttt{new\_tys}}
\newcommand\nlength[0]{\texttt{n\_length}}
\newcommand\nmonads[0]{\texttt{nmonads}}
\newcommand\num[0]{\textit{num}}
\newcommand\op[0]{\texttt{op}}
\newcommand\opp[0]{\texttt{op'}}
\newcommand\opfordir[0]{\texttt{op\_for\_dir}}
\newcommand\opone[0]{\texttt{op1}}
\newcommand\optwo[0]{\texttt{op2}}
\newcommand\orderedps[0]{\texttt{ordered\_ps}}
\newcommand\otherfuncsig[0]{\texttt{other\_func\_sig}}
\newcommand\othernames[0]{\texttt{other\_names}}
\newcommand\others[0]{\texttt{others}}
\newcommand\otherwise[0]{\texttt{otherwise}}
\newcommand\otherwiseopt[0]{\texttt{otherwise\_opt}}
\newcommand\otherwiseoptp[0]{\texttt{otherwise\_opt'}}
\newcommand\otherwisep[0]{\texttt{otherwise'}}
\newcommand\pairs[0]{\texttt{pairs}}
\newcommand\triples[0]{\texttt{triples}}
\newcommand\paramargs[0]{\texttt{param\_args}}
\newcommand\paramaritymatch[0]{\texttt{param\_arity\_match}}
\newcommand\paramdecls[0]{\texttt{param\_decls}}
\newcommand\params[0]{\texttt{params}}
\newcommand\newparams[0]{\texttt{new\_params}}
\newcommand\paramnames[0]{\texttt{param\_names}}
\newcommand\paramsp[0]{\texttt{params'}}
\newcommand\paramsone[0]{\texttt{params1}}
\newcommand\paramsonep[0]{\texttt{params1'}}
\newcommand\positions[0]{\texttt{positions}}
\newcommand\sliceranges[0]{\texttt{slice\_ranges}}
\newcommand\positionsone[0]{\texttt{positions1}}
\newcommand\positionstwo[0]{\texttt{positions2}}
\newcommand\positionsoneopt[0]{\texttt{positions1\_opt}}
\newcommand\positionstwoopt[0]{\texttt{positions2\_opt}}
\newcommand\range[0]{\texttt{range}}
\newcommand\rangeopt[0]{\texttt{range\_opt}}
\newcommand\ranges[0]{\texttt{ranges}}
\newcommand\rangesone[0]{\texttt{ranges1}}
\newcommand\rearray[0]{\texttt{re\_array}}
\newcommand\record[0]{\texttt{record}}
\newcommand\recordone[0]{\texttt{record1}}
\newcommand\renamingset[0]{\texttt{renaming\_set}}
\newcommand\rerecord[0]{\texttt{re\_record}}
\newcommand\retenv[0]{\texttt{ret\_env}}
\newcommand\retty[0]{\texttt{ret\_ty}}
\newcommand\rettyopt[0]{\texttt{ret\_ty\_opt}}
\newcommand\newrettyopt[0]{\texttt{new\_ret\_ty\_opt}}
\newcommand\rhs[0]{\texttt{rhs}}
\newcommand\rmarray[0]{\texttt{rm\_array}}
\newcommand\rmrecord[0]{\texttt{rm\_record}}
\newcommand\rvarray[0]{\texttt{rv\_array}}
\newcommand\rvrecord[0]{\texttt{rv\_record}}
\newcommand\pragmas[0]{\texttt{pragmas}}
\newcommand\sg[0]{\texttt{s\_g}}
\newcommand\size[0]{\texttt{size}}
\newcommand\sliceone[0]{\texttt{slice1}}
\newcommand\slices[0]{\texttt{slices}}
\newcommand\slicesone[0]{\texttt{slices1}}
\newcommand\slicesp[0]{\texttt{slices'}}
\newcommand\slicestwo[0]{\texttt{slices2}}
\newcommand\slicesannotated[0]{\texttt{slices\_annotated}}
\newcommand\slicetwo[0]{\texttt{slice2}}
\newcommand\sm[0]{\texttt{s\_m}}
\newcommand\smnew[0]{\texttt{s\_m\_new}}
\newcommand\src[0]{\texttt{src}}
\newcommand\stm[0]{\texttt{stm}}
\newcommand\structone[0]{\texttt{struct1}}
\newcommand\structtep[0]{\texttt{struct\_t\_e'}}
\newcommand\tleoneanon[0]{\texttt{t\_le1\_anon}}
\newcommand\structtwo[0]{\texttt{struct2}}
\newcommand\subfields[0]{\texttt{subfields}}
\newcommand\subpgmtype[0]{\texttt{subpgm\_type}}
\newcommand\subpgmtypeone[0]{\texttt{subpgm\_type1}}
\newcommand\subpgmtypetwo[0]{\texttt{subpgm\_type2}}
\newcommand\substs[0]{\texttt{substs}}
\newcommand\subtys[0]{\texttt{sub\_tys}}
\newcommand\tbase[0]{\texttt{t\_base}}
\newcommand\tbaseannot[0]{\texttt{t\_base\_annot}}
\newcommand\tbaseanon[0]{\texttt{t\_base\_anon}}
\newcommand\tbaseannotanon[0]{\texttt{t\_base\_annot\_anon}}
\newcommand\tcond[0]{\texttt{t\_cond}}
\newcommand\telem[0]{\texttt{t\_elem}}
\newcommand\tenv[0]{\texttt{tenv}}
\newcommand\denv[0]{\texttt{denv}}
\newcommand\env[0]{\texttt{env}}
\newcommand\tenvone[0]{\texttt{tenv1}}
\newcommand\tenvp[0]{\texttt{tenv'}}
\newcommand\tenvthree[0]{\texttt{tenv3}}
\newcommand\tenvtwo[0]{\texttt{tenv2}}
\newcommand\vtfield[0]{\texttt{t\_field}}
\newcommand\vtopabsolute[0]{\texttt{top\_absolute}}
\newcommand\tenvwithparams[0]{\texttt{tenv\_with\_params}}
\newcommand\tenvwithargs[0]{\texttt{tenv\_with\_args}}
\newcommand\tep[0]{\texttt{t\_e'}}
\newcommand\testruct[0]{\texttt{t\_e\_struct}}
\newcommand\tfalse[0]{\texttt{t\_false}}
\newcommand\tindexp[0]{\texttt{t\_index'}}
\newcommand\toffset[0]{\texttt{t\_offset}}
\newcommand\truep[0]{\texttt{true'}}
\newcommand\ts[0]{\textit{ts}}
\newcommand\tspecp[0]{\texttt{t\_spec'}}
\newcommand\tstwo[0]{\textit{ts2}}
\newcommand\tsy[0]{\texttt{sy}}
\newcommand\ttrue[0]{\texttt{t\_true}}
\newcommand\tty[0]{\texttt{ty}}
\newcommand\ttyanon[0]{\texttt{ty\_anon}}
\newcommand\tyactual[0]{\texttt{ty\_actual}}
\newcommand\tydecl[0]{\texttt{ty\_decl}}
\newcommand\tydeclp[0]{\texttt{ty\_decl'}}
\newcommand\tydeclopt[0]{\texttt{ty\_decl\_opt}}
\newcommand\ttyone[0]{\texttt{ty1}}
\newcommand\ttyp[0]{\texttt{ty'}}
\newcommand\ttytwo[0]{\texttt{ty2}}
\newcommand\tyopt[0]{\texttt{ty\_opt}}
\newcommand\tyoptp[0]{\texttt{ty\_opt'}}
\newcommand\typedargs[0]{\texttt{typed\_args}}
\newcommand\vtypedexpr[0]{\texttt{typed\_expr}}
\newcommand\typedexprs[0]{\texttt{typed\_exprs}}
\newcommand\typedexprsone[0]{\texttt{typed\_exprs1}}
\newcommand\typedspec[0]{\texttt{typed\_spec}}
\newcommand\tys[0]{\texttt{tys}}
\newcommand\tysp[0]{\texttt{tys'}}
\newcommand\va[0]{\texttt{a}}
\newcommand\vapprox[0]{\texttt{approx}}
\newcommand\vap[0]{\texttt{a'}}
\newcommand\vaccess[0]{\texttt{access}}
\newcommand\vaccessors[0]{\texttt{accessors}}
\newcommand\vaccessorpair[0]{\texttt{accessor\_pair}}
\newcommand\vabsname[0]{\texttt{absolute\_name}}
\newcommand\vabsslice[0]{\texttt{absolute\_slice}}
\newcommand\vabsslices[0]{\texttt{absolute\_slices}}
\newcommand\vsetteraccess[0]{\texttt{setter\_access}}
\newcommand\vslicesasindices[0]{\texttt{slices\_as\_indices}}
\newcommand\vabsoluteparent[0]{\texttt{absolute\_parent}}
\newcommand\vabsbitfields[0]{\texttt{abs\_bitfields}}
\newcommand\vabsbitfieldsone[0]{\texttt{abs\_bitfields1}}
\newcommand\vdeclx[0]{\texttt{decl\_x}}
\newcommand\vadecls[0]{\texttt{adecls}}
\newcommand\vbitvector[0]{\texttt{bitvector}}
\newcommand\vcases[0]{\texttt{cases}}
\newcommand\vcasealtlist[0]{\texttt{case\_alt\_list}}
\newcommand\vcasealtlistast[0]{\texttt{case\_alt\_list\_ast}}
\newcommand\votherwise[0]{\texttt{otherwise}}
\newcommand\voverride[0]{\texttt{override}}
\newcommand\voverridden[0]{\texttt{overridden}}
\newcommand\vanons[0]{\texttt{anon\_s}}
\newcommand\vanont[0]{\texttt{anon\_t}}
\newcommand\varg[0]{\texttt{arg}}
\newcommand\vargasts[0]{\texttt{arg\_asts}}
\newcommand\vargs[0]{\texttt{args}}
\newcommand\vargsone[0]{\texttt{args1}}
\newcommand\vargsonep[0]{\texttt{args1'}}
\newcommand\vargsp[0]{\texttt{args'}}
\newcommand\vargstwo[0]{\texttt{args2}}
\newcommand\varray[0]{\texttt{v\_array}}
\newcommand\vastnode[0]{\texttt{ast\_node}}
\newcommand\vasty[0]{\texttt{as\_ty}}
\newcommand\vastyopt[0]{\texttt{as\_ty\_opt}}
\newcommand\vsamescope[0]{\texttt{same\_scope}}
\newcommand\vallwayssucceeds[0]{\texttt{allways\_succeeds}}
\newcommand\vb[0]{\texttt{b}}
\newcommand\vbuiltin[0]{\texttt{builtin}}
\newcommand\vbits[0]{\texttt{bits}}
\newcommand\vbp[0]{\texttt{b'}}
\newcommand\vbase[0]{\texttt{base}}
\newcommand\vbasefields[0]{\texttt{base\_fields}}
\newcommand\vbasiclexpr[0]{\texttt{basic\_lexpr}}
\newcommand\vbequal[0]{\texttt{b\_equal}}
\newcommand\vbequallength[0]{\texttt{b\_equal\_length}}
\newcommand\vbf[0]{\texttt{bf}}
\newcommand\vbfindices[0]{\texttt{bf\_indices}}
\newcommand\vbfabsolute[0]{\texttt{bf\_absolute}}
\newcommand\vbfname[0]{\texttt{bf\_name}}
\newcommand\vbfone[0]{\texttt{bf1}}
\newcommand\vbftwo[0]{\texttt{bf2}}
\newcommand\vbinop[0]{\texttt{binop}}
\newcommand\vbitfieldasts[0]{\texttt{bitfield\_asts}}
\newcommand\vbitfields[0]{\texttt{bitfields}}
\newcommand\vbitfieldsone[0]{\texttt{bitfields1}}
\newcommand\vbody[0]{\texttt{body}}
\newcommand\vnewbody[0]{\texttt{new\_body}}
\newcommand\vbodyp[0]{\texttt{body'}}
\newcommand\vbone[0]{\texttt{b1}}
\newcommand\vbs[0]{\texttt{bs}}
\newcommand\vbthree[0]{\texttt{b3}}
\newcommand\vbtoolarge[0]{\texttt{b\_too\_large}}
\newcommand\vbtwo[0]{\texttt{b2}}
\newcommand\vbv[0]{\texttt{v\_bv}}
\newcommand\vc[0]{\texttt{c}}
\newcommand\vcp[0]{\texttt{c'}}
\newcommand\vcall[0]{\texttt{call}}
\newcommand\vcallp[0]{\texttt{call'}}
\newcommand\vcase[0]{\texttt{case}}
\newcommand\vcasesone[0]{\texttt{cases1}}
\newcommand\vcatcherlist[0]{\texttt{catcher\_list}}
\newcommand\vcond[0]{\texttt{v\_cond}}
\newcommand\vcondexpr[0]{\texttt{cond\_expr}}
\newcommand\vcone[0]{\texttt{c1}}
\newcommand\vcsasts[0]{\texttt{cs\_asts}}
\newcommand\vconstraintkind[0]{\texttt{constraint\_kind}}
\newcommand\vconstraintkindopt[0]{\texttt{constraint\_kind\_opt}}
\newcommand\vcopt[0]{\texttt{c\_opt}}
\newcommand\vcs[0]{\texttt{vcs}}
\newcommand\vcsnew[0]{\texttt{cs\_new}}
\newcommand\vcsone[0]{\texttt{cs1}}
\newcommand\vcstwo[0]{\texttt{cs2}}
\newcommand\vctwo[0]{\texttt{c2}}
\newcommand\vd[0]{\texttt{d}}
\newcommand\vdp[0]{\texttt{d'}}
\newcommand\vdebug[0]{\texttt{debug}}
\newcommand\vdecl[0]{\texttt{decl}}
\newcommand\vdeclp[0]{\texttt{decl'}}
\newcommand\vdeclitem[0]{\texttt{decl\_item}}
\newcommand\vdecls[0]{\texttt{decls}}
\newcommand\vdeclsone[0]{\texttt{decls1}}
\newcommand\vnewdecl[0]{\texttt{new\_decl}}
\newcommand\vnewdecls[0]{\texttt{new\_decls}}
\newcommand\vdiff[0]{\texttt{v\_diff}}
\newcommand\vdir[0]{\texttt{dir}}
\newcommand\vdirection[0]{\texttt{direction}}
\newcommand\vdiscarded[0]{\texttt{discarded}}
\newcommand\vdiscardedp[0]{\texttt{discarded'}}
\newcommand\vdone[0]{\texttt{d1}}
\newcommand\ve[0]{\texttt{e}}
\newcommand\vediscriminant[0]{\texttt{e\_discriminant}}
\newcommand\veminusone[0]{\texttt{e\_minus\_1}}
\newcommand\veupper[0]{\texttt{e\_upper}}
\newcommand\veinit[0]{\texttt{e\_init}}
\newcommand\vefive[0]{\texttt{e5}}
\newcommand\vefour[0]{\texttt{e4}}
\newcommand\vefourfive[0]{\texttt{e45}}
\newcommand\velimit[0]{\texttt{e\_limit}}
\newcommand\velimitopt[0]{\texttt{e\_limit\_opt}}
\newcommand\velimitoptp[0]{\texttt{e\_limit\_opt'}}
\newcommand\velse[0]{\texttt{else}}
\newcommand\velseexpr[0]{\texttt{else\_expr}}
\newcommand\vend[0]{\texttt{v\_end}}
\newcommand\vende[0]{\texttt{end\_e}}
\newcommand\vendep[0]{\texttt{end\_e'}}
\newcommand\vendstruct[0]{\texttt{end\_struct}}
\newcommand\vendt[0]{\texttt{end\_t}}
\newcommand\vendv[0]{\texttt{end\_v}}
\newcommand\vezero[0]{\texttt{e0}}
\newcommand\veone[0]{\texttt{e1}}
\newcommand\venum[0]{\texttt{enum}}
\newcommand\venumone[0]{\texttt{enum1}}
\newcommand\venumtwo[0]{\texttt{enum2}}
\newcommand\veoneopt[0]{\texttt{e1\_opt}}
\newcommand\vetwoopt[0]{\texttt{e2\_opt}}
\newcommand\vzoneopt[0]{\texttt{z1\_opt}}
\newcommand\vztwoopt[0]{\texttt{z2\_opt}}
\newcommand\veoneone[0]{\texttt{e1\_1}}
\newcommand\veonep[0]{\texttt{e1'}}
\newcommand\veonethree[0]{\texttt{e1\_3}}
\newcommand\veonetwo[0]{\texttt{e1\_2}}
\newcommand\veopt[0]{\texttt{e\_opt}}
\newcommand\veoptp[0]{\texttt{e\_opt'}}
\newcommand\vep[0]{\texttt{e'}}
\newcommand\vepp[0]{\texttt{e''}}
\newcommand\vepattern[0]{\texttt{e\_pattern}}
\newcommand\ves[0]{\texttt{e\_s}}
\newcommand\vesp[0]{\texttt{es'}}
\newcommand\vet[0]{\texttt{e\_t}}
\newcommand\vethree[0]{\texttt{e3}}
\newcommand\vethreep[0]{\texttt{e3'}}
\newcommand\vetwo[0]{\texttt{e2}}
\newcommand\vetwoone[0]{\texttt{e2\_1}}
\newcommand\vetwop[0]{\texttt{e2'}}
\newcommand\vetwothree[0]{\texttt{e2\_3}}
\newcommand\vetwotwo[0]{\texttt{e2\_2}}
\newcommand\vewhere[0]{\texttt{e\_where}}
\newcommand\vex[0]{\texttt{ex}}
\newcommand\vexpr[0]{\texttt{expr}}
\newcommand\vexprasts[0]{\texttt{expr\_asts}}
\newcommand\vexprpattern[0]{\texttt{expr\_pattern}}
\newcommand\vexprs[0]{\texttt{exprs}}
\newcommand\vf[0]{\texttt{f}}
\newcommand\vfl[0]{\texttt{fl}}
\newcommand\vnewf[0]{\texttt{f\_new}}
\newcommand\vfs[0]{\texttt{fs}}
\newcommand\vfactor[0]{\texttt{v\_factor}}
\newcommand\vfieldwidth[0]{\texttt{field\_width}}
\newcommand\vfieldassignasts[0]{\texttt{field\_assign\_asts}}
\newcommand\vfieldassigns[0]{\texttt{field\_assigns}}
\newcommand\vfieldasts[0]{\texttt{field\_asts}}
\newcommand\vfieldone[0]{\texttt{field1}}
\newcommand\vfields[0]{\texttt{fields}}
\newcommand\vlefields[0]{\texttt{le\_fields}}
\newcommand\vlefieldsone[0]{\texttt{le\_fields1}}
\newcommand\vlerecord[0]{\texttt{le\_record}}
\newcommand\vfieldsone[0]{\texttt{fields1}}
\newcommand\vfieldss[0]{\texttt{fields\_s}}
\newcommand\vfieldst[0]{\texttt{fields\_t}}
\newcommand\vfieldstwo[0]{\texttt{fields2}}
\newcommand\vfieldtwo[0]{\texttt{field2}}
\newcommand\vfone[0]{\texttt{f1}}
\newcommand\vfp[0]{\texttt{f'}}
\newcommand\vfromexpr[0]{\texttt{from\_expr}}
\newcommand\vftwo[0]{\texttt{f2}}
\newcommand\vfuncone[0]{\texttt{func1}}
\newcommand\vfunctwo[0]{\texttt{func2}}
\newcommand\vfuncs[0]{\texttt{funcs}}
\newcommand\vfuncsone[0]{\texttt{funcs1}}
\newcommand\vfuncargs[0]{\texttt{func\_args}}
\newcommand\vfuncbody[0]{\texttt{func\_body}}
\newcommand\vg[0]{\texttt{g}}
\newcommand\vgfour[0]{\texttt{g4}}
\newcommand\vgfive[0]{\texttt{g5}}
\newcommand\vglobalkeywordnonconfig[0]{\texttt{global\_keyword\_non\_config}}
\newcommand\vh[0]{\texttt{h}}
\newcommand\vhp[0]{\texttt{h'}}
\newcommand\vlhsaccess[0]{\texttt{lhs\_access}}
\newcommand\vlhsaccessopt[0]{\texttt{lhs\_access\_opt}}
\newcommand\vlhsaccessopts[0]{\texttt{lhs\_access\_opts}}
\newcommand\vlocal[0]{\texttt{local}}
\newcommand\vlooplimit[0]{\texttt{looplimit}}
\newcommand\vglobal[0]{\texttt{global}}
\newcommand\vgone[0]{\texttt{g1}}
\newcommand\vgp[0]{\texttt{g'}}
\newcommand\vgthree[0]{\texttt{g3}}
\newcommand\vgtwo[0]{\texttt{g2}}
\newcommand\vgzero[0]{\texttt{g0}}
\newcommand\vi[0]{\texttt{i}}
\newcommand\vip[0]{\texttt{i'}}
\newcommand\annotatedcs[0]{\texttt{annotated\_cs}}
\newcommand\refinedcs[0]{\texttt{refined\_cs}}
\newcommand\vics[0]{\texttt{ics}}
\newcommand\vidasts[0]{\texttt{id\_asts}}
\newcommand\vids[0]{\texttt{ids}}
\newcommand\vidsone[0]{\texttt{ids1}}
\newcommand\vimpdefs[0]{\texttt{impdefs}}
\newcommand\vimpdefsp[0]{\texttt{impdefs'}}
\newcommand\vimpls[0]{\texttt{impls}}
\newcommand\vindex[0]{\texttt{index}}
\newcommand\vindexname[0]{\texttt{index\_name}}
\newcommand\vinitialvalue[0]{\texttt{initial\_value}}
\newcommand\vinnerenv[0]{\texttt{inner\_env}}
\newcommand\vione[0]{\texttt{i1}}
\newcommand\vis[0]{\texttt{vis}}
\newcommand\visone[0]{\texttt{is1}}
\newcommand\visreadonly[0]{\texttt{is\_readonly}}
\newcommand\vistwo[0]{\texttt{is2}}
\newcommand\vitwo[0]{\texttt{i2}}
\newcommand\vj[0]{\texttt{j}}
\newcommand\vkeyword[0]{\texttt{keyword}}
\newcommand\vl[0]{\texttt{l}}
\newcommand\vlabel[0]{\texttt{label}}
\newcommand\vlabels[0]{\texttt{labels}}
\newcommand\vlabelsone[0]{\texttt{labels1}}
\newcommand\vlastindex[0]{\texttt{last\_index}}
\newcommand\vle[0]{\texttt{le}}
\newcommand\vlep[0]{\texttt{le'}}
\newcommand\vlelist[0]{\texttt{le\_list}}
\newcommand\vlelistone[0]{\texttt{le\_list1}}
\newcommand\vlength[0]{\texttt{v\_length}}
\newcommand\vlengthexprs[0]{\texttt{length\_expr\_s}}
\newcommand\vlengthexprt[0]{\texttt{length\_expr\_t}}
\newcommand\vlengths[0]{\texttt{length\_s}}
\newcommand\vlengtht[0]{\texttt{length\_t}}
\newcommand\vleone[0]{\texttt{le1}}
\newcommand\vleonep[0]{\texttt{le1'}}
\newcommand\vles[0]{\texttt{les}}
\newcommand\vlesp[0]{\texttt{les'}}
\newcommand\vlethree[0]{\texttt{le3}}
\newcommand\vletwo[0]{\texttt{le2}}
\newcommand\vlexpr[0]{\texttt{lexpr}}
\newcommand\vlexprasts[0]{\texttt{lexpr\_asts}}
\newcommand\vlexprs[0]{\texttt{lexprs}}
\newcommand\vli[0]{\texttt{li}}
\newcommand\vlimit[0]{\texttt{limit}}
\newcommand\vlimitopt[0]{\texttt{limit\_opt}}
\newcommand\vlimitoptp[0]{\texttt{limit\_opt'}}
\newcommand\vlimitoptone[0]{\texttt{limit\_opt1}}
\newcommand\vlimitopttwo[0]{\texttt{limit\_opt2}}
\newcommand\vlimitexpr[0]{\texttt{limit\_expr}}
\newcommand\vlimitone[0]{\texttt{limit1}}
\newcommand\vlimitp[0]{\texttt{limit'}}
\newcommand\vlimittwo[0]{\texttt{limit2}}
\newcommand\vlione[0]{\texttt{li1}}
\newcommand\vlip[0]{\texttt{li'}}
\newcommand\vlis[0]{\texttt{lis\_s}}
\newcommand\vlist[0]{\texttt{v\_list}}
\newcommand\vlit[0]{\texttt{lis\_t}}
\newcommand\vlitwo[0]{\texttt{li2}}
\newcommand\vlocaldeclkeyword[0]{\texttt{local\_decl\_keyword}}
\newcommand\vlone[0]{\texttt{l1}}
\newcommand\vltwo[0]{\texttt{l2}}
\newcommand\vlthree[0]{\texttt{l3}}
\newcommand\vimmutable[0]{\texttt{immutable}}
\newcommand\vm[0]{\texttt{m}}
\newcommand\vmustbepure[0]{\texttt{must\_be\_pure}}
\newcommand\extpairs[0]{\texttt{extpairs}}
\newcommand\extpairsab[0]{\texttt{extpairs\_a\_b}}
\newcommand\extpairscd[0]{\texttt{extpairs\_c\_d}}
\newcommand\vmac[0]{\texttt{mac}}
\newcommand\vmad[0]{\texttt{mad}}
\newcommand\vmatching[0]{\texttt{matching}}
\newcommand\vmlist[0]{\texttt{vm\_list}}
\newcommand\vmlistone[0]{\texttt{vm\_list1}}
\newcommand\vmone[0]{\texttt{m1}}
\newcommand\vmono[0]{\texttt{mono}}
\newcommand\vms[0]{\texttt{vms}}
\newcommand\vmsone[0]{\texttt{vms1}}
\newcommand\vmstwo[0]{\texttt{vms2}}
\newcommand\vmtwo[0]{\texttt{m2}}
\newcommand\vn[0]{\texttt{n}}
\newcommand\vnames[0]{\texttt{name\_s}}
\newcommand\vnamess[0]{\texttt{names\_s}}
\newcommand\vnamest[0]{\texttt{names\_t}}
\newcommand\vnamet[0]{\texttt{name\_t}}
\newcommand\vneg[0]{\texttt{neg}}
\newcommand\vnested[0]{\texttt{nested}}
\newcommand\vnextlimitopt[0]{\texttt{next\_limit\_opt}}
\newcommand\vnew[0]{\texttt{new}}
\newcommand\vnonmatching[0]{\texttt{nonmatching}}
\newcommand\vnormal[0]{\texttt{normal}}
\newcommand\vnewargs[0]{\texttt{new\_args}}
\newcommand\voptlimit[0]{\texttt{opt\_limit}}
\newcommand\votherwiseopt[0]{\texttt{otherwise\_opt}}
\newcommand\vouterenv[0]{\texttt{outer\_env}}
\newcommand\vp[0]{\texttt{p}}
\newcommand\vpurity[0]{\texttt{purity}}
\newcommand\paramtype[0]{\texttt{param\_type}}
\newcommand\vprev[0]{\texttt{prev}}
\newcommand\vparameters[0]{\texttt{parameters}}
\newcommand\vparams[0]{\texttt{params}}
\newcommand\vparamsopt[0]{\texttt{params\_opt}}
\newcommand\vparsednode[0]{\texttt{parsed\_node}}
\newcommand\vpat[0]{\texttt{pat}}
\newcommand\vpatp[0]{\texttt{pat'}}
\newcommand\vpattern[0]{\texttt{pattern}}
\newcommand\vpatternasts[0]{\texttt{pattern\_asts}}
\newcommand\vpatternlist[0]{\texttt{pattern\_list}}
\newcommand\vpatterns[0]{\texttt{patterns}}
\newcommand\vpatternset[0]{\texttt{pattern\_set}}
\newcommand\vpone[0]{\texttt{p1}}
\newcommand\vpp[0]{\texttt{p'}}
\newcommand\vps[0]{\texttt{ps}}
\newcommand\vptwo[0]{\texttt{p2}}
\newcommand\vpthree[0]{\texttt{p3}}
\newcommand\vpzero[0]{\texttt{p0}}
\newcommand\vq[0]{\texttt{q}}
\newcommand\vqone[0]{\texttt{q1}}
\newcommand\vqtwo[0]{\texttt{q2}}
\newcommand\vqualifier[0]{\texttt{qualifier}}
\newcommand\vr[0]{\texttt{r}}
\newcommand\vrenameddiscarded[0]{\texttt{renamed\_discarded}}
\newcommand\vrmrecordnew[0]{\texttt{rm\_record\_new}}
\newcommand\vre[0]{\texttt{re}}
\newcommand\vrecord[0]{\texttt{v\_record}}
\newcommand\vrecurselimit[0]{\texttt{recurse\_limit}}
\newcommand\vreone[0]{\texttt{re1}}
\newcommand\vres[0]{\texttt{res}}
\newcommand\vresult[0]{\texttt{result}}
\newcommand\vret[0]{\texttt{ret}}
\newcommand\vreturntype[0]{\texttt{return\_type}}
\newcommand\vs[0]{\texttt{s}}
\newcommand\vsetterarg[0]{\texttt{setter\_arg}}
\newcommand\vsetterargs[0]{\texttt{setter\_args}}
\newcommand\vscopeone[0]{\texttt{scope1}}
\newcommand\vscopetwo[0]{\texttt{scope2}}
\newcommand\vscond[0]{\texttt{s\_cond}}
\newcommand\vselse[0]{\texttt{s\_else}}
\newcommand\vsesinitialvalue[0]{\texttt{ses\_initial\_value}}
\newcommand\vsesbase[0]{\texttt{ses\_base}}
\newcommand\vsesf[0]{\texttt{ses\_f}}
\newcommand\vsesfone[0]{\texttt{ses\_f1}}
\newcommand\vsescatchers[0]{\texttt{ses\_catchers}}
\newcommand\vseswithparams[0]{\texttt{ses\_with\_params}}
\newcommand\vseswithargs[0]{\texttt{ses\_with\_args}}
\newcommand\vseswithreturn[0]{\texttt{ses\_with\_return}}
\newcommand\vses[0]{\texttt{ses}}
\newcommand\vsesbody[0]{\texttt{ses\_body}}
\newcommand\vsesfuncsig[0]{\texttt{ses\_func\_sig}}
\newcommand\vsesin[0]{\texttt{ses\_in}}
\newcommand\vsesotherwise[0]{\texttt{ses\_otherwise}}
\newcommand\vsesldi[0]{\texttt{ses\_ldi}}
\newcommand\vsesblock[0]{\texttt{ses\_block}}
\newcommand\vsesstart[0]{\texttt{ses\_start}}
\newcommand\vsesend[0]{\texttt{ses\_end}}
\newcommand\vseslimit[0]{\texttt{ses\_limit}}
\newcommand\vsesrecurselimit[0]{\texttt{ses\_recurse\_limit}}
\newcommand\vsespat[0]{\texttt{ses\_pat}}
\newcommand\vsesnew[0]{\texttt{ses\_new}}
\newcommand\vsess[0]{\texttt{sess}}
\newcommand\vsesp[0]{\texttt{ses'}}
\newcommand\vsese[0]{\texttt{ses\_e}}
\newcommand\vsest[0]{\texttt{ses\_t}}
\newcommand\vsesactual[0]{\texttt{ses\_actual}}
\newcommand\vsescall[0]{\texttt{ses\_call}}
\newcommand\vsescond[0]{\texttt{ses\_cond}}
\newcommand\vsestrue[0]{\texttt{ses\_true}}
\newcommand\vsesfalse[0]{\texttt{ses\_false}}
\newcommand\vsesone[0]{\texttt{ses1}}
\newcommand\vsestwo[0]{\texttt{ses2}}
\newcommand\sesslices[0]{\texttt{ses\_slices}}
\newcommand\vsesthree[0]{\texttt{ses3}}
\newcommand\vsesoffset[0]{\texttt{ses\_offset}}
\newcommand\vseslength[0]{\texttt{ses\_length}}
\newcommand\vsesindex[0]{\texttt{ses\_index}}
\newcommand\vsesslices[0]{\texttt{ses\_slices}}
\newcommand\vsesargs[0]{\texttt{ses\_args}}
\newcommand\vsessargs[0]{\texttt{sess\_args}}
\newcommand\vsesbitfields[0]{\texttt{ses\_bitfields}}
\newcommand\vsesle[0]{\texttt{ses\_le}}
\newcommand\vsesre[0]{\texttt{ses\_re}}
\newcommand\vsesty[0]{\texttt{ses\_ty}}
\newcommand\newses[0]{\texttt{new\_ses}}
\newcommand\vslice[0]{\texttt{slice}}
\newcommand\vnewslice[0]{\texttt{new\_slice}}
\newcommand\vsliceasts[0]{\texttt{slice\_asts}}
\newcommand\vsliceone[0]{\texttt{slice1}}
\newcommand\vslicetwo[0]{\texttt{slice2}}
\newcommand\vslices[0]{\texttt{slices}}
\newcommand\vslicesp[0]{\texttt{slices'}}
\newcommand\vslicesone[0]{\texttt{slices1}}
\newcommand\vsone[0]{\texttt{s1}}
\newcommand\vsonep[0]{\texttt{s1'}}
\newcommand\vsonelone[0]{\texttt{s1l1}}
\newcommand\vsonelonestwo[0]{\texttt{s1l1s2}}
\newcommand\vstwoltwo[0]{\texttt{s2l2}}
\newcommand\vstwoltwosone[0]{\texttt{s2l2s1}}
\newcommand\vsp[0]{\texttt{s'}}
\newcommand\vspec[0]{\texttt{spec}}
\newcommand\vspectext[0]{\texttt{spec\_text}}
\newcommand\vstdtext[0]{\texttt{std\_text}}
\newcommand\vspectokens[0]{\texttt{spec\_tokens}}
\newcommand\vstdtokens[0]{\texttt{std\_tokens}}
\newcommand\vspecparse[0]{\texttt{spec\_parse}}
\newcommand\vstdparse[0]{\texttt{std\_parse}}
\newcommand\vspecast[0]{\texttt{spec\_ast}}
\newcommand\vstdast[0]{\texttt{std\_ast}}
\newcommand\vstdastrenamed[0]{\texttt{std\_ast\_renamed}}
\newcommand\vstddeclast[0]{\texttt{std\_decl\_ast}}
\newcommand\vstdasbuiltin[0]{\texttt{std\_as\_builtin}}
\newcommand\vspp[0]{\texttt{s''}}
\newcommand\vsstruct[0]{\texttt{s\_struct}}
\newcommand\vstart[0]{\texttt{v\_start}}
\newcommand\vstarte[0]{\texttt{start\_e}}
\newcommand\vstartep[0]{\texttt{start\_e'}}
\newcommand\vstartstruct[0]{\texttt{start\_struct}}
\newcommand\vstartt[0]{\texttt{start\_t}}
\newcommand\vstartv[0]{\texttt{start\_v}}
\newcommand\vstep[0]{\texttt{v\_step}}
\newcommand\vstmt[0]{\texttt{stmt}}
\newcommand\vstmtlist[0]{\texttt{stmt\_list}}
\newcommand\vstmts[0]{\texttt{stmts}}
\newcommand\vstmtsone[0]{\texttt{stmts1}}
\newcommand\vstwo[0]{\texttt{s2}}
\newcommand\vstwop[0]{\texttt{s2'}}
\newcommand\vsthree[0]{\texttt{s3}}
\newcommand\vsinterval[0]{\texttt{s\_interval}}
\newcommand\vsbottomtop[0]{\texttt{s\_bottom\_top}}
\newcommand\vsubtype[0]{\texttt{subtype}}
\newcommand\vsubtypeopt[0]{\texttt{subtype\_opt}}
\newcommand\vsuper[0]{\texttt{super}}
\newcommand\vsym[0]{\texttt{sym}}
\newcommand\vsymast[0]{\texttt{sym\_ast}}
\newcommand\vsymasts[0]{\texttt{sym\_asts}}
\newcommand\vsymastsone[0]{\texttt{sym\_asts1}}
\newcommand\vsyms[0]{\texttt{syms}}
\newcommand\vsymsone[0]{\texttt{syms1}}
\newcommand\vt[0]{\texttt{t}}
\newcommand\vtail[0]{\texttt{tail}}
\newcommand\vte[0]{\texttt{t\_e}}
\newcommand\vtefour[0]{\texttt{t\_e4}}
\newcommand\vteone[0]{\texttt{t\_e1}}
\newcommand\vteonestruct[0]{\texttt{t\_e1\_struct}}
\newcommand\vtep[0]{\texttt{t\_e'}}
\newcommand\vtestruct[0]{\texttt{t\_e\_struct}}
\newcommand\vtetwo[0]{\texttt{t\_e2}}
\newcommand\vtefield[0]{\texttt{t\_field}}
\newcommand\vtetwostruct[0]{\texttt{t\_e2\_struct}}
\newcommand\vthenexpr[0]{\texttt{then\_expr}}
\newcommand\vtlhs[0]{\texttt{t\_lhs}}
\newcommand\vtleone[0]{\texttt{t\_le1}}
\newcommand\vtoexpr[0]{\texttt{to\_expr}}
\newcommand\vtone[0]{\texttt{t1}}
\newcommand\vtonewellconstrained[0]{\texttt{t1\_well\_constrained}}
\newcommand\vttwowellconstrained[0]{\texttt{t2\_well\_constrained}}
\newcommand\vtoneanon[0]{\texttt{t1\_anon}}
\newcommand\vtp[0]{\texttt{t'}}
\newcommand\vtpp[0]{\texttt{t''}}
\newcommand\vtre[0]{\texttt{t\_re}}
\newcommand\vts[0]{\texttt{t\_s}}
\newcommand\vtsone[0]{\texttt{ts1}}
\newcommand\vtstruct[0]{\texttt{t\_struct}}
\newcommand\vtstwo[0]{\texttt{ts2}}
\newcommand\vtsupers[0]{\texttt{t\_supers}}
\newcommand\vttwo[0]{\texttt{t2}}
\newcommand\vttwoanon[0]{\texttt{t2\_anon}}
\newcommand\vtypedast[0]{\texttt{typed\_ast}}
\newcommand\vtypeasts[0]{\texttt{type\_asts}}
\newcommand\vtypes[0]{\texttt{types}}
\newcommand\vtys[0]{\texttt{ty\_s}}
\newcommand\vtystruct[0]{\texttt{ty\_struct}}
\newcommand\vtyt[0]{\texttt{ty\_t}}
\newcommand\vu[0]{\texttt{u}}
\newcommand\vunop[0]{\texttt{unop}}
\newcommand\vuntypedast[0]{\texttt{untyped\_ast}}
\newcommand\vv[0]{\texttt{v}}
\newcommand\recordslices[0]{\texttt{record\_slices}}
\newcommand\vvalue[0]{\texttt{value}}
\newcommand\vvaluep[0]{\texttt{value'}}
\newcommand\vvargs[0]{\texttt{vargs}}
\newcommand\vvec[0]{\texttt{vec}}
\newcommand\vvfields[0]{\texttt{v\_fields}}
\newcommand\vvopt[0]{\texttt{v\_opt}}
\newcommand\vvoptnew[0]{\texttt{v\_opt\_new}}
\newcommand\vvopts[0]{\texttt{v\_opts}}
\newcommand\vvoptsp[0]{\texttt{v\_opts'}}
\newcommand\vvoptp[0]{\texttt{v\_opt'}}
\newcommand\vvone[0]{\texttt{v1}}
\newcommand\vvp[0]{\texttt{v'}}
\newcommand\vvparams[0]{\texttt{vparams}}
\newcommand\vvs[0]{\texttt{vs}}
\newcommand\newv[0]{\texttt{new\_v}}
\newcommand\newvs[0]{\texttt{new\_vs}}
\newcommand\vvsp[0]{\texttt{vs'}}
\newcommand\vvsone[0]{\texttt{vs1}}
\newcommand\vvsubtop[0]{\texttt{v\_top}}
\newcommand\vvtwo[0]{\texttt{v2}}
\newcommand\vvtuple[0]{\texttt{v\_tuple}}
\newcommand\vvty[0]{\texttt{v\_ty}}
\newcommand\vw[0]{\texttt{w}}
\newcommand\vwhere[0]{\texttt{where}}
\newcommand\vwhereast[0]{\texttt{where\_ast}}
\newcommand\vwhereopt[0]{\texttt{where\_opt}}
\newcommand\vwidth[0]{\texttt{width}}
\newcommand\vwone[0]{\texttt{w1}}
\newcommand\vwp[0]{\texttt{w'}}
\newcommand\vwtwo[0]{\texttt{w2}}
\newcommand\vx[0]{\texttt{x}}
\newcommand\vxp[0]{\texttt{x'}}
\newcommand\vxs[0]{\texttt{xs}}
\newcommand\vy[0]{\texttt{y}}
\newcommand\vz[0]{\texttt{z}}
\newcommand\vzminlist[0]{\texttt{z\_min\_list}}
\newcommand\vzmin[0]{\texttt{z\_min}}
\newcommand\vza[0]{\texttt{za}}
\newcommand\vzaopt[0]{\texttt{za\_opt}}
\newcommand\vzb[0]{\texttt{zb}}
\newcommand\vzbopt[0]{\texttt{zb\_opt}}
\newcommand\vzone[0]{\texttt{z1}}
\newcommand\vzopt[0]{\texttt{z\_opt}}
\newcommand\vzoptone[0]{\texttt{z\_opt1}}
\newcommand\vzopttwo[0]{\texttt{z\_opt2}}
\newcommand\vzs[0]{\texttt{zs}}
\newcommand\vzthree[0]{\texttt{z3}}
\newcommand\vzfour[0]{\texttt{z4}}
\newcommand\vzfive[0]{\texttt{z5}}
\newcommand\vzsix[0]{\texttt{z6}}
\newcommand\vztwo[0]{\texttt{z2}}
\newcommand\wantedtindex[0]{\texttt{wanted\_t\_index}}
\newcommand\wid[0]{\texttt{wid}}
\newcommand\width[0]{\texttt{width}}
\newcommand\widthp[0]{\texttt{width'}}
\newcommand\widths[0]{\texttt{width\_s}}
\newcommand\widtht[0]{\texttt{width\_t}}
\newcommand\ws[0]{\texttt{w\_s}}
\newcommand\wt[0]{\texttt{w\_t}}
\newcommand\xs[0]{\texttt{xs}}
\newcommand\vEs[0]{\texttt{es}}
\newcommand\vesone[0]{\texttt{es1}}
\newcommand\veslice[0]{\texttt{e\_slice}}
\newcommand\vleslice[0]{\texttt{le\_slice}}
\newcommand\veslicewidth[0]{\texttt{e\_slice\_width}}
\newcommand\valueranges[0]{\texttt{value\_ranges}}
\newcommand\vrange[0]{\texttt{range}}
\newcommand\twidth[0]{\texttt{t\_width}}
\newcommand\seswidth[0]{\texttt{ses\_width}}
\newcommand\tfields[0]{\texttt{tfields}}
\newcommand\tanonbase[0]{\texttt{t\_anon\_base}}
\newcommand\rhsty[0]{\texttt{rhs\_ty}}
\newcommand\exnname[0]{\texttt{exn\_name}}
\newcommand\newcatcher[0]{\texttt{new\_catcher}}
\newcommand\typedinitialvalue[0]{\texttt{typed\_initial\_value}}
\newcommand\typede[0]{\texttt{typed\_e}}
\newcommand\vfieldnames[0]{\texttt{field\_names}}
\newcommand\vexpectedconstraintlength[0]{\texttt{expected\_constraint\_length}}
\newcommand\vconsolestream[0]{\texttt{consolestream}}
\newcommand\vlhstys[0]{\texttt{lhs\_tys}}
\newcommand\vlhstysp[0]{\texttt{lhs\_tys'}}
\newcommand\vrhstys[0]{\texttt{rhs\_tys}}
\newcommand\vargdecls[0]{\texttt{arg\_decls}}
\newcommand\vmain[0]{\texttt{main}}
\newcommand\vread[0]{\texttt{read}}
\newcommand\vmodify[0]{\texttt{modify}}
\newcommand\vsetter[0]{\texttt{setter}}
\newcommand\vgetter[0]{\texttt{getter}}
\newcommand\vcode[0]{\texttt{code}}
\newcommand\vAbf[0]{\textbf{A}}
\newcommand\vBbf[0]{\textbf{B}}
\newcommand\vBIG[0]{\textbf{BIG}}
\newcommand\vLITTLE[0]{\textbf{LITTLE}}
\newcommand\vHALFWORDSIZE[0]{\texttt{HALF\_WORD\_SIZE}}
\newcommand\vWORDSIZE[0]{\texttt{WORD\_SIZE}}
\newcommand\PI[0]{\texttt{PI}}
\newcommand\RED[0]{\texttt{RED}}
\newcommand\GREEN[0]{\texttt{GREEN}}
\newcommand\BLUE[0]{\texttt{BLUE}}
\newcommand\RecordBase[0]{\texttt{RecordBase}}
\newcommand\HalfWordBits[0]{\texttt{HALF\_WORD\_BITS}}
\newcommand\MyRecord[0]{\texttt{MyRecord}}
\newcommand\Color[0]{\texttt{Color}}
\newcommand\rotatecolor[0]{\texttt{rotate\_color}}
\newcommand\rotatezero[0]{\texttt{rotate0}}
\newcommand\rotateone[0]{\texttt{rotate1}}
\newcommand\foo[0]{\texttt{foo}}
\newcommand\flipbits[0]{\texttt{flip\_bits}}
\newcommand\addten[0]{\texttt{add\_10}}
\newcommand\addtenone[0]{\texttt{add\_10-1}}
\newcommand\factorial[0]{\texttt{factorial}}
\newcommand\declaredparam[0]{\texttt{declared\_param}}
\newcommand\caninsertstdlibparam[0]{\texttt{can\_insert\_stdlib\_param}}
\newcommand\inferredparameters[0]{\texttt{inferred\_parameters}}
\newcommand\declaredparameters[0]{\texttt{declared\_parameters}}
\newcommand\allparameters[0]{\texttt{all\_parameters}}
\newcommand\uniqueparameters[0]{\texttt{unique\_parameters}}
\newcommand\isstatic[0]{\texttt{is\_static}}
\newcommand\vLone[0]{\texttt{L1}}
\newcommand\vLtwo[0]{\texttt{L2}}
\newcommand\symdomsone[0]{\texttt{symdoms1}}
\newcommand\symdomstwo[0]{\texttt{symdoms1}}
\newcommand\symdomsonenorm[0]{\texttt{symdoms1\_norm}}
\newcommand\symdomstwonorm[0]{\texttt{symdoms2\_norm}}
\newcommand\symdoms[0]{\texttt{symdoms}}
\newcommand\newsymdoms[0]{\texttt{new\_symdoms}}
\newcommand\vfinitedomains[0]{\texttt{finite\_domains}}
\newcommand\cdone[0]{\texttt{cd1}}
\newcommand\cdtwo[0]{\texttt{cd2}}
\newcommand\sdone[0]{\texttt{sd1}}
\newcommand\sdtwo[0]{\texttt{sd2}}
\newcommand\plf[0]{\texttt{plf}}
\newcommand\vsapprox[0]{\texttt{s\_approx}}
\newcommand\intervals[0]{\texttt{intervals}}
\newcommand\increment[0]{\texttt{increment}}
\newcommand\incrementone[0]{\texttt{increment-1}}
\newcommand\BitsArray[0]{\texttt{BitsArray}}
\newcommand\undeclaredidentifier[0]{\texttt{undeclared\_identifier}}
\newcommand\fldone[0]{\texttt{fld1}}
\newcommand\fldtwo[0]{\texttt{fld2}}
\newcommand\newbits[0]{\texttt{newbits}}
\newcommand\vN[0]{\texttt{N}}
\newcommand\vA[0]{\texttt{A}}
\newcommand\newprec[0]{\texttt{new\_prec}}
\newcommand\absconfigs[0]{\texttt{abs\_configs}}
\newcommand\configsone[0]{\texttt{configs1}}
\newcommand\configstwo[0]{\texttt{configs2}}
\newcommand\bodyconfigs[0]{\texttt{body\_configs}}
\newcommand\otherwiseconfigs[0]{\texttt{otherwise\_configs}}
\newcommand\otherwises[0]{\texttt{otherwise\_s}}
\newcommand\catchconfigs[0]{\texttt{catch\_configs}}
\newcommand\paramname[0]{\texttt{param\_name}}
\newcommand\elemty[0]{\texttt{elem\_ty}}
\newcommand\newindex[0]{\texttt{new\_index}}
\newcommand\newelength[0]{\texttt{new\_e\_length}}
\newcommand\pl[0]{\texttt{pl}}
\newcommand\newpl[0]{\texttt{newpl}}
\newcommand\vpe[0]{\texttt{p\_e}}
\newcommand\vnewpe[0]{\texttt{new\_p\_e}}
\newcommand\optexnname[0]{\texttt{opt\_exn\_name}}
\newcommand\optexnnamep[0]{\texttt{opt\_exn\_name'}}
\newcommand\whenstmt[0]{\texttt{when\_stmt}}
\newcommand\whenstmtp[0]{\texttt{when\_stmt'}}
\newcommand\exnty[0]{\texttt{exn\_ty}}
\newcommand\exntyp[0]{\texttt{exn\_ty'}}
\newcommand\ColorType[0]{\texttt{Color}}
\newcommand\RedLabel[0]{\texttt{RED}}
\newcommand\GreenLabel[0]{\texttt{GREEN}}
\newcommand\BlueLabel[0]{\texttt{BLUE}}
\newcommand\vpendingcalls[0]{\texttt{pending\_calls}}

