%%%%%%%%%%%%%%%%%%%%%%%%%%%%%%%%%%%%%%%%%%%%%%%%%%%%%%%%%%%%%%%%%%%%%%%%%%%%%%%
\chapter{Semantics Utility Rules\label{chap:SemanticsUtilityRules}}
%%%%%%%%%%%%%%%%%%%%%%%%%%%%%%%%%%%%%%%%%%%%%%%%%%%%%%%%%%%%%%%%%%%%%%%%%%%%%%%

This chapter defines the following helper relations for operating on \nativevaluesterm{},
\hyperlink{def-envs}{environments}, and operations involving values and types:
\begin{itemize}
  \item \SemanticsRuleRef{GetPendingCalls} retrieves the value of the $\pendingcalls$ value from an environment;
  \item \SemanticsRuleRef{SetPendingCalls} updates the value of the $\pendingcalls$ value in an environment;
  \item \SemanticsRuleRef{IncrPendingCalls} updates the value of the $\pendingcalls$ value in an environment by incrementing it by one;
  \item \SemanticsRuleRef{DecrPendingCalls} updates the value of the $\pendingcalls$ value in an environment by decrementing it by one;
  \item \SemanticsRuleRef{RemoveLocal} removes a local storage element from a given environment;
  \item \SemanticsRuleRef{ReadIdentifier} creates an \executiongraphterm{} representing a Read Effect for a given identifier;
  \item \SemanticsRuleRef{WriteIdentifier} creates an \executiongraphterm{} representing a Write Effect for a given identifier;
  \item \SemanticsRuleRef{ConcatBitvectors} concatenates a list of bitvector \nativevaluesterm{};
  \item \SemanticsRuleRef{ReadFromBitvector} slices a bitvector \nativevalueterm{};
  \item \SemanticsRuleRef{WriteToBitvector} updates a slice of a bitvector \nativevalueterm{};
  \item \SemanticsRuleRef{GetIndex} reads a vector \nativevalueterm{} at a given index;
  \item \SemanticsRuleRef{SetIndex} updates a vector \nativevalueterm{} at a given index;
  \item \SemanticsRuleRef{GetField} reads a given field in a \nativevalueterm{} record;
  \item \SemanticsRuleRef{SetField} updates a given field in a \nativevalueterm{} record;
  \item \SemanticsRuleRef{DeclareLocalIdentifier} updates an environment by associating an identifier with a given \nativevalueterm{}
        and also generating an \executiongraphterm{} with a corresponding Write Effect;
  \item \SemanticsRuleRef{DeclareLocalIdentifierM} updates an environment by associating an identifier with a given \nativevalueterm{}
        and also updating a given \executiongraphterm{} with a corresponding Write Effect and an $\asldata$ edge;
  \item \SemanticsRuleRef{DeclareLocalIdentifierMM} updates an environment by associating an identifier with a given \nativevalueterm{}
        and also updating a given \executiongraphterm{} with a corresponding Write Effect and an $\aslpo$ edge.
\end{itemize}

\SemanticsRuleDef{GetPendingCalls}
\RenderRelation{get_pending_calls}
\BackupOriginalRelation{
The function
\[
\getpendingcalls(\overname{\dynamicenvs}{\denv} \aslsep \overname{\Identifier}{\name}) \aslto \overname{\N}{\vs}
\]
retrieves the value associated with $\name$ in $\denv.\pendingcalls$ or $0$ if no value is associated with it.
} % END_OF_BACKUP_RELATION

\ExampleDef{Retrieving the Number of Pending Calls}
In \listingref{CheckRecurseLimit}, the number of pending calls upon the call to \verb|factorial| from \verb|main|
with the environment $\env_0 \eqdef (\tenv_0, \denv_0)$ is $0$.\\
That is, $\getpendingcalls(\denv_0, \factorial) \evalarrow 0$.

\ProseParagraph
define $\vs$ is $0$ if no value is associated with $\name$ in $\denv.\pendingcalls$ and the value bound to
$\name$ in $\denv.\pendingcalls$ otherwise.

\FormallyParagraph
\begin{mathpar}
\inferrule{
  \vs \eqdef \choice{\name \in \dom(\denv.\pendingcalls)}{\denv.\pendingcalls(\name)}{0}
}{
  \getpendingcalls(\denv, \name) \evalarrow \vs
}
\end{mathpar}

\SemanticsRuleDef{SetPendingCalls}
\RenderRelation{set_pending_calls}
\BackupOriginalRelation{
The function
\[
\setpendingcalls(\overname{\globaldynamicenvs}{\genv} \aslsep \overname{\Identifier}{\name} \aslsep \overname{\N}{\vv}) \aslto
\overname{\dynamicenvs}{\newgenv}
\]
updates the value bound to $\name$ in $\genv.\storage$ to $\vv$, yielding the new global dynamic environment $\newgenv$.
} % END_OF_BACKUP_RELATION

\ExampleDef{Setting the Number of Pending Calls}
In \listingref{CheckRecurseLimit}, the number of pending calls upon the call to \verb|factorial| from \verb|main|
with the environment $\env_0 \eqdef (\tenv_0, \denv_0)$ is $0$, and it is then set to $1$.
That is, \\
$\setpendingcalls(\denv_0, \factorial, 1) \evalarrow \denv_1$
and\\
$\getpendingcalls(\denv_1, \factorial) \evalarrow 1$.

\ProseParagraph
define $\newdenv$ as $\genv$ updated to bind $\name$ to $\vv$ in $\genv.\pendingcalls$.

\FormallyParagraph
\begin{mathpar}
\inferrule{}{
  \setpendingcalls(\genv, \name, \vv) \evalarrow \overname{\genv.\pendingcalls[\name\mapsto\vv]}{\newgenv}
}
\end{mathpar}

\SemanticsRuleDef{IncrPendingCalls}
\RenderRelation{incr_pending_calls}
\BackupOriginalRelation{
The function
\[
\incrpendingcalls(\overname{\globaldynamicenvs}{\genv} \aslsep \overname{\Identifier}{\name}) \aslto
\overname{\globaldynamicenvs}{\newgenv}
\]
increments the value associated with $\name$ in $\genv.\pendingcalls$, yielding the updated global dynamic environment $\newgenv$.
} % END_OF_BACKUP_RELATION

\ExampleDef{Incrementing the Number of Pending Calls}
In \listingref{CheckRecurseLimit}, the number of pending calls upon the call to \verb|factorial| from \verb|main|
with the environment $\env_0 \eqdef (\tenv_0, \denv_0)$ is $0$, and it is incremented to $1$.
That is, \\
$\incrpendingcalls(\denv_0, \factorial) \evalarrow \denv_1$
and\\
$\getpendingcalls(\denv_1, \factorial) \evalarrow 1$.

\ProseParagraph
\AllApply
\begin{itemize}
  \item applying $\getpendingcalls$ to $\name$ in $(\genv, \emptyfunc)$ yields $\vprev$;
  \item applying $\setpendingcalls$ to $\name$ and $\vprev + 1$ in $\genv$ yields $\newgenv$.
\end{itemize}

\FormallyParagraph
\begin{mathpar}
\inferrule{
  \getpendingcalls((\genv, \emptyfunc), \name) \evalarrow \vprev\\
  \setpendingcalls(\genv, \name, \vprev + 1) \evalarrow \newgenv
}{
  \incrpendingcalls(\genv, \name) \evalarrow \newgenv
}
\end{mathpar}

\SemanticsRuleDef{DecrPendingCalls}
\RenderRelation{decr_pending_calls}
\BackupOriginalRelation{
The function
\[
\decrpendingcalls(\overname{\globaldynamicenvs}{\genv} \aslsep \overname{\Identifier}{\name}) \aslto
\overname{\globaldynamicenvs}{\newgenv}
\]
decrements the value associated with $\name$ in $\genv.\pendingcalls$, yielding the updated global dynamic environment $\newgenv$.
It is assumed that $\getpendingcalls((\genv, \emptyfunc), \name)$ yields a positive value.
} % END_OF_BACKUP_RELATION

\ExampleDef{Decrementing the Number of Pending Calls}
In \listingref{CheckRecurseLimit}, the number of pending calls upon returning from \verb|main| to \verb|factorial|
with the environment $\env_1 \eqdef (\tenv_1, \denv_1)$ is $1$, and it is decremented to $0$.
That is, \\
$\decrpendingcalls(\denv_1, \factorial) \evalarrow \denv_0$
and\\
$\getpendingcalls(\denv_0, \factorial) \evalarrow 0$.

\ProseParagraph
\AllApply
\begin{itemize}
  \item applying $\getpendingcalls$ to $\name$ in $(\genv, \emptyfunc)$ yields $\vprev$;
  \item applying $\setpendingcalls$ to $\name$ and $\vprev - 1$ in $\genv$ yields $\newgenv$.
\end{itemize}

\FormallyParagraph
\begin{mathpar}
\inferrule{
  \getpendingcalls((\genv, \emptyfunc), \name) \evalarrow \vprev\\
  \setpendingcalls(\genv, \name, \vprev - 1) \evalarrow \newgenv
}{
  \decrpendingcalls(\genv, \name) \evalarrow \newgenv
}
\end{mathpar}

\SemanticsRuleDef{RemoveLocal}
\ProseParagraph
\RenderRelation{remove_local}
\BackupOriginalRelation{
The relation
\[
  \removelocal(\overname{\envs}{\env} \aslsep \overname{\Identifier}{\name}) \;\aslrel\; \overname{\envs}{\newenv}
\]
removes the binding of the identifier $\name$ from the local storage of the environment $\env$,
yielding the environment $\newenv$.
} % END_OF_BACKUP_RELATION

This relation is used to maintain the invariant that the dynamic environment
maintains bindings only for identifiers that are in scope. That is, only identifiers
that would be in the local dynamic environment.
%
This invariant is not necessary for correctness.
Removing identifiers from the dynamic environment does not have any
observable effect.

\ExampleDef{Removing Local Storage Elements from Dynamic Environments}
In \listingref{RemoveLocal}, the local storage element \verb|i| is removed after the (only) \forstatementterm,
and the local storage element \verb|exn| is removed after the (only) \trystatementterm.

\ASLListing{Removing local storage elements from dynamic environments}{RemoveLocal}{\semanticstests/SemanticsRule.RemoveLocal.asl}
The output to the console is the following:
% CONSOLE_BEGIN aslref \semanticstests/SemanticsRule.RemoveLocal.asl
\begin{Verbatim}[fontsize=\footnotesize, frame=single]
i=0
i=11/2
exn=TRUE
\end{Verbatim}
% CONSOLE_END

\AllApply
\begin{itemize}
  \item $\env$ consists of the \staticenvironmentterm{} $\tenv$ and dynamic environment $\denv$;
  \item $\newenv$ consists of the \staticenvironmentterm{} $\tenv$ and the dynamic environment
  with the same global component as $\denv$ --- $G^\denv$, and local component $L^\denv$,
  with the identifier $\name$ removed from its domain.
\end{itemize}

\FormallyParagraph
(Recall that $[\name\mapsto\bot]$ means that $\name$ is not in the domain of the resulting function.)
\begin{mathpar}
  \inferrule{
    \env \eqname (\tenv, (G^\denv, L^\denv))\\
    \newenv \eqdef (\tenv, (G^\denv, L^\denv[\name \mapsto \bot]))
  }
  {
    \removelocal(\env, \name) \evalarrow \newenv
  }
\end{mathpar}

\SemanticsRuleDef{ReadIdentifier}
\ProseParagraph
\RenderRelation{read_identifier}
\BackupOriginalRelation{
The relation
\[
  \readidentifier(\overname{\Identifier}{\name}\aslsep\overname{\nativevalue}{\vv}) \;\aslrel\; \XGraphs
\]
creates an \executiongraphterm{} that represents the reading of the value $\vv$ into a storage element
given by the identifier $\name$.
The result is an execution graph containing a single Read Effect,
which denotes reading from $\name$.
%
The value $\vv$ is ignored, as execution graphs do not contain values.
} % END_OF_BACKUP_RELATION

\ExampleDef{The Effect of Reading from an Identifier}
In \listingref{ReadIdentifier}, the first iteration of the (only) \forstatementterm{}
generates (see \SemanticsRuleRef{EvalFor}) the following transition at the first iteration:
\[
\readidentifier(\vi, \nvint(0)) \evalarrow \ReadEffect(\vi) \enspace.
\]

\ASLListing{A simple for loop}{ReadIdentifier}{\semanticstests/SemanticsRule.ReadIdentifier.asl}

\FormallyParagraph
\begin{mathpar}
\inferrule{}
{
  \readidentifier(\name, \vv) \evalarrow \ReadEffect(\name)
}
\end{mathpar}

\SemanticsRuleDef{WriteIdentifier}
\ProseParagraph
\RenderRelation{write_identifier}
\BackupOriginalRelation{
The relation
\[
  \writeidentifier(\overname{\Identifier}{\name}\aslsep\overname{\nativevalue}{\vv}) \;\aslrel\; \XGraphs
\]
creates an \executiongraphterm{} that represents the writing of the value $\vv$ into
the storage element given by an identifier $\name$.
The result is an execution graph containing a single Write Effect,
which denotes writing into $\name$.
%
The value $\vv$ is ignored, as execution graphs do not contain values.
} % END_OF_BACKUP_RELATION

\ExampleDef{The Effect of Writing to an Identifier}
In \listingref{ReadIdentifier}, the first iteration of the (only) \forstatementterm{}
generates (see \SemanticsRuleRef{EvalFor}) the following transition at the first iteration,
as the value of the index variable $\vi$ is incremented from $\nvint(0)$ to $\nvint(1)$:
\[
\writeidentifier(\vi, \nvint(1)) \evalarrow \WriteEffect(\vi) \enspace.
\]

\FormallyParagraph
\begin{mathpar}
\inferrule{}
{
  \writeidentifier(\name, \vv) \evalarrow \WriteEffect(\name)
}
\end{mathpar}

\SemanticsRuleDef{ConcatBitvectors}
\RenderRelation{concat_bitvectors}
\BackupOriginalRelation{
The relation
\[
  \concatbitvectors(\overname{\KleeneStar{\tbitvector}}{\vvs}) \;\aslrel\; \overname{\tbitvector}{\newvs}
\]
transforms a (possibly empty) list of bitvector \nativevaluesterm{}$\vvs$ into a single bitvector
$\newvs$.
} % END_OF_BACKUP_RELATION

\ExampleDef{Concatenating Bitvectors}
The specification in \listingref{ConcatBitvectors} shows examples of concatenating
the \bitvectortypeterm{} fields of a \collectiontypeterm{} and of a \recordtypeterm.
\ASLListing{Concatenating bitvectors}{ConcatBitvectors}{\semanticstests/SemanticsRule.ConcatBitvectors.asl}

\ProseParagraph
\OneApplies
\begin{itemize}
  \item \AllApplyCase{empty}
  \begin{itemize}
    \item \Proseemptylist{$\vvs$};
    \item \Proseeqdef{$\newvs$}{the \nativevalueterm{} bitvector for the empty sequence of bits}.
  \end{itemize}

  \item \AllApplyCase{non\_empty}
  \begin{itemize}
    \item $\vvs$ is a \Proselist{$\vv$}{$\vvs'$};
    \item view $\vv$ as the \nativevalueterm{} bitvector for the sequence of bits $\bv$;
    \item applying $\concatbitvectors$ to $\vvs'$ yields the
          \nativevalueterm{} bitvector for the sequence of bits $\bv$;
    \item \Proseeqdef{$\vres$}{the concatenation of $\bv$ and $\bv'$};
    \item \Proseeqdef{$\newvs$}{the \nativevalueterm{} bitvector for sequence of bits $\vres$}.
  \end{itemize}
\end{itemize}

Define $\newvs$ as the concatenation of bitvectors listed in $\vvs$.

\FormallyParagraph
\begin{mathpar}
\inferrule[empty]{}
{
  \concatbitvectors(\overname{\emptylist}{\vvs}) \evalarrow \overname{\nvbitvector(\emptylist)}{\newvs}
}
\end{mathpar}

\begin{mathpar}
\inferrule[non\_empty]{
  \vvs = [\vv] \concat \vvs'\\
  \vv\eqname\nvbitvector(\bv)\\
  \concatbitvectors(\vvs') \evalarrow \nvbitvector(\bv')\\
  \vres \eqdef \bv \concat \bv'
}{
  \concatbitvectors(\vvs) \evalarrow \nvbitvector(\vres)
}
\end{mathpar}

\SemanticsRuleDef{SlicesToPositions}
The helper type \RenderType{slice_as_pair} expresses a slice via a pair of integers
where the first integer is the starting position and the second integer is the length of the slice.

\RenderRelation{slices_to_positions}
\BackupOriginalRelation{
The relation
\[
  \slicestopositions(\overname{\KleenePlus{(\overname{\tint}{\vs}\times\overname{\tint}{\vl})}}{\slices}) \;\aslrel\;
  \overname{\KleeneStar{\N}}{\positions} \cup\ \TDynError
\]
returns the list of positions (indices) specified by the slices $\slices$,
if all slices consist of only non-negative integers.
\ProseOtherwiseDynamicError
} % END_OF_BACKUP_RELATION

\ExampleDef{Converting Slices to Lists of Indices}
In \listingref{SlicesToPositions}, the slices \verb|1+:3, 7:5|
are converted into the list of position $3, 2, 1, 7, 6, 5$.
\ASLListing{Converting slices to lists of indices}{SlicesToPositions}{\semanticstests/SemanticsRule.SlicesToPositions.asl}

The specification in \listingref{SlicesToPositions-bad} terminates with a \dynamicerrorterm,
since the slice \verb|from+:6| evaluates to $(\nvint(-1), \nvint(6))$ and the starting position $-1$
is illegal for a slice.
\ASLListing{An illegal slice}{SlicesToPositions-bad}{\semanticstests/SemanticsRule.SlicesToPositions.bad.asl}

% Transliteration note: there are two slices_to_positions functions:
% the one in Native.ml, which calls the one in ASTUtils.ml.
% This rule transliterates the version in Native.ml, inlining the version
% in ASTUtils.ml.

\ProseParagraph
\OneApplies
\begin{itemize}
  \item \AllApplyCase{empty}
  \begin{itemize}
    \item $\slices$ is empty;
    \item \Proseeqdef{$\positions$}{the empty list}.
  \end{itemize}

  \item \AllApplyCase{non\_empty}
  \begin{itemize}
    \item $\slices$ is the \Proselist{the range starting at $\vs$ of length $\vl$}{$\slicestwo$};
    \item checking that both $\vs$ and $\vl$ are non-negative yields $\True$\ProseTerminateAs{\BadIndex};
    \item \Proseeqdef{$\positionsone$}{the range of indices starting at $\vs + \vl - 1$ and going down to $\vs$};
    \item applying $\slicestopositions$ to $\slicestwo$ yields $\positionstwo$\ProseOrError;
    \item \Proseeqdef{$\positions$}{the concatenation of $\slicesone$ and $\slicestwo$}.
  \end{itemize}
\end{itemize}

\FormallyParagraph
\begin{mathpar}
\inferrule[empty]{}{
  \slicestopositions(\overname{\emptylist}{\slices}) \evalarrow \overname{\emptylist}{\positions}
}
\end{mathpar}

\begin{mathpar}
\inferrule[non\_empty]{
  \decheck(\vs \geq 0 \land \vl \geq 0, \BadIndex) \evalarrow \DynErrorConfig\\
  \positionsone \eqdef (\vs + \vl - 1)..\vs\\
  \slicestopositions(\slicestwo) \evalarrow \positionstwo \OrDynError
}{
  \slicestopositions(\overname{[(\nvint(\vs), \nvint(\vl))] \concat \slicestwo}{\slices}) \evalarrow
  \overname{\positionsone \concat \positionstwo}{\positions}
}
\end{mathpar}

\SemanticsRuleDef{MaxPosOfSlice}
\RenderRelation{max_pos_of_slice}
\BackupOriginalRelation{
The relation
\[
  \maxposofslice(\overname{\tint}{\vs}\times\overname{\tint}{\vl}) \;\aslrel\;
  (\overname{\tint}{\maxpos})
\]
returns the maximum position specified by the slices $\slices$,
assuming that all slices consist only of non-negative integers.
} % END_OF_BACKUP_RELATION

\ExampleDef{Maximum positions of slices}
The slice \verb|1+:3| has maximum position 3.
The slice \verb|42+:0| has maximum position 42.

\ProseParagraph
\OneApplies
\begin{itemize}
  \item \AllApplyCase{zero\_width}
  \begin{itemize}
    \item $\vl$ is zero;
    \item \Proseeqdef{$\maxpos$}{$\vs$}.
  \end{itemize}

  \item \AllApplyCase{non\_zero\_width}
  \begin{itemize}
    \item $\vl$ is not zero;
    \item \Proseeqdef{$\maxpos$}{$\vs + \vl - 1$}.
  \end{itemize}
\end{itemize}

\FormallyParagraph
\begin{mathpar}
\inferrule[zero\_width]{
  \vl = 0
}{
  \maxposofslice(\vs, \vl) \evalarrow \vs
}
\end{mathpar}

\begin{mathpar}
\inferrule[non\_zero\_width]{
  \vl \neq 0
}{
  \maxposofslice(\vs, \vl) \evalarrow \vs + \vl - 1
}
\end{mathpar}

\SemanticsRuleDef{ReadFromBitvector}
\RenderRelation{read_from_bitvector}
\BackupOriginalRelation{
The relation
\[
  \readfrombitvector(\overname{\nativevalue}{\vv} \aslsep \overname{\KleeneStar{(\tint\times\tint)}}{\slices}) \;\aslrel\;
  \overname{\tbitvector}{\vnew} \cup \overname{\TDynError}{\DynErrorConfig}
\]
reads from a bitvector $\bv$, or an integer seen as a bitvector, the indices specified by the list of slices $\slices$,
thereby concatenating their values.
} % END_OF_BACKUP_RELATION

Notice that the bits of a bitvector go from the least significant bit being on the right to the most significant bit being on the left,
which is reflected by how the rules list the bits.
The effect of placing the bits in sequence is that of concatenating the results
from all of the given slices.
Also notice that bitvector bits are numbered from 1 and onwards, which is why we add 1 to the indices specified
by the slices when accessing a bit.

\ExampleDef{Reading From a Bitvector}
The specification in \listingref{ReadFromBitvector} shows examples of reading slices from a bitvector
and reading slices from an integer (first converted to a bitvector), followed by the output to the console.

\ASLListing{Reading from a bitvector}{ReadFromBitvector}{\semanticstests/SemanticsRule.ReadFromBitvector.asl}
% CONSOLE_BEGIN aslref \semanticstests/SemanticsRule.ReadFromBitvector.asl
\begin{Verbatim}[fontsize=\footnotesize, frame=single]
empty_bv_slice = 0x, empty_i_slice = 0x
slice_bv = 0x9c5, slice_i = 0x9c5
\end{Verbatim}
% CONSOLE_END

\ExampleDef{Conversion of Integers to Bitvectors}
The specification in \listingref{AsBitvector} shows an example of converting a negative
integer into a bitvector, followed by the output to the console.

\ASLListing{Converting a signed integer into a bitvector}{AsBitvector}{\semanticstests/SemanticsRule.AsBitvector.asl}
% CONSOLE_BEGIN aslref \semanticstests/SemanticsRule.AsBitvector.asl
\begin{Verbatim}[fontsize=\footnotesize, frame=single]
bv = 0xfdf0, bv_i = 0xfdf0
\end{Verbatim}
% CONSOLE_END

\ProseParagraph
\OneApplies
\begin{itemize}
  \item \AllApplyCase{bitvector}
  \begin{itemize}
    \item $\vv$ is a native bitvector for the list of bits $\bv$;
    \item applying $\slicestopositions$ to $\slices$ yields the list $\positions$\ProseOrError;
    \item \Proseeqdef{$\maxpos$}{the maximal position in $\slices$};
    \item let $\bits$ be the sequence of bits $\bv[n] ... \bv[1]$;
    \item checking that $\maxpos$ is less than $n$ yields $\True$\ProseTerminateAs{\BadIndex};
    \item \Proseeqdef{$\newbits$}{the list of bits $\bits[j]$ for every \Proselistrange{$j$}{$\bits$}};
    \item \Proseeqdef{$\vv$}{the native bitvector for the list of bits $\newbits$}.
  \end{itemize}

  \item \AllApplyCase{integer}
  \begin{itemize}
    \item $\vv$ is a native integer for the integer $\vi$;
    \item applying $\slicestopositions$ to $\slices$ yields the list $\positions$\ProseOrError;
    \item \Proseeqdef{$\maxpos$}{the maximal position in $\slices$};
    \item \Proseeqdef{$\bits$}{$\vi$  given as two's complement little endian form of $\maxpos + 1$ bits};
    \item \Proseeqdef{$\newbits$}{the list of bits $\bits[j]$ for every \Proselistrange{$j$}{$\bits$}};
    \item \Proseeqdef{$\vv$}{the native bitvector for the list of bits $\newbits$}.
  \end{itemize}
\end{itemize}

\FormallyParagraph
\begin{mathpar}
\inferrule[bitvector]{
  \slicestopositions(\slices) \evalarrow \positions \OrDynError\\\\
  \vi \in \listrange(\slices): \maxposofslice(\slice[\vi]) \evalarrow \vp_\vi\\\\
  \maxpos \eqdef \max(\ \{ \vi\in\listrange(\slices) : \vp_\vi \}\ )\\\\
  \commonprefixline\\\\
  \bits \eqname \bv[n] ... \bv[1]\\
  \decheck(\maxpos < n, \BadIndex) \evalarrow \OrDynError\\\\
  \commonsuffixline\\\\
  % The following premise is essentially Bitvector.extract_slice
  \newbits \eqdef [\ \vj \in \listrange(\positions): \bits[\vj]\ ]
}{
  \readfrombitvector(\overname{\nvbitvector(\bv)}{\vv}, \slices) \evalarrow \overname{\nvbitvector(\newbits)}{\newv}
}
\end{mathpar}

\begin{mathpar}
\inferrule[integer]{
  \slicestopositions(\slices) \evalarrow \positions \OrDynError\\\\
  \vi \in \listrange(\slices): \maxposofslice(\slice[\vi]) \evalarrow \vp_\vi\\\\
  \maxpos \eqdef \max(\ \{ \vi\in\listrange(\slices) : \vp_\vi \}\ )\\\\
  \commonprefixline\\\\
  \bits \eqdef \vi\text{ given as two's complement little endian form of }(\maxpos + 1)\text{ bits}\\
  \commonsuffixline\\\\
  % The following premise is essentially Bitvector.extract_slice
  \newbits \eqdef [\ \vj \in \listrange(\positions): \bits[\vj]\ ]
}{
  \readfrombitvector(\overname{\nvint(\vi)}{\vv}, \slices) \evalarrow \overname{\nvbitvector(\newbits)}{\newv}
}
\end{mathpar}

\SemanticsRuleDef{WriteToBitvector}
\RenderRelation{write_to_bitvector}
\BackupOriginalRelation{
The relation
\[
  \writetobitvector(\overname{\KleeneStar{(\tint\times\tint)}}{\slices} \aslsep \overname{\tbitvector}{\src} \aslsep \overname{\tbitvector}{\dst})
  \;\aslrel\; \overname{\tbitvector}{\vv} \cup \overname{\TDynError}{\DynErrorConfig}
\]
overwrites the bits of $\dst$ at the positions given by $\slices$ with the bits of $\src$.
} % END_OF_BACKUP_RELATION

See \ExampleRef{Writing to a Bitvector}, following the definition of $\writetobitvector$.

\ProseParagraph
\AllApply
\begin{itemize}
  \item $\src$ is a native bitvector consisting of bits $\vs_m \ldots \vs_0$;
  \item $\dst$ is a native bitvector consisting of bits $\vd_n \ldots \vd_0$;
  \item applying $\slicestopositions$ to $\slices$ yields the list of indices $\positions$;
  \item checking that the length of $\positions$ is equal to the length of $\src$ yields $\True$\ProseTerminateAs{\BadIndex};
  \item \Proseeqdef{$\maxpos$}{the maximal position in $\slices$};
  \item checking that $\maxpos$ is less than the length of $\dst$ yields $\True$\ProseTerminateAs{\BadIndex};
  \item view $\positions$ as the list $I_m \ldots I_0$;
  \item define the function $\bitfunc$ as mapping an index $i$ in $0$ to $n$ to
        $\vs_j$, if there exists an index $I_j$ in $\positions$ such that $I_j$ is equal to $i$,
        and $\vd_i$, otherwise.
  \item \Proseeqdef{$\vbits$}{the list of bits defined as
        $\bitfunc(n)\ldots\bitfunc(0)$};
  \item \Proseeqdef{$\vv$}{the native bitvector for $\vbits$}.
\end{itemize}

\FormallyParagraph
\begin{mathpar}
\inferrule{
  \src \eqname \nvbitvector(\vs_m \ldots \vs_0)\\
  \dst \eqname \nvbitvector(\vd_n \ldots \vd_0)\\
  \slicestopositions(\slices) \evalarrow \positions \OrDynError\\\\
  \decheck(\listlen{\positions} = m + 1, \BadIndex) \evalarrow \OrDynError\\\\
  \vi \in \listrange(\slices): \maxposofslice(\slice[\vi]) \evalarrow \vp_\vi\\\\
  \maxpos \eqdef \max(\ \{ \vi\in\listrange(\slices) : \vp_\vi \}\ )\\\\
  \decheck(\maxpos < n + 1, \BadIndex) \evalarrow \OrDynError\\\\
  \positions \eqname I_m \ldots I_0 \\
  {\bitfunc = \lambda i \in 0..n.\left\{ \begin{array}{ll}
    \vs_j & \exists j\in 1..m.\ i = I_j\\
    \vd_i & \text{otherwise}
  \end{array} \right.}\\\\
  \vbits\eqdef [i=n..0: \bitfunc(i)]\\
}{
  \writetobitvector(\slices, \src, \dst) \evalarrow \overname{\nvbitvector(\vbits)}{\vv}
}
\end{mathpar}

\ExampleDef{Writing to a Bitvector}
In reference to \listingref{semantics-leslice}, we have the following application of the current rule:
\begin{mathpar} % SUPPRESS_TEXTTT_LINTER
\inferrule{
  \src = \overname{0}{\vs_5}\overname{0}{\vs_4}\overname{0}{\vs_3}\overname{0}{\vs_2}\overname{0}{\vs_1}\overname{0}{\vs_0}\\
  \dst = \overname{1}{\vd_7}\overname{1}{\vd_6}\overname{1}{\vd_5}\overname{1}{\vd_4}\overname{1}{\vd_3}\overname{1}{\vd_2}\overname{1}{\vd_1}\overname{1}{\vd_0}\\
  \slicestopositions(8, [\overname{(0, 4)}{\texttt{3:0}}, \overname{(6, 2)}{\texttt{7:6}}]) \evalarrow
  [3, 2, 1, 0, 7, 6]\\
  \positions \eqdef [\overname{3}{I_5}, \overname{2}{I_4}, \overname{1}{I_3}, \overname{0}{I_2}, \overname{7}{I_1}, \overname{6}{I_0}]\\
  {\bitfunc = \lambda i \in 0..7.\left\{ \begin{array}{ll}
    \vs_j & \exists j\in 1..5.\ i = I_j\\
    \vd_i & \text{otherwise}
  \end{array} \right.}\\
  \vbits \eqdef \bitfunc(7)\ \bitfunc(6)\ \bitfunc(5)\ \bitfunc(4)\ \bitfunc(3)\ \bitfunc(2)\ \bitfunc(1)\ \bitfunc(0)
}{
  {
  \begin{array}{r}
    \writetobitvector(
      [\overname{(0, 4)}{\texttt{3:0}}, \overname{(6, 2)}{\texttt{7:6}}],
      \overname{\nvbitvector(000000)}{\src},
      \overname{\nvbitvector(11111111)}{\dst}) \evalarrow\\
    \nvbitvector(
      \overname{0}{\vs_1}
      \overname{0}{\vs_0}
      \overname{1}{\vd_5}
      \overname{1}{\vd_4}
      \overname{0}{\vs_5}
      \overname{0}{\vs_4}
      \overname{0}{\vs_3}
      \overname{0}{\vs_2})
  \end{array}
  }
}
\end{mathpar}

\SemanticsRuleDef{GetIndex}
\RenderRelation{get_index}
\BackupOriginalRelation{
The relation
\[
  \getindex(\overname{\N}{\vi} \aslsep \overname{\tvector}{\vvec}) \;\aslrel\;
  (\overname{\tvector}{\vr} \cup \overname{\TDynError}{\DynamicErrorVal{\BadIndex}})
\]
reads the value $\vr$ from the vector of values $\vvec$ at the index $\vi$.
\ProseOtherwiseDynamicError
} % END_OF_BACKUP_RELATION

\ExampleDef{Reading a Vector at an Index}
The specification in \listingref{GetIndex} shows examples of accessing indices of native vectors:
\begin{itemize}
  \item evaluating \verb|assert arr[[3]] == 3| involves \\
        $\getindex(3, \NVVector(\nvint(0), \nvint(3), \nvint(0))) \evalarrow \nvint(3)$;
  \item evaluating \verb|assert (5, 7).item0 == 5| involves \\
        $\getindex(0, \NVVector(\nvint(5), \nvint(7))) \evalarrow \nvint(5)$;
  \item evaluating \verb|(5, 7) as (integer, integer)| involves \\
        $\getindex(0, \NVVector(\nvint(5), \nvint(7))) \evalarrow \nvint(5)$ and\\
        $\getindex(1, \NVVector(\nvint(5), \nvint(7))) \evalarrow \nvint(7)$ to check the asserting type conversion.
\end{itemize}
\ASLListing{Reading a vector at an index}{GetIndex}{\semanticstests/SemanticsRule.GetIndex.asl}

\ProseParagraph
\OneApplies
\begin{itemize}
  \item \AllApplyCase{ok}
  \begin{itemize}
    \item $\vi$ is greater than or equal to zero and less than the number of elements in $\vvec$;
    \item \Proseeqdef{$\vr$}{the element of $\vvec$ at index $\vi$}.
  \end{itemize}

  \item \AllApplyCase{error}
  \begin{itemize}
    \item $\vi$ is less than zero or greater than or equal to the number of elements in $\vvec$;
    \item the result is the \dynamicerrorterm{} for an out of bounds index (\BadIndex).
  \end{itemize}
\end{itemize}

\FormallyParagraph
\begin{mathpar}
\inferrule[ok]{
  0 \leq \vi < \listlen{\vvec}\\
}{
  \getindex(\vi, \vvec) \evalarrow \overname{\vv[\vi]}{\vr}
}
\end{mathpar}

\begin{mathpar}
\inferrule[error]{
  \vi < 0 \lor \vi \geq \listlen{\vvec}\\
}{
  \getindex(\vi, \vvec) \evalarrow \DynamicErrorVal{\BadIndex}
}
\end{mathpar}

\SemanticsRuleDef{SetIndex}
\RenderRelation{set_index}
\BackupOriginalRelation{
The relation
\[
  \setindex(\overname{\N}{\vi} \aslsep \overname{\nativevalue}{\vv} \aslsep \overname{\tvector}{\vvec}) \;\aslrel\;
  (\overname{\tvector}{\vres} \cup \overname{\TDynError}{\DynamicErrorVal{\BadIndex}})
\]
overwrites the value at the given index $\vi$ in a vector of values $\vvec$ with the new value $\vv$.
\ProseOtherwiseDynamicError
} % END_OF_BACKUP_RELATION

\ExampleDef{Writing a Vector at an Index}
In \listingref{GetIndex}, evaluating the statement \verb|arr[[1]] = 3;|
involves the following premise:
\[
\begin{array}{r}
\setindex(1, \nvint(3), \NVVector(\nvint(0), \nvint(0), \nvint(0))) \evalarrow\\
\NVVector(\nvint(0), \nvint(3), \nvint(0)) \enspace.
\end{array}
\]

\ProseParagraph
\OneApplies
\begin{itemize}
  \item \AllApplyCase{ok}
  \begin{itemize}
    \item $\vi$ is greater than or equal to zero and less than the number of elements in $\vvec$;
    \item $\vvec$ is the sequence $\vu_{0..k}$;
    \item $\vres$ is the sequence of values identical to $\vvec$,
          except that at index $\vi$ the value is $\vv$.
  \end{itemize}

  \item \AllApplyCase{error}
  \begin{itemize}
    \item $\vi$ is less than zero or greater than or equal to the number of elements in $\vvec$;
    \item the result is the \dynamicerrorterm{} for an out of bounds index (\BadIndex).
  \end{itemize}
\end{itemize}

\FormallyParagraph
\begin{mathpar}
\inferrule[ok]{
  0 \leq \vi < \listlen{\vvec}\\
  \vvec \eqname \vu_{0..k}\\
  \vres = \vw_{0..k}\\
  \vv = \vw_{\vi} \\
  j \in \{0..k\} \setminus \{\vi\}.\ \vw_{j} = \vu_j\\
}{
  \setindex(\vi, \vv, \vvec) \evalarrow \vres
}
\end{mathpar}

\begin{mathpar}
\inferrule[error]{
  \vi < 0 \lor \vi \geq \listlen{\vvec}
}{
  \setindex(\vi, \vv, \vvec) \evalarrow \DynamicErrorVal{\BadIndex}
}
\end{mathpar}

\SemanticsRuleDef{GetField}
\ProseParagraph
\RenderRelation{get_field}
\BackupOriginalRelation{
The relation
\[
  \getfield(\overname{\Identifier}{\name} \aslsep \overname{\trecord}{\record}) \;\aslrel\; \nativevalue
\]
retrieves the value corresponding to the field name $\name$ from the record value $\record$.
} % END_OF_BACKUP_RELATION

\ExampleDef{Reading a Record Field}
In \listingref{GetField}, there are the following examples of reading a record field:
\begin{itemize}
  \item evaluating the expression \verb|color_to_int[[RED]]| involves\\
        $\getfield(\RED, \NVRecord(\RED\mapsto \nvint(0), \GREEN\mapsto \nvint(1), \BLUE\mapsto \nvint(2))) \evalarrow \nvint(0)$;
  \item evaluating the expression \verb|r.RED| involves\\
        $\getfield(\RED, \NVRecord(\RED\mapsto \nvint(0), \GREEN\mapsto \nvint(1), \BLUE\mapsto \nvint(2))) \evalarrow \nvint(0)$.
\end{itemize}
\ASLListing{Reading a record field}{GetField}{\semanticstests/SemanticsRule.GetField.asl}

\FormallyParagraph
\begin{mathpar}
\inferrule{
  \record \eqname \NVRecord(\fieldmap)
}{
  \getfield(\name, \record) \evalarrow \fieldmap(\name)
}
\end{mathpar}
The typechecker ensures, via \TypingRuleRef{EGetRecordField}, that the field $\name$ exists in $\record$.

\SemanticsRuleDef{SetField}
\ProseParagraph
\RenderRelation{set_field}
\BackupOriginalRelation{
The function
\[
  \setfield(\overname{\Identifier}{\name} \aslsep \overname{\nativevalue}{\vv} \aslsep \overname{\trecord}{\record}) \;\aslto\; \trecord
\]
overwrites the value corresponding to the field name $\name$ in the record value $\record$ with the value $\vv$.
} % END_OF_BACKUP_RELATION

\ExampleDef{Writing to a Record Field}
In \listingref{GetField}, there are the following examples of writing to a record field:
\begin{itemize}
  \item evaluating the statement \verb|color_to_int[[GREEN]] = 1;| involves\\
        $
        \begin{array}{r}
        \setfield\left(
          \begin{array}{l}
          \GREEN, \\
          \nvint(1),\\
          \NVRecord(\RED\mapsto \nvint(0), \GREEN\mapsto \nvint(0), \BLUE\mapsto \nvint(0))
          \end{array}
          \right) \evalarrow\\
        \NVRecord(\RED\mapsto \nvint(0), \GREEN\mapsto \nvint(1), \BLUE\mapsto \nvint(0)) \enspace;
        \end{array}
        $
  \item evaluating the statement \verb|r.BLUE = -1;| involves\\
        $
        \begin{array}{r}
        \setfield\left(
          \begin{array}{l}
          \BLUE, \\
          \nvint(-1),\\
          \NVRecord(\RED\mapsto \nvint(0), \GREEN\mapsto \nvint(1), \BLUE\mapsto \nvint(2))
          \end{array}
          \right) \evalarrow\\
        \NVRecord(\RED\mapsto \nvint(0), \GREEN\mapsto \nvint(1), \BLUE\mapsto \nvint(-1)) \enspace.
        \end{array}
        $
\end{itemize}

\FormallyParagraph
\begin{mathpar}
\inferrule{
  \record \eqname \NVRecord(\fieldmap)\\
  \fieldmapp \eqdef \fieldmap[\name\mapsto\vv]
}{
  \setfield(\name, \vv, \record) \evalarrow \NVRecord(\fieldmapp)
}
\end{mathpar}
The typechecker ensures that the field $\name$ exists in $\record$.

\SemanticsRuleDef{DeclareLocalIdentifier}
\ProseParagraph
\RenderRelation{declare_local_identifier}
\BackupOriginalRelation{
The relation
\[
  \declarelocalidentifier(\overname{\envs}{\env} \aslsep \overname{\Identifier}{\name} \aslsep \overname{\nativevalue}{\vv}) \;\aslrel\;
  (\overname{\envs}{\newenv}\times\overname{\XGraphs}{\vg})
\]
associates $\vv$ to $\name$ as a local storage element in the environment $\env$ and
returns the updated environment $\newenv$ with the execution graph consisting of a Write Effect to $\name$.
} % END_OF_BACKUP_RELATION

\ExampleDef{Evaluating Local Declarations}
In \listingref{semantics-levar}, evaluating \verb|var x: integer = 3;|
binds \verb|x| to $\nvint(3)$ in the environment where \verb|x| is unbound
as well as producing the \executiongraphterm{} $\WriteEffect(\vx)$.

\FormallyParagraph
\begin{mathpar}
  \inferrule{
    \vg \eqdef \WriteEffect(\name)\\
    \env \eqname (\tenv, (G^\denv, L^\denv))\\
    \newenv \eqdef (\tenv, (G^\denv, L^\denv[\name\mapsto \vv]))
  }
  { \declarelocalidentifier(\env, \name, \vv) \evalarrow (\newenv, \vg)  }
\end{mathpar}

\SemanticsRuleDef{DeclareLocalIdentifierM}
\RenderRelation{declare_local_identifier_m}
\BackupOriginalRelation{
The relation
\[
  \declarelocalidentifierm(\overname{\envs}{\env} \aslsep
   \overname{\Identifier}{\vx} \aslsep
   \overname{(\overname{\nativevalue}{\vv}\times\overname{\XGraphs}{\vg})}{\vm}) \;\aslrel\;
  (\overname{\envs}{\newenv} \times \overname{\XGraphs}{\newg})
\]
declares the local identifier $\vx$ in the environment $\env$, in the context
of the value-graph pair $(\vv, \vg)$, yielding a pair consisting
of the environment $\newenv$ and \executiongraphterm{} $\newg$.
} % END_OF_BACKUP_RELATION

See \ExampleRef{Evaluating Local Declarations}.

\ProseParagraph
\AllApply
\begin{itemize}
  \item \newenv\ is the environment $\env$ modified to declare the variable $\vx$ as a local storage element;
  \item $\vgone$ is the execution graph resulting from the declaration of $\vx$;
  \item \Proseeqdef{$\newg$}{\executiongraphterm{} resulting from the ordered composition
        of $\vg$ and $\vgone$ with the $\asldata$ edge}.
\end{itemize}

\FormallyParagraph
\begin{mathpar}
  \inferrule{
    \vm \eqname (\vv, \vg)\\
    \declarelocalidentifier(\env, \vx, \vv) \evalarrow (\newenv, \vgone)\\
    \newg \eqdef \ordered{\vg}{\asldata}{\vgone}
  }
  {
    \declarelocalidentifierm(\env, \vx, \vm) \evalarrow (\newenv, \newg)
  }
\end{mathpar}

\SemanticsRuleDef{DeclareLocalIdentifierMM}
\RenderRelation{declare_local_identifier_mm}
\BackupOriginalRelation{
The relation
\[
  \declarelocalidentifiermm(\overname{\envs}{\env} \aslsep
   \overname{\Identifier}{\vx} \aslsep
   \overname{(\overname{\nativevalue}{\vv}\times\overname{\XGraphs}{\vg})}{\vm}) \;\aslrel\;
  (\overname{\envs}{\newenv} \times \overname{\XGraphs}{\newg})
\]
declares the local identifier $\vx$ in the environment $\env$,
in the context of the value-graph pair $(\vv, \vg)$,
yielding a pair consisting of an environment $\newenv$
and an \executiongraphterm{} $\vgtwo$.
} % END_OF_BACKUP_RELATION

See \ExampleRef{Evaluating Local Declarations}.

\ProseParagraph
\AllApply
\begin{itemize}
  \item \newenv\ is the environment $\env$ modified to declare the variable $\vx$ as a local storage element;
  \item $\vgone$ is the execution graph resulting from the declaration of $\vx$;
  \item \Proseeqdef{$\newg$}{the execution graph resulting from the ordered composition
        of $\vg$ and $\vgone$ with the $\aslpo$ edge.}
\end{itemize}

\FormallyParagraph
\begin{mathpar}
\inferrule{
  \declarelocalidentifierm(\env, \vm) \evalarrow (\newenv, \vgone)\\
  \newg \eqdef \ordered{\vg}{\aslpo}{\vgone}
}{
  \declarelocalidentifiermm(\env, \vx, \vm) \evalarrow (\newenv, \newg)
}
\end{mathpar}
